
\documentclass[12pt, a4paper]{article}

\usepackage{biblatex}
\bibliography{bibliography}

\setlength{\oddsidemargin}{0.5cm}
\setlength{\evensidemargin}{0.5cm}
\setlength{\topmargin}{-1.6cm}
\setlength{\leftmargin}{0.5cm}
\setlength{\rightmargin}{0.5cm}
\setlength{\textheight}{24.00cm} 
\setlength{\textwidth}{15.00cm}
\parindent 0pt
\parskip 5pt
\pagestyle{plain}

\title{Research Proposal}
\author{}
\date{}

\newcommand{\namelistlabel}[1]{\mbox{#1}\hfil}
\newenvironment{namelist}[1]{%1
\begin{list}{}
    {
        \let\makelabel\namelistlabel
        \settowidth{\labelwidth}{#1}
        \setlength{\leftmargin}{1.1\labelwidth}
    }
  }{%1
\end{list}}

\begin{document}
\maketitle

\begin{namelist}{xxxxxxxxxxxx}
\item[{\bf Title:}]
	Optional Static Type Checking for Clojure
\item[{\bf Author:}]
	Ambrose Bonnaire-Sergeant
\item[{\bf Supervisor:}]
	Rowan Davies
\end{namelist}

\section*{Background} 

% In this section you should give some background to your
% research area. What is the problem you are tackling, and why is it
% worthwhile solving? Who has already done some work in this area,
% and what have they achieved?

Dynamic languages are designed to be convenient for
writing programs quickly, and aspire to get out of the programmer's
way as much as possible. When programs grow large and stabilize,
some features of static languages are missed, specifically static
type checking. 

Having a dynamic language that supports static properties 
has many potential practical benefits. 
Module-by-module porting of existing untyped code to a typed language 
is the largest benefit cited by Tobin-Hochstadt \cite{TypedScheme:2010}.
He developed Typed Scheme
\cite{TypedScheme:2010} 
to safely and incrementally port existing Scheme code, a dynamic language, to
Typed Scheme, a static language.

Clojure is a dynamic, functional language with implementations for the Java Virtual
Machine and Common Language Runtime, and compiling to Javascript. Clojure
is also a Lisp, which makes it a good candidate to test the ideas developed in
Typed Scheme \cite{TypedScheme:2010}.

\section*{Aim} 

% Now state explicitly the hypothesis you aim to
% test. Make references to the items listed in the Reference section
% that back up your arguments for why this is a reasonable
% hypothesis to test, for example the work of Knuth~\cite{knuth}.
% Explain what you expect will be accomplished by undertaking this
% particular project.  Moreover, is it likely to have any other
% applications?

My goal is to develop a prototype optional static type system for Clojure, 
eventually intended for practical use, to be provided as a library.

It will be based on the lessons learnt throught the development
of Typed Scheme, and as a response to Tobin-Hochstadt's \cite{TypedScheme:2010}
suggestion to add type systems to existing dynamic languages.

There are several challenges to creating a satisfactory type system for Clojure.

Multimethods play a significant role in Clojure, and a satisfactory
type system for Clojure should understand them to some degree.
There is some experience statically typing multimethods
\cite{Millstein02modularstatically}
, but this is a challenge and will probably not be completed in the
timeframe proposed.

Clojure's core library includes variable-arity functions which
are not easily typed with current static type systems. 
Providing satisfactory types for functions 
like Clojure's 'map', 'filter', and 'reduce' require
support for non-uniform variable arity polymorphism. 
Strickland, Tobin-Hochstadt and Felleisen \cite{Strickland:2009:PVP:1532974.1532978}.
describe their approach for non-uniform variable arity polymorphism, as used
in Typed Racket. It is not clear if including a similar system is 
achievable in the given timeframe, but would be a very useful feature. 

Occurence typing is type checking technique developed for Typed Scheme 
\cite{Tobin-Hochstadt:2008:DIT:1328897.1328486}
and improved for Typed Racket
\cite{Tobin-Hochstadt:2010:LTU:1932681.1863561}.
It helps the type checker understand common programming idioms 
with minimal type annotations.
The general approach of occurence typing appears to fit Clojure well,
but the implementation challenges are not clear. Occurence typing,
or something similar, would be essential to a practical system.

Ensuring vigorous type safety is an important aspect of Typed Scheme 
\cite{TypedScheme:2010} ,
especially when interacting between untyped and typed modules.
I do not expect to concentrate on every combination of cross-module
interaction. In particular, safely using typed code from untyped code
is not important for my needs.

\section*{Method}

% In this section you should outline how you intend to go
% about accomplishing the aims you have set in the previous
% section. Try to break your grand aims down into small,
% achievable tasks. Try to estimate how long you will
% spend on each task, and draw up a timetable for each
% sub-task.

\begin{description}
\item[0.1]
	\begin{itemize}
	\item Concrete types (no type variables)
	\item Union types
	\item Fixed arity function types
	\item Typed deftype
	\item Uniform variable-arity function types
	\item Type checking without refinements or occurance typing
	\end{itemize}
\item[0.2]
	\begin{itemize}
	\item Occurance typing for type inference
	\item Type variables (polymorphic types)
	\end{itemize}
\item[0.3]
	\begin{itemize}
	\item Typed defrecord
	\item All Types annotated for clojure.core library
	\end{itemize}
\item[0.4]
	\begin{itemize}
	\item Manage interactions between typed namespaces
	\end{itemize}
\item[0.5]
	\begin{itemize}
	\end{itemize}
\item[0.6]
	\begin{itemize}
	\item Mutable reference types
	\item Non-uniform variable arity polymorphism
	\end{itemize}

\end{description}
\section*{Software and Hardware Requirements}
% Outline what your specific requirements will be with regard
% to software and hardware, but note that any special requests
% might need to be approved by your supervisor and the Head of
% Department.

Linux environment with Java, git, and maven installed.

% Overall, you should aim to produce roughly a two page document
% (and certainly no more than four pages)
% outlining your plan for the year.

% \begin{thebibliography}{9}
% \bibitem{knuth} D. E. Knuth. {\em The \TeX~book.}\/ Addison-Wesley,
% Reading, Massachusetts, 1984.
% \bibitem{lamport} L. Lamport. {\em \LaTeX~: A Document Preparation
% System}.\/ Addison-Wesley, Reading, Massachusetts, 1986.
% \bibitem{ken} Ken Wessen, Preparing a thesis using \LaTeX~, private
% communication, 1994.
% \bibitem{lamport2} L. Lamport. Document Production: Visual
% or Logical, {\em Notices of the Amer. Maths. Soc.},\/ Vol. 34,
% 1987, pp. 621-624.
% \end{thebibliography}

\printbibliography

\end{document}

