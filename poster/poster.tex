\documentclass[landscape,final,a0paper,fontscale=0.25]{baposter}

\usepackage{url}

\usepackage{graphicx}
\graphicspath{{images/}{../images/}}

\usepackage{listings}
\lstset{ %
  language=Lisp,                % choose the language of the code
  columns=fixed,basewidth=.5em,
  basicstyle=\small\ttfamily,       % the size of the fonts that are used for the code
  %numbers=left,                   % where to put the line-numbers
  %numberstyle=\small\ttfamily,      % the size of the fonts that are used for the line-numbers
  %stepnumber=1,                   % the step between two line-numbers. If it is 1 each line will be numbered
  %numbersep=5pt,                  % how far the line-numbers are from the code
  %backgroundcolor=\color{white},  % choose the background color. You must add \usepackage{color}
  %showspaces=false,               % show spaces adding particular underscores
  showstringspaces=false,         % underline spaces within strings
  %showtabs=false,                 % show tabs within strings adding particular underscores
  frame=single,           % adds a frame around the code
  %tabsize=2,          % sets default tabsize to 2 spaces
  captionpos=t,           % sets the caption-position to bottom
  breaklines=true,        % sets automatic line breaking
  breakatwhitespace=true,    % sets if automatic breaks should only happen at whitespace
  %escapeinside={\%*}{*)},          % if you want to add a comment within your code
}

\begin{document}

\definecolor{lightorange}{rgb}{0.9,0.4,0}
\definecolor{lightestorange}{rgb}{1,0.8,0.5}
\definecolor{darkorange}{rgb}{0.2,0.1,0}
%%
\begin{poster}%
  % Poster Options
  {
  % Show grid to help with alignment
  grid=false,
  % Column spacing
  colspacing=1em,
  % Color style
  bgColorOne=lightestorange,
  bgColorTwo=white,
  borderColor=darkorange,
  headerColorOne=darkorange,
  headerColorTwo=lightorange,
  headerFontColor=white,
  boxColorOne=lightestorange,
  boxColorTwo=lightorange,
  columns=3,
  % Format of textbox
  textborder=faded,
  % Format of text header
  eyecatcher=true,
  headerborder=closed,
  headerheight=0.1\textheight,
%  textfont=\sc, An example of changing the text font
  headershape=roundedright,
  headershade=shadelr,
  headerfont=\Large\bf\textsc, %Sans Serif
  textfont={\setlength{\parindent}{1.5em}},
  boxshade=plain,
%  background=shade-tb,
  background=none,
  linewidth=2pt
  }
  % Eye Catcher
  {\includegraphics{images/1000px-UWA_crest}}
   
  % Title
  { A Practical Optional Type System for Clojure }
  % Authors
  { Ambrose Bonnaire-Sergeant <abonnairesergeant@gmail.com>\\
    \emph{Supervised by Rowan Davies <rowan@csse.uwa.edu.au>}}
  % University logo
  { University of Western Australia }
%%%%%%%%%%%%%%%%%%%%%%%%%%%%%%%%%%%%%%%%%%%%%%%%%%%%%%%%%%%%%%%%%%%%%%%%%%%%%%
%%% Now define the boxes that make up the poster
%%%---------------------------------------------------------------------------
%%% Each box has a name and can be placed absolutely or relatively.
%%% The only inconvenience is that you can only specify a relative position 
%%% towards an already declared box. So if you have a box attached to the 
%%% bottom, one to the top and a third one which should be in between, you 
%%% have to specify the top and bottom boxes before you specify the middle 
%%% box.
%%%%%%%%%%%%%%%%%%%%%%%%%%%%%%%%%%%%%%%%%%%%%%%%%%%%%%%%%%%%%%%%%%%%%%%%%%%%%%

%%%%%%%%%%%%%%%%%%%%%%%%%%%%%%%%%%%%%%%%%%%%%%%%%%%%%%%%%%%%%%%%%%%%%%%%%%%%%%
  \begin{posterbox}[name=problem,column=0,row=0]{Background}
%%%%%%%%%%%%%%%%%%%%%%%%%%%%%%%%%%%%%%%%%%%%%%%%%%%%%%%%%%%%%%%%%%%%%%%%%%%%%%

\section*{Static and Dynamic Types}

Traditionally there has been a clear distinction between 
programming languages that perform type checking at compile-time, and those that do not.
A recent trend is to aim to combine the advantages of both kinds of languages by adding optional static 
type systems to languages without static type checking.

\section*{Clojure}

Clojure is a recent and increasingly popular programming language that a is dynamically typed dialect of Lisp.
Designed to be a pragmatic language, it supports functional programming and immutability by default,
while being hosted on, and providing direct interoperability with popular platforms.
\end{posterbox}

%%%%%%%%%%%%%%%%%%%%%%%%%%%%%%%%%%%%%%%%%%%%%%%%%%%%%%%%%%%%%%%%%%%%%%%%%%%%%%
\begin{posterbox}[name=contribution,column=0,below=problem]{Contributions}
%%%%%%%%%%%%%%%%%%%%%%%%%%%%%%%%%%%%%%%%%%%%%%%%%%%%%%%%%%%%%%%%%%%%%%%%%%%%%%

We take the lessons learnt from several projects to design \emph{Typed Clojure},
an optional type system for Clojure running on the JVM. 
The main contributions are:
\begin{itemize}
  \item We develop a prototype type checker based on Typed Racket, which is able to type check many Clojure idioms
  \item We describe a novel use of \emph{occurrence typing} for eliminating occurrences of Java's \emph{null}
  \item We show how a combination of occurrence typing and intersection types allows accurrate typing checking of
        the most common usages of Clojure's sequence abstraction 
  \item We capture the Clojure idiom of using maps with known keyword fields by using heterogeneous map types
  \item We identify the main future issues in typing Clojure code.

\end{itemize}

\end{posterbox}

%%%%%%%%%%%%%%%%%%%%%%%%%%%%%%%%%%%%%%%%%%%%%%%%%%%%%%%%%%%%%%%%%%%%%%%%%%%%%%
\begin{posterbox}[name=sequenceabs,column=1]{Sequence Abstraction}
%%%%%%%%%%%%%%%%%%%%%%%%%%%%%%%%%%%%%%%%%%%%%%%%%%%%%%%%%%%%%%%%%%%%%%%%%%%%%%

Some common idioms that Typed Clojure types can express
are usages of Clojure's sequence abstraction, a set of functions providing a common
interface to collections.
These functions include:

\begin{itemize}
\item \lstinline|seq|, which takes a collection (or \lstinline|nil|) and returns
   \begin{itemize}
      \item a true value if the argument is non-empty
      \item a false value if the argument is empty (or \lstinline|nil|)
   \end{itemize}
\item \lstinline|first|, which takes a collection (or \lstinline|nil|) and returns
   \begin{itemize}
      \item the first item of the argument if the argument is non-empty
      \item \lstinline|nil| if the argument is empty (or \lstinline|nil|)
    \end{itemize}
\end{itemize}

By encoding these invariants in the respective types, 
we can type check common combinations of these functions.

\begin{lstlisting}[caption={Assuming \lstinline|coll| is a collection of numbers, we can infer that this expression always returns a number}]
(if (seq coll)
  (first coll)
  0)
\end{lstlisting}

\end{posterbox}

%%%%%%%%%%%%%%%%%%%%%%%%%%%%%%%%%%%%%%%%%%%%%%%%%%%%%%%%%%%%%%%%%%%%%%%%%%%%%%
\begin{posterbox}[name=hmaps,column=1,below=sequenceabs]{Heterogeneous Maps}
%%%%%%%%%%%%%%%%%%%%%%%%%%%%%%%%%%%%%%%%%%%%%%%%%%%%%%%%%%%%%%%%%%%%%%%%%%%%%%

Using plain maps with known keyword keys is common style in Clojure.
Heterogeneous map types that record the \emph{presence} of keyword keys
are sufficient to type check many usages of this idiom.
Accurate types for lookup, associating, and dissociating of known keyword keys are supported.

\end{posterbox}


%%%%%%%%%%%%%%%%%%%%%%%%%%%%%%%%%%%%%%%%%%%%%%%%%%%%%%%%%%%%%%%%%%%%%%%%%%%%%%
%%%%%%%%%%%%%%%%%%%%%%%%%%%%%%%%%%%%%%%%%%%%%%%%%%%%%%%%%%%%%%%%%%%%%%%%%%%%%%
\begin{posterbox}[name=nulltreatment,column=1,below=hmaps]{Treatment of \emph{null}}

Clojure represents Java's \emph{null} as the value \lstinline|nil|,
which has the singleton type \lstinline|nil| in Typed Clojure.
This means, unlike Java's static type system, \emph{null} and reference types are separated as static types.
The programmer can then annotate Java methods and fields, specifying where \emph{null} is allowed.

\end{posterbox}


\begin{posterbox}[name=experiments,column=2]{Experiments}

\section*{Monads}

Almost all of a Clojure library for monadic programming (\emph{algo.monads}) was successfully ported
to Typed Clojure. This included most:
\begin{itemize}
  \item Monad definitions
  \item Monad function definitions
  \item Monad transformer defintions
\end{itemize}

This experiment warranted an extension for type functions (functions at the type level), with variance.

\section*{Java Interoperability}

A function relying heavily on Java interoperability was ported successfully.
This function chained several Java calls together, and
Typed Clojure was able to express and check invariants, mostly related to occurrences of \emph{null}
and primitive arrays.

\end{posterbox}

\begin{posterbox}[name=futurework,column=2,below=experiments]{Future Work}

There is significant future work to type check all Clojure idioms:

\begin{itemize}
\item Multimethod definitions should infer types from their dispatch function
\item Implement a blame calculus to improve error messages when using untyped code
\end{itemize}

\end{posterbox}

\begin{posterbox}[name=conclusion,column=2,below=futurework]{Conclusion}
This work has demonstrated that it is both practical and useful to
design and implement an optional type system for Clojure that preserves
current Clojure style.
\end{posterbox}

\begin{posterbox}[name=source,column=2,above=bottom]{Source}
Typed Clojure is available at: \\
\url{https://github.com/frenchy64/typed-clojure}
\end{posterbox}

\end{poster}

\end{document}

