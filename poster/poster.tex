\documentclass[landscape,final,a0paper,fontscale=0.277]{baposter}

\usepackage{url}
\usepackage{relsize}

\usepackage{color} % for colorbox
%\usepackage[svgnames]{xcolor}

\usepackage{graphicx}
\graphicspath{{images/}}

\definecolor{codebg}{HTML}{EEEEEE}

\usepackage{listings}
\lstset{ %
  language=Lisp,                % choose the language of the code
  columns=fixed,basewidth=.5em,
  basicstyle=\small\ttfamily,       % the size of the fonts that are used for the code
  %numbers=left,                   % where to put the line-numbers
  %numberstyle=\small\ttfamily,      % the size of the fonts that are used for the line-numbers
  %stepnumber=1,                   % the step between two line-numbers. If it is 1 each line will be numbered
  %numbersep=5pt,                  % how far the line-numbers are from the code
  backgroundcolor=\color{codebg},  % choose the background color. You must add \usepackage{color}
  %showspaces=false,               % show spaces adding particular underscores
  showstringspaces=false,         % underline spaces within strings
  %showtabs=false,                 % show tabs within strings adding particular underscores
  frame=single,           % adds a frame around the code
  %tabsize=2,          % sets default tabsize to 2 spaces
  captionpos=t,           % sets the caption-position to bottom
  breaklines=true,        % sets automatic line breaking
  breakatwhitespace=true,    % sets if automatic breaks should only happen at whitespace
  %escapeinside={\%*}{*)},          % if you want to add a comment within your code
}

\begin{document}

\definecolor{lightorange}{rgb}{0.9,0.4,0}
\definecolor{lightestorange}{rgb}{1,0.8,0.5}
\definecolor{darkorange}{rgb}{0.2,0.1,0}
\definecolor{ForestGreen}{rgb}{0.13, 0.54, 0.13}
\definecolor{navyblue}{rgb}{0,0,0.5}
%%
\begin{poster}%
  % Poster Options
  {
  % Show grid to help with alignment
  grid=false,
  % Column spacing
  colspacing=1em,
  % Color style
  bgColorOne=lightestorange,
  bgColorTwo=white,
  borderColor=darkorange,
  headerColorOne=ForestGreen,
  headerColorTwo=navyblue,
  headerFontColor=white,
  boxColorOne=green,
  boxColorTwo=blue,
  columns=3,
  % Format of textbox
  textborder=faded,
  % Format of text header
  eyecatcher=true,
  headerborder=closed,
  headerheight=0.1\textheight,
%  textfont=\sc, An example of changing the text font
  headershape=roundedright,
  headershade=shadelr,
  headerfont=\Large\bf\textsc, %Sans Serif
  textfont={\setlength{\parindent}{1.5em}},
  boxshade=plain,
%  background=shade-tb,
  background=none,
  linewidth=2pt
  }
  % Eye Catcher
  {\includegraphics[height=7em]{images/clojure-logo10.png}} 
  % Title
  {\bf\textsc{A Practical Optional Type System for Clojure}}
  % Authors
  %{\textsc{\{ Brian.Amberg and Thomas.Vetter \}@unibas.ch}}
  { \textsc{Ambrose Bonnaire-Sergeant <abonnairesergeant@gmail.com> (20350292)}
    \\
    \textsc{{Supervised by Rowan Davies <rowan@csse.uwa.edu.au>}}}
  % University logo
  {% The makebox allows the title to flow into the logo, this is a hack because of the L shaped logo.
    \includegraphics[width=14em]{images/UWA_logo.png}
  }
%
%  % Eye Catcher
%  {\includegraphics{images/1000px-UWA_crest}}
%   
%  % Title
%  { \textsc}
%  % Authors
%  % University logo
%  { University of Western Australia }
%%%%%%%%%%%%%%%%%%%%%%%%%%%%%%%%%%%%%%%%%%%%%%%%%%%%%%%%%%%%%%%%%%%%%%%%%%%%%%
%%% Now define the boxes that make up the poster
%%%---------------------------------------------------------------------------
%%% Each box has a name and can be placed absolutely or relatively.
%%% The only inconvenience is that you can only specify a relative position 
%%% towards an already declared box. So if you have a box attached to the 
%%% bottom, one to the top and a third one which should be in between, you 
%%% have to specify the top and bottom boxes before you specify the middle 
%%% box.
%%%%%%%%%%%%%%%%%%%%%%%%%%%%%%%%%%%%%%%%%%%%%%%%%%%%%%%%%%%%%%%%%%%%%%%%%%%%%%

%%%%%%%%%%%%%%%%%%%%%%%%%%%%%%%%%%%%%%%%%%%%%%%%%%%%%%%%%%%%%%%%%%%%%%%%%%%%%%
  \begin{posterbox}[name=problem,column=0,row=0]{Background}
%%%%%%%%%%%%%%%%%%%%%%%%%%%%%%%%%%%%%%%%%%%%%%%%%%%%%%%%%%%%%%%%%%%%%%%%%%%%%%

\section*{\textsc{Static and Dynamic Types}}

Traditionally there has been a clear distinction between 
programming languages that perform type checking at compile-time, and those that do not.
A recent trend is to aim to combine the advantages of both kinds of languages by adding optional static 
type systems to languages without static type checking.

\section*{\textsc{Clojure}}

Clojure is a recent and increasingly popular dynamically typed programming language.
Designed to be pragmatic, it is a dialect of Lisp supporting functional programming and immutability by default,
while being hosted on, and providing direct interoperability with, popular platforms.
\end{posterbox}

%%%%%%%%%%%%%%%%%%%%%%%%%%%%%%%%%%%%%%%%%%%%%%%%%%%%%%%%%%%%%%%%%%%%%%%%%%%%%%
\begin{posterbox}[name=contribution,column=0,below=problem]{Contributions}
%%%%%%%%%%%%%%%%%%%%%%%%%%%%%%%%%%%%%%%%%%%%%%%%%%%%%%%%%%%%%%%%%%%%%%%%%%%%%%

We take the lessons learnt from several projects to design \emph{Typed Clojure},
an optional type system for Clojure running on the Java Virtual Machine. 
The main contributions are:
\begin{itemize}
  \item We develop a prototype type checker based on Typed Racket [1], which is able to type check many Clojure idioms
  \item We describe a novel use of \emph{occurrence typing} for eliminating erroneous access to Java's \emph{null} pointer
  \item We show how a combination of occurrence typing and intersection types allows accurate typing checking of
        the most common usages of Clojure's \emph{sequence abstraction} 
  \item We capture the Clojure idiom of using maps with known keyword fields by using \emph{heterogeneous map types}
  \item We identify the main \emph{future issues} in typing Clojure code.

\end{itemize}

\end{posterbox}

\begin{posterbox}[name=nulltreatment,column=0,above=bottom]{Treatment of \emph{null}}

Clojure represents Java's \emph{null} as the value \colorbox{codebg}{\lstinline{nil}},
which has the singleton type \colorbox{yellow}{\lstinline|nil|} in Typed Clojure.
This means, unlike Java's static type system, \emph{null} and reference types are separated as static types.
The programmer can then annotate Java methods and fields, specifying where \emph{null} is allowed.

\begin{lstlisting}[caption={Annotating Java methods that never return \emph{null}}]
(non-nil-return java.lang.Object/getClass :all)
(non-nil-return java.io.File/getName :all)
\end{lstlisting}
\end{posterbox}

%%%%%%%%%%%%%%%%%%%%%%%%%%%%%%%%%%%%%%%%%%%%%%%%%%%%%%%%%%%%%%%%%%%%%%%%%%%%%%
\begin{posterbox}[name=sequenceabs,column=1]{Sequence Abstraction}
%%%%%%%%%%%%%%%%%%%%%%%%%%%%%%%%%%%%%%%%%%%%%%%%%%%%%%%%%%%%%%%%%%%%%%%%%%%%%%

Typed Clojure types can express the most common usages of Clojure's sequence abstraction, 
which is a set of functions providing a common interface to collections.
These functions include:

\begin{itemize}
\item \colorbox{codebg}{\lstinline|seq|}, which takes a collection (or \colorbox{codebg}{\lstinline|nil|}) and returns
   \begin{itemize}
      \item a true value if the argument is non-empty
      \item a false value if the argument is empty (or \colorbox{codebg}{\lstinline|nil|})
   \end{itemize}
\item \colorbox{codebg}{\lstinline|first|}, which takes a collection (or \colorbox{codebg}{\lstinline|nil|}) and returns
   \begin{itemize}
      \item the first item of the argument if the argument is non-empty
      \item \colorbox{codebg}{\lstinline|nil|} if the argument is empty (or \colorbox{codebg}{\lstinline|nil|})
    \end{itemize}
\end{itemize}

By encoding these invariants in the respective types, 
we can type check common combinations of these functions.

\begin{lstlisting}[caption={Assuming \lstinline|coll| is a collection of numbers currently in local scope, we can infer that this expression always returns a number}]
(if (seq coll)
  (first coll)
  0)
\end{lstlisting}
\end{posterbox}

%%%%%%%%%%%%%%%%%%%%%%%%%%%%%%%%%%%%%%%%%%%%%%%%%%%%%%%%%%%%%%%%%%%%%%%%%%%%%%
\begin{posterbox}[name=hmaps,column=1,below=sequenceabs]{Heterogeneous Maps}
%%%%%%%%%%%%%%%%%%%%%%%%%%%%%%%%%%%%%%%%%%%%%%%%%%%%%%%%%%%%%%%%%%%%%%%%%%%%%%

Heterogeneous map types that record the \emph{presence} of keyword keys
are sufficient to type check the most common usages of the Clojure idiom of using plain maps with known keyword keys.

\begin{description}
  \item[Map Literals] \colorbox{codebg}{\lstinline|\{:a 1 :b 2\}|} creates a map value with two entries, and
        is of the heterogeneous map type \colorbox{yellow}{\lstinline{'\{:a Number :b Number\}}}
  \item[Associating keyword keys] \colorbox{codebg}{\lstinline|(assoc \{:a 1 :b 2\} :c 3)|} 
      extends the first argument with a new entry with key \lstinline|:c|, and
        is of type \colorbox{yellow}{\lstinline{'\{:a Number :b Number :c Number\}}}
  \item[Dissociating keyword keys] \colorbox{codebg}{\lstinline|(dissoc \{:a 1 :b 2\} :a)|} dissociates 
      from the first argument the entry with key \lstinline|:a|, and
        is of type \colorbox{yellow}{\lstinline|'\{:b Number\}|}
  \item[Keyword lookup] \colorbox{codebg}{\lstinline|(get \{:a 1 :b 2\} :a)|} 
      returns the value corresponding to the key \lstinline|:a| in the first argument, and
        is of type \colorbox{yellow}{\lstinline|Number|}
\end{description}

\end{posterbox}

%%%%%%%%%%%%%%%%%%%%%%%%%%%%%%%%%%%%%%%%%%%%%%%%%%%%%%%%%%%%%%%%%%%%%%%%%%%%%%
%%%%%%%%%%%%%%%%%%%%%%%%%%%%%%%%%%%%%%%%%%%%%%%%%%%%%%%%%%%%%%%%%%%%%%%%%%%%%%

\begin{posterbox}[name=source,column=2]{Source}
Typed Clojure is available at:\\\url{https://github.com/frenchy64/typed-clojure}
\end{posterbox}


\begin{posterbox}[name=experiments,column=2,below=source]{Experiments}

\section*{\textsc{Monads}}

Almost all of a Clojure library for monadic programming (\emph{algo.monads}) was successfully ported
to Typed Clojure, including most monad, monad function, and monad transformer definitions.
This experiment warranted an extension for type functions (functions at the type level) with variance.

\section*{\textsc{Java Interoperability}}

A function relying heavily on Java interoperability was ported successfully.
This function chained several Java calls together, and
Typed Clojure was able to express and check invariants related to occurrences of \emph{null}
and primitive arrays.

\end{posterbox}

\begin{posterbox}[name=futurework,column=2,below=experiments]{Future Work}

There is significant future work.

\begin{itemize}
\item Implement a blame calculus to improve error messages when using untyped code
\item Support type checking multimethod definitions
\item Support Clojure records, which are datatypes with known keyword keys
\item Port Typed Clojure to other Clojure implementations
\end{itemize}

\end{posterbox}

\begin{posterbox}[name=conclusion,column=2,below=futurework]{Conclusion}
This work has demonstrated that it is both practical and useful to
design and implement an optional type system for Clojure that preserves
current Clojure style.
\end{posterbox}

\begin{posterbox}[name=references,column=2,below=conclusion]{References}
  \smaller
  \bibliographystyle{ieee}
  \renewcommand{\section}[2]{\vskip 0.05em}
  \begin{thebibliography}{1}\itemsep=-0.01em
    \setlength{\baselineskip}{0.4em}
    \bibitem{tob10:bnb}
    Tobin-Hochstadt, S.
    \newblock {T}yped {S}cheme: {F}rom {S}cripts to {P}rograms
    \newblock PhD Dissertation,
    \newblock Northeastern University, 2010
  \end{thebibliography}
  \vspace{0.3em}

  \textsc{\\\smaller Bachelor of Computer Science Honours (Semester 1, 2012)}
\end{posterbox}

\end{poster}

\end{document}

