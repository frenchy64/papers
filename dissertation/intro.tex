\chapter{Introduction}

\section{Thesis}

\emph{It is practical and useful to design and implement an optional typing system 
for the Clojure programming language using bidirectional checking that allows Clojure programmers to continue 
using idioms and style found in current Clojure code.}

\section{Motivation}

In the last decade it has become increasingly common to enhance
dynamically typed languages with static type systems. 
This is idea not new, but recent attempts are noteworthy for 
their broad success in matching many of the advantages of statically typed
languages, notably due to the use of \emph{bidirectional checking}.
Instead of always attempting to infer types, this algorithm relies on programmer annotations
appearing in some natural places such as giving the type of each top-level function.

The Clojure programming language is a dynamically typed dialect of Lisp invented
by Hickey, designed to run on popular platforms\footnote{http://clojure.org/}.
It emphasises functional programming with immutable data structures
and provides direct interoperability with its host platform.
Notable implementations of Clojure exist for the Java Virtual Machine (JVM),
the Common Language Runtime, and for Javascript virtual machines.
At the current time, Clojure on the JVM is the most mature implementation,
and therefore this project focuses on the JVM implementation.

Clojure has attracted wide-spread users in part by concentrating on pragmatism.
Performance is a key feature, for example the JVM implementation of Clojure
offers ways to access Java-like speed for certain operations.
Also, Clojure's extensive host interoperability offers Clojure programmers
access to existing libraries for their platform, such as the extensive Java library ecosystem.
By coupling pragmatic necessities with elegant features like Lisp-style macros, functional programming,
and immutability by default, Clojure is a compelling general purpose programming language.

Recently a number of languages have been created to or been modified to support aspects
of both static and dynamic typing.
Dart\footnote{http://www.dartlang.org/} is dynamically typed but offers a simple form of optional
static typing that specifically do not affect runtime semantics.
Typescript\footnote{http://www.typescriptlang.org/} adds an optional type system to Javascript,
a well-known dynamic language.
Typed Racket~\cite{TF08,Tob10} goes further by offering safe interoperability
between typed and untyped modules by generating appropriate runtime assertions based
on expected static types.

When a static type checker is not available, which describes the situation
for most dynamic languages, other techniques are used for checking
type invariants. For example, ``design by contract'' is often used,
introduced by Meyer for the Eiffel language~\cite{Mey92},
in which the programmer defines contracts that are enforced at runtime.
Unit testing is also a popular verification technique in dynamic languages.
Clojure adopts these approaches, providing easy syntax for defining 
Eiffel-inspired pre- and post-conditions and a library for writing unit tests.

Static type systems, however, are still desirable for many situations.
Powerful type systems like ML's~\cite{Mil97} and Haskell's~\cite{Mar10} have proved particularly useful
when complicated programming styles are required. For example,
Haskell's advanced static type system helps the programmer write correct monadic code
(as detailed by Wadler~\cite{Wad95})
especially in more complicated situations like combining monads via monad transformers.

\subsection{Why implement an optional type system for Clojure?}

The initial motivation for implementing an optional type system
for Clojure was an anecdotal account of the struggles
of a Clojure programmer. In an apparently heroic effort, he managed to 
implement a Clojure library for conduits, an advanced form of ``pipes'',
using arrows, a generalisation of monads.
\footnote{CONDUIT REFERENCE TODO}
Conduits also significantly surpass monads in complexity and are usually reserved
for languages with advanced type systems like Haskell.

He highlighted a strong desire for a type system for several reasons.
Firstly, to verify the correctness of the library.
Without a static type system, it is a significant task
to verify such a implementation correct due to heavy use
of higher-order functions.
Secondly, to aid him while writing the library.

Since then, the Clojure community has shown significant interest in this work.
I also developed this project as a Google Summer of Code 2012 project,
after it was selected by the Clojure community as a Clojure Google Summer of Code
project for 2012.
\footnote{CITE GSOC, or FOOTNOTE}
I have also been invited to speak on this project at the Clojure Conj
2012 conference in November, the main international conference related to Clojure.

\subsection{What kind of type system does Typed Clojure provide?}

% refinements check invariants of programs
% while ordinary static types check whether programs
% are basically meaningful - Rowan + Frank P
% Page 1 Rowan's PhD

There are many concepts associated with \emph{types} and \emph{type systems} in both the
literature and informal discourse.
A programmer who uses dynamically-typed languages may have a drastically different notion
of what a type is than, say, a programmer preferring languages with advanced static type systems.
There is some debate as to whether optional static type systems like Typed Clojure
can even be called a type system. We choose to
follow the terminology of Pfenning~\cite{Pfe08} and Reynolds~\cite{Rey01},
where such optional type systems are \emph{extrinsic} type systems, and more
traditional type systems are \emph{intrinsic} type systems.
This distinction has a long history, originating in work in the 

An ordinary static type system is used to check whether programs are basically
meaningful. Pfenning and Reynolds call these type systems \emph{intrinsic}. A language with an intrinsic
static type system has run-time semantics that depends on the types of associated
variables and expressions during type checking.
For example, C, Java, and ML have intrinsic types.
This means programs written in these languages must pass the type checker before being run.

A static type system is \emph{extrinsic} when runtime semantics
does not depend on the type system. In other words, passing a static type checker
is not essential to running programs. A dynamically typed language can be viewed
as having a trivial static type system that supports exact one type,
a view strongly advocated by Harper~\cite{Har12}
and common in the literature on static types (for example, Pierce~\cite{Pie02}).

\section{Typed Clojure through Examples}

% Hello world
This section introduces Typed Clojure with example code. 
Typed Clojure is developed for the JVM implementation of Clojure, therefore
the rest of this chapter uses that implementation.
An attempt is
made to introduce some Clojure syntax and semantics to those unfamiliar or needing a refresher.
A basic knowledge of Lisp syntax is handy, but a brief tutorial is given
for newcomers.

\subsection{Preliminary: Lisp Syntax}

The core of understanding Lisp syntax when coming from a popular language
like Java or Javascript can be summarised by these points.

\begin{itemize}
  \item Operators are always in prefix position.
  \item Invocations are always wrapped in a pair of balanced parenthesis.
  \item Parenthesis start to the left of the operator.
\end{itemize}

For example, the Java expressions \emph{1 + 2 / 3} is written in Lisp pseudocode \lstinline|(/ (+ 1 2) 3)|
and \emph{numberCrunch(1, 2)} written \lstinline|(numberCrunch 1 2)|.

Clojure also adds other syntax:

\begin{itemize}
  \item Prefixing \lstinline|:| to a symbol defines a \emph{keyword}, often used for map keys. eg. \lstinline|:my-keyword|.
  \item Square brackets delimit vector literals. eg. \lstinline|[1 2]| is a 2 place vector.
  \item Curly brackets define map literals. eg. \lstinline|{:a 1 :b 2}| is a map from 
        \lstinline|:a| to \lstinline|1| and \lstinline|:b| to \lstinline|2|.
  \item Commas are always optional, and treated as whitespace.
\end{itemize}

\subsection{Simple Examples}

We begin with the obligatory \emph{Hello world} example.

\begin{lstlisting}[caption=Typed Hello world, label=lst:helloworld]
(ns typed.test.hello-world
  (:require [typed.core :refer [check-ns]]]))

(println "Hello world")
\end{lstlisting}

At this point, it is worth understanding Clojure's namespacing feature.
Clojure code is always executed in a \emph{namespace}, and each file of Clojure code should 
have a \lstinline|ns| declaration with the namespace name and its dependencies,
which switches the current namespace and executes the given dependency commands.
There is one special namespace, \lstinline|clojure.core| which is
loaded with every namespace, implicitly ``referring'' all its vars in the namespace.
For example, \lstinline|ns| refers to the var \lstinline|clojure.core/ns|,
similarly \lstinline|println| refers to \lstinline|clojure.core/println|.

The example in listing \ref{lst:helloworld} declares a dependency to 
\lstinline|typed.core|, Typed Clojure's main namespace. It also refers the var \lstinline|typed.core/check-ns|
into scope\footnote{\lstinline|check-ns| is the top level function for type checking a namespace}.
Other than this dependency, this is identical to the untyped \emph{Hello world}.

More complex code require extra annotations to type check:

\begin{lstlisting}[caption=Annotating vars in Typed Clojure]
(ns typed.test.collatz
  (:require [typed.core :refer [check-ns ann]]))

(ann collatz [Number -> Number])
(defn collatz [n]
  (cond
    (= 1 n) 
      1
    (and (integer? n) 
         (even? n)) 
      (collatz (/ n 2))
    :else 
      (collatz (inc (* 3 n)))))
\end{lstlisting}
\footnote{Example adapted from Tobin-Hochstadt~\cite{Tob10}}

In this example, we define a new var \lstinline|typed.test.collatz/collatz|. Typed Clojure requires all 
used vars to be annotated. Here \lstinline|typed.core/ann| annotates \lstinline|typed.test.collatz/collatz|
to be a function from \lstinline{java.lang.Number} to 
\lstinline{java.lang.Number}\footnote{All Classes in the \lstinline|java.lang| package
are automatically imported in every Clojure namespace (the equivalent of Java's \lstinline|import java.lang.*;|).}.

\subsection{Datatypes and Protocols}

We can annotate datatype and protocol definitions similarly.

\begin{lstlisting}[caption=Annotating protocols and datatypes in Typed Clojure]
(ns typed.test.deftype
  (:require [typed.core 
             :refer [check-ns ann-datatype
                     tc-ignore ann-protocol AnyInteger]]))

(ann-protocol Age 
  :methods
  {age [Age -> AnyInteger]})
(tc-ignore
  (defprotocol Age
    (age [this])))

(ann-datatype Person 
  [[name :- String]
   [age :- AnyInteger]])
(deftype Person [name age]
  Age
  (age [this] age))

(age (Person. "Lucy" 34))
\end{lstlisting}

\lstinline|clojure.core/defprotocol| defines a new Clojure protocol\footnote{See http://clojure.org/protocols for a full description of protocols}
with a set of methods. \lstinline|typed.core/ann-protocol| annotates a protocol with the types of its methods.
In this example, we define a protocol \lstinline|typed.test.person/Age| with an \lstinline|age| method.
The call to \lstinline|clojure.core/defprotocol| also defines a new var \lstinline|typed.test.person/age|, a first-class function
wrapping the \lstinline|age| method, but taking the target Object as the first parameter. The
type signature provided with \lstinline|typed.core/ann-protocol|, here \lstinline|[typed.test.person/Age -> typed.core/AnyInteger]|, 
is for this function.

Invocations of \lstinline|clojure.core/defprotocol| are currently not able to be type checked
and are ignored by Typed Clojure by passing them to \lstinline|typed.core/tc-ignore|.

\lstinline|clojure.core/deftype|
defines a new Clojure datatype\footnote{See http://clojure.org/datatypes for a full description of datatypes}
in the current namespace with a number of fields and methods. 
\lstinline|typed.core/ann-datatype| annotates a datatype with its field types.
In this example, we create a datatype \lstinline|typed.test.person.Person|
with fields \lstinline|name| and \lstinline|age| and implement the \lstinline|age|
method from protocol \lstinline|typed.test.person/Age|.

Java constructors are invoked in Clojure by suffixing the Class we want to instantiate with a dot.
Datatypes are implemented as Java Classes with immutable fields (by default) and a single constructor, taking as arguments its fields 
in the order they are passed to \lstinline|deftype|
\footnote{When unambiguous, I omit the qualifying namespace/package for the remainder of the chapter.}.
\lstinline|(Person. "Lucy" 34)| constructs a new \lstinline|Person|
instance, setting the fields to their corresponding positional arguments.
Typed Clojure checks the datatype constructor to be the equivalent of 
\lstinline|[String AnyInteger -> Person]|.

Finally, Typed Clojure checks invocations of Protocol methods. It infers \lstinline|Person|
is an instance of \lstinline|Age| from the datatype definition, therefore \lstinline|(age (Person. "Lucy" 34))| is type-safe.

\subsection{Polymorphism}

Typed Clojure supports F-bounded polymorphism, first introduced by Canning, Cook, Hill and Olthoff~\cite{CCHOM89}. 
F-bounded polymorphism is an extension of bounded polymorphism, where polymorphic type variables
can be restricted by \emph{bounds}.
In particular, F-bounded polymorphism allows type variable bounds to recursively refer to the
variable being bounded. Typed Clojure supports upper and lower type variable bounds.

Typed Clojure parameterises some of Clojure's data structures. For example,
the interface behind Clojure's \emph{seq} abstraction \lstinline|clojure.lang.Seqable| has one 
covariant parameter\footnote{\lstinline|(Seqable Integer)| being a subtype of \lstinline|(Seqable Number)|
because Integer is a subtype of Number.}.

\begin{lstlisting}[caption=Polymorphism in Typed Clojure]
...
(ann to-set 
     (All [x]
       [(U nil (Seqable x)) -> (clojure.lang.PersistentHashSet x)]))
(defn to-set [a]
  (set a))
...
\end{lstlisting}

In this example\footnote{When convenient, namespace declarations are omitted for the remainder of the chapter.}, 
we define \lstinline|to-set|, aliasing \lstinline|clojure.core/set|.
\lstinline|All| introduces a set of type variables to the body of a type,
here \lstinline|x| is used to define a relationship between the input type and return type.

\lstinline|(U nil (Seqable x))| is a common type in Typed Clojure, read as the union
of the type \lstinline|nil| and the type \lstinline|(Seqable x)|.
The vast majority of types for collection processing functions in the Clojure core library feature
it as an input type, where passing \lstinline|nil| either has some special behaviour 
or is synonymous with passing an empty \lstinline|Seqable|.

\subsection{Singleton Types}

Following Typed Racket, singleton types for certain values are provided
in Typed Clojure.
A singleton type is a type with a single member, like \lstinline|1|,
\lstinline|:a|, or \lstinline|"a"|\footnote{Strings are delimited by " in Clojure}.
Typed Clojure provides syntax for singleton types, either by passing
the value literal to the \lstinline|Value| primitive, or by prefixing
a quote (').

\begin{lstlisting}[caption=Singleton Types, label=lst:singletoneeg]
(ann k ':my-keyword)
(def k :my-keyword)
\end{lstlisting}

Listing \ref{lst:singletoneeg} shows a simple example of using
singleton types in Typed Clojure.
Singleton types are are discussed further in section \ref{sec:intersectionunion}.

\subsection{Heterogeneous Maps}

A novel feature of Clojure compared to other dialects of Lisp (including Racket)
is an emphasis on key-value paired, unordered \emph{maps}.
Where particularly object-oriented languages reach for objects, Clojure
utilises maps. Clojure provides a map literal using curly braces. For example,
\lstinline|{:a 1, :b 2}| is a map value with two key-value entries: from keyword key \lstinline|:a|
to value \lstinline|1|, and keyword key \lstinline|:b| to value \lstinline|2|. Note that commas are always
whitespace in Clojure and are included occasionally for readability.

Typed Clojure provides a heterogeneous map type which captures this common
``maps as objects'' pattern. A heterogeneous map type has only \emph{positive}
information on the types of key-value entries. In other words, it conveys
whether a particular key is present, but not whether it is absent.
The implications of this is discussed in section \ref{ref:designhmap}.
Heterogeneous maps only support keys that are singleton Keyword types. This restriction is reflected
in the syntax for defining heterogeneous map types.

\begin{lstlisting}[caption=Heterogeneous map types in Typed Clojure]
...
(ann config '{:file String, :ns Symbol}))
(def config
  {:file "clojure/core.clj",
   :ns 'clojure.core})
...
\end{lstlisting}

This example checks \lstinline|config| to be a heterogeneous map
with \lstinline|:file| and \lstinline|:ns| keys, with values of
type \lstinline|String| and \lstinline|Symbol| respectively.

Heterogeneous vector and seq types are also provided.

\subsection{Variable-Arity Functions}

Functions in Clojure are multi-arity, which means a function
can be defined with several function bodies (or \emph{arities})
and which arity is executed depends on the number of arguments passed
to the function. A function can have any number of arities with fixed parameters, and at most one arity with
variable-parameters. Each arity must have a unique number of
parameters and have a lower number of parameters than the ``variable arity'', if present.

Strickland, Tobin-Hochstadt, and Felleisen invented a calculus and corresponding
implementation in Typed Racket for variable-arity polymorphism~\cite{STF09}
that is sufficient to handle uniform and non-uniform variable-arity functions.
Typed Clojure includes an implementation of the most immediately useful parts of variable-arity
polymorphism using algorithms, nomenclature, and implementation based on this work.

\subsubsection{Uniform Variable-Arity}

A function with uniform variable parameters can treat its variable parameter
as a homogeneous list. 
Strickland et al. attaches a \emph{starred pre-type} \lstinline|T *| to the right of the fixed arguments
in a function type, where \emph{T} is some type, and we take an identical approach.
For example, \lstinline|+| in Clojure accepts any number of arguments
of type \lstinline|Number|, represented by the type \lstinline|[Number * -> Number]|.


\begin{lstlisting}[caption=Typing multi-arity functions, label=lst:noteq]
(ann not= (Fn [Any -> boolean]
              [Any Any -> boolean]
              [Any Any Any * -> boolean]))
(defn not=
  "Same as (not (= obj1 obj2))"
  ([x] false)
  ([x y] (not (= x y)))
  ([x y & more]
   (not (apply = x y more))))
\end{lstlisting}

It is common to find Clojure library functions that define seemingly redundant
function arities for performance reasons.
Listing \ref{lst:noteq} defines the multi-arity function \lstinline|not=|,
taken from the Clojure standard library \lstinline|clojure.core| that uses this pattern.
\lstinline|not=| has three arities in its definition, including
one that takes a variable number of arguments.
The \lstinline|Fn| type constructor builds an \emph{ordered function intersection type} from
function types. Each arity must have at least one matching function type associated with it.
The two ``fixed arities'' are given familiar function types, with one and two fixed parameters respectively.
The type given for the arity with variable arguments \lstinline|[Any Any Any * -> boolean]|
uses a starred pre-type to signify any number of arguments of type \lstinline|Any|
can be provided to the right of its fixed arguments.

\subsubsection{Non-uniform Variable-Arity Functions}

Where \emph{uniform} variable-arity function types use \emph{starred pre-types}, \emph{non-uniform}
variable-arity function types use \emph{dotted pre-types}.
Typed Clojure supports usages of \emph{non-uniform} variable-arity functions,
where the variable parameter is a heterogeneous list, represented by a \emph{dotted type variable}.

For example, the variable argument function \lstinline|clojure.core/map| takes a function and one or more sequences,
and returns the result of applying the function argument to each element of the sequences in a pair-wise fashion.
Its type is given in listing \ref{lst:maptype}.

\begin{lstlisting}[caption=Type signature for \lstinline|clojure.core/map|, label=lst:maptype]
(ann clojure.core/map
     (All [c a b ...]
       [[a b ... b -> c] (U nil (Seqable a)) (U nil (Seqable b)) ... b -> (LazySeq c)]))
\end{lstlisting}

By adding \lstinline|...| after the last type variable in an \lstinline|All| binder
we can introduce a \emph{dotted type variable} into scope, which is a placeholder for a sequence of types.
A \emph{dotted pre-type} \lstinline|T ... b| over \emph{base} \lstinline|T| (a type) and \emph{bound}
\lstinline|b| (a dotted type variable)
serves as a placeholder for this sequence of types.
Dotted pre-types must appear to the right of all fixed parameters in a function type,
and cannot be mixed with other kinds of variable parameters like starred pre-types.
When a sequence of types of length \emph{n} is associated with a dotted pre-type, 
the dotted pre-type is expanded to \emph{n} copies of \lstinline|T|.
One other special property of a dotted pre-type is that the bound \lstinline|b|
is \emph{in scope} as a \emph{normal} type variable in its base \lstinline|T|.

To demonstrate dotted pre-types, we use \lstinline|typed.core/inst| to instantiate
\lstinline|map|. \lstinline|inst| takes a polymorphic expression and a number of types
that are satisfactory for instantiating the polymorphic type and returns an expression
of the instantiated polymorphic type.

The instantiation

\begin{lstlisting}
(inst map Number boolean String)
\end{lstlisting}

returns an expression of type

\begin{lstlisting}
[[boolean String Integer -> Number] (U (Seqable boolean) nil) (U (Seqable String) nil) (U (Seqable Integer) nil) -> (LazySeq Number)]
\end{lstlisting}

However, if sufficient types are given, the instantiation for \lstinline|map| can be inferred.
The invocation

\begin{lstlisting}
(map (ann-form (fn [a b c] c) 
               [boolean String Number -> Number]) 
     [true false] ["mystr" "astr"] [1 4])
\end{lstlisting}

uses \lstinline|typed.core/ann-form| to assign an expected to
the first argument,
which is sufficient to infer the result type

\begin{lstlisting}
(LazySeq Number)
\end{lstlisting}


\subsection{Occurrence Typing}

It is common in Clojure, like other dynamically-typed languages, to
encode implicit type invariants in conditional tests.
In listing \ref{lst:numvec2} conditional tests are used
to refine the type of the bindings \lstinline|a| and \lstinline|b|.
Occurrence typing is a technique useful for capturing these kinds
of type invariants (occurrence typing is discussed in further detail in section \ref{sec:OccurrenceTyping}).

\begin{lstlisting}[caption=Example of occurrence typing in Typed Clojure, label=lst:numvec2]
(ann num-vec2 
     [(U nil Number) (U nil Number) -> (Vector* Number Number)])
(defn num-vec2 [a b]
  [(if a a 0) 
   (if b b 0)])
\end{lstlisting}

To check this example, occurrence typing infers type information based on the result of each test.
In Clojure, \lstinline|nil| and \lstinline|false| are false values and all other values are true.
The test \lstinline|(if a a 0)| follows the \emph{then} branch is \lstinline|a| is not \lstinline|nil|
or \lstinline|false|. When checking this branch, we can safely refine the type of \lstinline|a| to \lstinline|Number| from
\lstinline|(U nil Number)|. Similarly, following the \emph{else} branch refines the type of \lstinline|a|
to \lstinline|nil| from \lstinline|(U nil Number)|.

\subsection{Java Interoperability}

Typed Clojure currently provides safe interoperability
\footnote{Compatibility with Generic Java is planned future work}.

The key to safe Java interoperability is the treatment of Java's \emph{null}.
\emph{null} is a subtype to all reference types
represented in Clojure by \lstinline|nil|. Unlike Java's type system
Clojure explicitly separates \emph{null} and reference types, allowing
Typed Clojure to express
\emph{nullable}
\footnote{A type is \emph{nullable} if it may also be an instance of \emph{null},
which is expressed in Typed Clojure by creating a union of the reference type and \lstinline|nil|.}
positions.

\begin{lstlisting}[caption=Java interoperability with Typed Clojure]
(ns typed.test.interop
  (:import (java.io File))
  (:require [typed.core :refer [ann non-nil-return check-ns]]))

(ann f File)
(def f (File. "a"))

(ann prt (U nil String))
(def prt (.getParent ^File f))

(non-nil-return java.io.File/getName :all)
(ann nme String)
(def nme (.getName ^File f))

\end{lstlisting}

This example shows how Typed Clojure handles \emph{null} while creating and
using an instance of \emph{java.io.File}\footnote{See Javadoc: http://docs.oracle.com/javase/1.4.2/docs/api/java/io/File.html}.

Typed Clojure checks calls to Java constructors by requiring the provided
arguments be acceptable input to at least one constructor for that Class.
In this case, \emph{java.io.File} has a constructor accepting a \emph{java.lang.String}
argument, so \lstinline|(File. "a")| is type safe. Java constructors never
return \emph{null}, so Typed Clojure assigns the return type to be \lstinline|File|.
This constructor is equivalent to \lstinline|[String -> File]| in Typed Clojure.

Next, we see how Typed Clojure's default behaviour treats method return positions as nullable.
The \emph{java.io.File} instance method \emph{getParent}
is equivalent to \lstinline|[-> (U nil String]| in Typed Clojure. This happens to be
a valid approximation of the method as \emph{getParent} returns \emph{null} 
``if the pathname does not name a parent directory''\footnote{See Javadoc: http://docs.oracle.com/javase/1.4.2/docs/api/java/io/File.html}.
On the other hand, the instance method \emph{getName} always returns an
instance of \emph{java.lang.String}, so we set the return position of
\emph{getName} to non-nil with \lstinline|typed.core/non-nil-return|
\footnote{The second parameter to \lstinline|non-nil-return| specifies which arities to assume non-nil 
returns, accepting either a set of parameter counts of the relevant arities, or \lstinline|:all|
to override all arities of that method.}.
