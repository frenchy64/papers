\appendix
\chapter{Heterogeous map type theory prototype}
\label{sec:hmaptheory}

\section{Operational Semantics}
 
%$$
%\begin{tdisplay}{Test1}
%\begin{altgrammar}
%  \s{},\t{} &::= & \int \alt \bool \alt {\proctype {\s{}} {\t{}}} 
%  &\mbox{Types}\\
%  \M{}, \N{}, \P{}, \dots &::=& \x{} \alt \abs{\x{}}{\M{}} \alt
%  \comb{\M1}{\M2}  &\mbox{Raw Terms} \\
%\end{altgrammar}
%\end{tdisplay} 
%$$

 
$$
\begin{tdisplay}{Syntax of Terms}
\begin{altgrammar}
  \exp{} &::=& (\c{} \overrightarrow{\exp{}})         % function application, multiple arguments
             \alt \{ \overrightarrow{\exp{}\ \exp{}} \} % map literal
             &\mbox{Expressions} \\ 
  \v{} &::=& \k{} \alt \nil{}
              \alt \{ \overrightarrow{\v{}\ \v{}} \}   % map literal
              &\mbox{Values} \\
  \c{} &::=& \assoc{} \alt \dissoc{} \alt \get{}
              &\mbox{Constants}
\end{altgrammar}
\end{tdisplay}
$$

$$
\begin{tdisplay}{Evaluation Contexts}
  \begin{altgrammar}
    \E{} &::=& [ ] % application rules
              \alt (\c{}\ \overrightarrow{\v{}}\ \E{}\ \overrightarrow{\exp{}}) % eval arguments left-to-right
              % map rules
              \alt \{\overrightarrow{\v{}\ \v{}}\ \E{}\ \exp{}\ \overrightarrow{\exp{}\ \exp{}} \} % key first
              \alt \{\overrightarrow{\v{}\ \v{}}\ \v{}\ \E{}\ \overrightarrow{\exp{}\ \exp{}} \}   % value next
              &\mbox{Evaluation Contexts}
  \end{altgrammar}
\end{tdisplay}
$$ 
 
% \mapsto for updating maps
$$
\begin{tdisplay}{Operational Semantics}
\begin{array}{c}
{\inferrule[E-Assoc]
  { {v_1 = \{ \overrightarrow{\v{}\ \v{}} \} } \\
    {v_4 = v_1\ with\ entry\ v_2\ to\ v_3} }
    {{(\assoc{}\ v_1\ v_2\ v_3) \hookrightarrow v_4} }} \\
\\
{\inferrule[E-Dissoc]
  { {v_1 = \{ \overrightarrow{\v{}\ \v{}} \} } \\
    {v_3 = v_1\ without\ entry\ indexed\ by\ v_2} }
  {(dissoc\ v_1\ v_2) \hookrightarrow v_3} } \\
\\
{\inferrule[E-GetMapExist]
  { {v_1 = \{ \overrightarrow{\v{}\ \v{}} \} } \\
    {v_2\ v_3\ in\ v_1} }
%    {v_1\ has\ entry\ with\ key\ v_2} \\
%    {v_3 = corresponding\ value\ in\ entry\ with\ key\ v_2\ in\ v_1}
  { {(get\ v_1\ v_2) \hookrightarrow v_3} }} \\
\\
{\inferrule[E-GetMapNotExist]
  { {v_1 = \{ \overrightarrow{\v{}\ \v{}} \} } \\
    {v_1\ has\ no\ entry\ with\ key\ v_2} \\
    {v_3 = \nil{}} }
  { {(get\ v_1\ v_2) \hookrightarrow v_3} }}
\end{array} 
\end{tdisplay} 
$$

$$
\begin{tdisplay}{Type Syntax}
\begin{altgrammar}
  \T{} &::=& \quad \Top \alt \Bottom \alt \Nil \alt \k{} \alt \{ \overrightarrow{\k{}\ \T{}} \} 
             \alt {\IPersistentMap {\T{}} {\T{}}} \\
             && \alt ( \vee\ \overrightarrow{\T{}} )
                \alt ( \wedge\ \overrightarrow{\T{}} )
\end{altgrammar}
\end{tdisplay}
$$

% what about get? Core Type Rules

$$
\begin{tdisplay}{Core Type Rules}
\begin{array}{c}
{\inferrule[T-AssocHMap]
  { {\Gamma \vdash {\hastype {{\exp{}}_{m}} {{\T{}}_{m}}} } \\
    {\Gamma \vdash {\hastype {{\exp{}}_{k}} {{\k{}}}} } \\
    {\Gamma \vdash {\hastype {{\exp{}}_{v}} {{\T{}}_{v}}} } \\
    {\vdash {\issubtype {{\T{}}_{m}} {\TopHMap} }} }
  { {\Gamma \vdash {\hastype {( \assoc{}\ {\exp{}}_{m}\ {\exp{}}_{k}\ {\exp{}}_{v})} 
        {(update\_hmap\ {{\T{}}_{m}}\ {{\k{}}}\ {{\T{}}_{v}})}}}}} \\
\\
{\inferrule[T-AssocPromote]
  { {\Gamma \vdash {\hastype {{\exp{}}_{m}} {{\T{}}_{m}}} } \\
    {\Gamma \vdash {\hastype {{\exp{}}_{k}} {{\T{}}_{k}}} } \\
    {\Gamma \vdash {\hastype {{\exp{}}_{v}} {{\T{}}_{v}}} } \\
    {\vdash {\issubtype {{\T{}}_{m}} {\TopIPersistentMap}}} \\
    {\T{} = (promote\_hmap\ {\T{}}_{m}\ {\T{}}_{k}\ {\T{}}_{v})} }
  { {\Gamma \vdash {\hastype {( \assoc{}\ {\exp{}}_{m}\ {\exp{}}_{k}\ {\exp{}}_{v})} {\T{}}}}}} \\
\end{array}
\end{tdisplay}
$$

$$
\begin{tdisplay}{Type Metafunctions}
\begingroup
\fontsize{10pt}{12pt}
\begin{altgrammar}
  (update\_hmap\ \{ \overrightarrow{{\T{}}_{k}\ {\T{}}_{v}} \}\ {\k{}}_1\ {\T{}}_1) = 
                 \{ \overrightarrow{{\T{}}_{k}\ {\T{}}_{v}}\ {\k{}}_1\ {\T{}}_1 \}\ \\
  (update\_hmap\ (\wedge\ \overrightarrow{{\T{}}})\ {\T{}}_{k}\ {\T{}}_{v}) = 
                               (\wedge\ \overrightarrow{(update\_hmap\ {\T{}}\ {\T{}}_{k}\ {\T{}}_{v})}) \\
  (update\_hmap\ (\vee\ \overrightarrow{{\T{}}})\ {\T{}}_{k}\ {\T{}}_{v}) = 
                               (\vee\ \overrightarrow{(update\_hmap\ {\T{}}\ {\T{}}_{k}\ {\T{}}_{v})})
  \\ \\
  (promote\_hmap\ \{ \overrightarrow{{\T{}}_{k}\ {\T{}}_{v}} \}\ {\T{}}_{kn}\ {\T{}}_{vn})
                                    = {\IPersistentMap {(\vee\ \overrightarrow{{\T{}}_{k}}\ {\T{}}_{kn})}
                                                       {(\vee\ \overrightarrow{{\T{}}_{v}}\ {\T{}}_{vn})}} \\
  (promote\_hmap\ {\IPersistentMap {{\T{}}_{k}} {{\T{}}_{v}}} \ {\T{}}_{kn}\ {\T{}}_{vn})
                                    = {\IPersistentMap {(\vee\ {\T{}}_{k}\ {\T{}}_{kn})}
                                                       {(\vee\ {\T{}}_{v}\ {\T{}}_{vn})}}
\end{altgrammar}
\endgroup
\end{tdisplay}
$$

%$$
%\begin{tdisplay}{Subtyping}
%\begin{array}{c}
%\inferrule[S-Refl] { } {\vdash {\issubtype {\T{}} {\T{}}}} \\
%\inferrule[S-Any] { } {\vdash {\issubtype {\T{}} {\Top}}}
%\inferrule[S-UnionSuper] 
%  {\ } 
%  {\vdash {\issubtype {\T{}} {\Top}}}
%\end{array}
%\end{tdisplay}
%$$

\chapter{Dissertation Proposal}

\section*{Background} 

% In this section you should give some background to your
% research area. What is the problem you are tackling, and why is it
% worthwhile solving? Who has already done some work in this area,
% and what have they achieved?

Dynamically typed languages (also known as monotyped) are designed to be convenient for
writing programs quickly, and aspire to get out of the programmer's
way as much as possible. When programs grow large and stabilize,
some features of static languages are missed, specifically static
type checking. 

Tobin-Hochstadt notes that ``untyped scripts are difficult to 
maintain over the long run''~\cite{Tob10} because types
contain valuable design information.
He developed Typed Scheme~\cite{Tob10} 
to safely and incrementally port existing Scheme code, a dynamic language, to
Typed Scheme, a static language.

Clojure is a dynamically typed, functional language with implementations for the Java Virtual
Machine and Common Language Runtime, and compiling to Javascript. Clojure
is also a Lisp, which makes it a good candidate to test the generality of
ideas developed for Typed Scheme~\cite{Tob10}.

\section*{Aim} 

% Now state explicitly the hypothesis you aim to
% test. Make references to the items listed in the Reference section
% that back up your arguments for why this is a reasonable
% hypothesis to test, for example the work of Knuth~\cite{knuth}.
% Explain what you expect will be accomplished by undertaking this
% particular project.  Moreover, is it likely to have any other
% applications?

My goal is to develop a prototype optional static type system for Clojure, 
eventually intended for practical use.

It will be based on the lessons learnt throught the development
of Typed Scheme, and as a response to Tobin-Hochstadt's~\cite{Tob10}
suggestion to add type systems to other existing dynamic languages.

There are several challenges to creating a satisfactory type system for Clojure.

Multimethods play a significant role in Clojure, and a satisfactory
type system for Clojure should understand them to some degree.
There is some experience statically typing multimethods~\cite{MS02}, 
but this is a challenge and may not be addressed in the
timeframe proposed, beyond initial consideration of how it would
fit with the rest of the system.

Clojure's core library includes variable-arity functions which
are not easily typed with current static type systems. 
Providing satisfactory types for functions 
like Clojure's 'map', 'filter', and 'reduce' require
support for non-uniform variable arity polymorphism. 
Strickland, Tobin-Hochstadt and Felleisen~\cite{STF09}.
describe their approach for non-uniform variable arity polymorphism, as used
in Typed Racket. 
At a minimum, the proposed project will include broadly identifiying 
what new issues arise when adapting this approach to Clojure, 
with a prototype design if these issues are not too major.

Occurence typing is type checking technique developed for Typed Scheme~\cite{TF08}
and improved for Typed Racket~\cite{TF10}.
It helps the type checker understand common programming idioms 
with minimal type annotations.
The proposed project will include a comparison of these approaches in
the context of Clojure, and a prototype design if a satisfactory approach 
is designed and no major issues are identifed.

Ensuring vigorous type safety is an important aspect of Typed Scheme~\cite{Tob10},
especially when interacting between untyped and typed modules.
I do not expect to concentrate on every combination of cross-module
interaction. In particular, safely using typed code from untyped code
will not be designed. Safe interaction between typed modules
however is needed for a practical type checker. I
will attempt to design a viable strategy or identify issues that
require further consideration.

\section*{Method}

% In this section you should outline how you intend to go
% about accomplishing the aims you have set in the previous
% section. Try to break your grand aims down into small,
% achievable tasks. Try to estimate how long you will
% spend on each task, and draw up a timetable for each
% sub-task.

The milestones for this project are broken into release milestones
for the prototype library.

My goal is to finish all features listed in releases below
and to begin work on the challenges, listed last.

For each of these it's not yet fully clear what novel issues may 
arise in adapting existing techniques to Clojure.  For each item 
below the project will involve either implemention or identification 
of the novel issues that make implementation difficult.  
The classification below reflects the level of uncertainity 
regarding such issues.

\begin{description}
\item[0.1]
  \begin{itemize}
  \item Union types
  \item Basic Local Type Inference algorithm
  \item Fixed arity function types
  \item Typed deftype (class definitions)
  \item Uniform variable-arity function types
  \item Enough functions annotated from clojure.core for proof-of-concept
  \item Bounded Type variables
  \end{itemize}
\item[Challenges]
  \begin{itemize}
  \item Devise strategy for type inference (eg. occurence typing)
  \item Manage interactions between typed namespaces
  \item Mutable reference types
  \item Non-uniform variable arity polymorphism
  \item Typing Multimethods
  \item Fine Grained Hash-Map Types
  \end{itemize}

\end{description}
\section*{Software and Hardware Requirements}
% Outline what your specific requirements will be with regard
% to software and hardware, but note that any special requests
% might need to be approved by your supervisor and the Head of
% Department.

Linux environment with Java, git, and maven installed.
No issues are anticipated with regards to access to these.

% Overall, you should aim to produce roughly a two page document
% (and certainly no more than four pages)
% outlining your plan for the year.
