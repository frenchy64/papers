\documentclass{cshonours}

\usepackage{mmm}
\usepackage{mathpartir}
\usepackage{clj-grammar}
\usepackage{url}
\usepackage[style=alphabetic]{biblatex}
\addbibresource{bibliography.bib}

\usepackage{listings}
\lstset{ %
  language=Lisp,                % choose the language of the code
  columns=fixed,basewidth=.5em,
  basicstyle=\small\ttfamily,       % the size of the fonts that are used for the code
  %numbers=left,                   % where to put the line-numbers
  %numberstyle=\small\ttfamily,      % the size of the fonts that are used for the line-numbers
  %stepnumber=1,                   % the step between two line-numbers. If it is 1 each line will be numbered
  %numbersep=5pt,                  % how far the line-numbers are from the code
  %backgroundcolor=\color{white},  % choose the background color. You must add \usepackage{color}
  %showspaces=false,               % show spaces adding particular underscores
  showstringspaces=false,         % underline spaces within strings
  %showtabs=false,                 % show tabs within strings adding particular underscores
  frame=single,           % adds a frame around the code
  %tabsize=2,          % sets default tabsize to 2 spaces
  captionpos=t,           % sets the caption-position to bottom
  breaklines=true,        % sets automatic line breaking
  breakatwhitespace=true,    % sets if automatic breaks should only happen at whitespace
  %escapeinside={\%*}{*)},          % if you want to add a comment within your code
}

\title{A Practical Optional Type System for Clojure}
\author{Ambrose Bonnaire-Sergeant}

\keywords{Programming languages, optional type systems, Clojure}
% TODO What are categories? cshonours requires them - but commented out now
%\categories{}

\begin{document}

\maketitle

\begin{abstract}

%Traditionally there has been a clear distinction between 
%programming languages that perform type checking at compile-time, and those that do not.
%In particular, the early wide-spread programming languages ALGOL and Lisp
%could be divided this way: ALGOL performs compile-time (or static) type checking,
%and Lisp does not, and most recent languages can be similarly classified.
%Static type checking eliminates many common user errors that are otherwise unnoticed or are caught 
%when the program is run.
%On the other hand, a language without static type checking can often be designed 
%to be more flexible and forgiving, making it easy, for example, to have a \emph{map} function that can map functions
%with any number of arguments (as in most Lisp dialects).
%
%A recent trend is to aim to combine the advantages of both kinds of languages by adding optional static 
%type systems to languages without static type checking. 
%A major challenge in these combinations is continuing to support the idiomatic style of the original 
%language while performing accurate enough checking to catch similar errors to standard statically
%typed languages. Some recent attempts appear to have finally produced type systems that have overcome this obstacle.
%Notably Typed Racket, a typed sister language of the Lisp dialect Racket, is able to statically type 
%check most idiomatic Racket code.
%These optional type systems have some important differences from standard static
%type systems, for example they often include \emph{union} types.
%

Dynamic programming languages often abandon the advantages of static type checking 
in favour of their characteristic convenience and flexibility.
Static type checking eliminates many common user errors at compile-time that
are otherwise unnoticed, or are caught later in languages without static type checking.
A recent trend is to aim to combine the advantages of both kinds of languages by adding \emph{optional} static 
type systems to languages without static type checking, while preserving the languages idioms and style.

This dissertation describes my work on designing an optional static type system for the Clojure programming language,
a dynamically typed dialect of Lisp, based on
the lessons learnt from several projects, primarily Typed Racket.
This work includes designing and building a type checker for Clojure running on the Java Virtual Machine.
Several experiments are conducted using this prototype, particularly involving
existing Clojure code that is sufficiently complicated that
type checking increases confidence that the code is correct.
For example, nearly all of \emph{algo.monads}, a Clojure Contrib library for monadic programming, 
is able to be type checked.
Most monad, monad transformer, and monadic function definitions can be type checked,
usually by adding type annotations in natural places like function definitions.

There is significant future work to fully type check all Clojure features and idioms.
For example, multimethod definitions and functions with particular constraints on the number of 
variable arguments they accept (particularly functions taking only an even number of variable arguments) are troublesome. 
Also, there are desirable features from the Typed Racket project that are missing, such
as automatic runtime contract generation and a sophisticated blame system, 
both which are designed to improve error messages when mixing typed and untyped code in similar systems.

Overall, the work described in this dissertation leads to the conclusion that it appears to 
be both practical and useful to design and implement an optional static type system for the
Clojure programming language.

\end{abstract}

\begin{acknowledgements}

The past year has included by far my most enjoyable and motivating programming-related experiences.

Last November I attended my first programming conference, Clojure Conj 2011 (Raleigh, NC),
where I delivered a talk on logic programming with Clojure and met some of my programming heroes.
A week later my supervisor Rowan Davies invited me to visit Carnegie Mellon University
where he worked on his PhD project. Most memorably, I was invited to a meeting between
Rowan and his PhD supervisor Frank Pfenning.

This trip was funded by donations from the programming community. Thank you to all who helped me get there:
this work was directly inspired from conversations during this period.

Many thanks go to my supervisor Rowan. His influence as a lecturer inspired my exploration of
functional programming languages. He has exceeded my expectations as an honours supervisor,
and we plan to continue to collaborate in this area.
Rowan's experience with refinement types brought a unique perspective to this project
which was very welcome.

I am grateful to Sam Tobin-Hochstadt for his pioneering work in this area of research.
He has always been happy to offer assistance and advise throughout this project, which always proved invaluable.

I would also like to thank David Nolen for mentoring this project as part of Google Summer of Code 2012,
and Chas Emerick, without whom I would have never even considered attending Clojure Conj 2011
(and this work probably would not have happened).
Also, Daniel Spiewak, Dan Friedman and William Byrd, Jim Duey, Baishampayan Ghose, Sam Aaron, and
Steve Strickland, thank you for your help and inspiration.
Thank you Rich Hickey for creating Clojure, it is a thrill to build tools that make
this great language even better.

Fittingly, I gave a talk on Typed Clojure at Clojure Conj 2012, which was extremely satisfying.

Finally, thanks to my family, friends and to my girlfriend Tamara, each for your love and support.

\end{acknowledgements}

\tableofcontents

\chapter{Introduction}

\section{Thesis}

\emph{It is practical and useful to design and implement an optional typing system 
for the Clojure programming language using bidirectional checking that allows Clojure programmers to continue 
using idioms and style found in current Clojure code.}

\section{Motivation}

In the last decade it has become increasingly common to enhance
dynamically typed languages with static type systems. 
This is idea not new, but recent attempts are noteworthy for 
their broad success in matching many of the advantages of statically typed
languages, notably due to the use of \emph{bidirectional checking}.
Instead of always attempting to infer types, this algorithm relies on programmer annotations
appearing in some natural places such as giving the type of each top-level function.

The Clojure programming language is a dynamically typed dialect of Lisp invented
by Hickey, designed to run on popular platforms\footnote{http://clojure.org/}.
It emphasises functional programming with immutable data structures
and provides direct interoperability with its host platform.
Notable implementations of Clojure exist for the Java Virtual Machine (JVM),
the Common Language Runtime, and for Javascript virtual machines.
At the current time, Clojure on the JVM is the most mature implementation,
and therefore this project focuses on the JVM implementation.

Clojure has attracted wide-spread users in part by concentrating on pragmatism.
Performance is a key feature, for example the JVM implementation of Clojure
offers ways to access Java-like speed for certain operations.
Also, Clojure's extensive host interoperability offers Clojure programmers
access to existing libraries for their platform, such as the vast Java library ecosystem.
By coupling pragmatic necessities with elegant features like Lisp-style macros, functional programming,
and immutability by default, Clojure is a compelling general purpose programming language.

Recently a number of languages have been created to or been modified to support aspects
of both static and dynamic typing.
Dart~\cite{Dart2012} (created by Google) is dynamically typed but offers a simple form of optional
static typing that specifically do not affect runtime semantics.
Typescript~\cite{TypeS2012} (created by Microsoft) adds an optional type system to Javascript,
a well-known dynamic language.
Typed Racket~\cite{TF08,Tob10} (by Tobin-Hochstadt et al.) goes further by offering safe interoperability
between typed and untyped modules by generating appropriate runtime assertions based
on expected static types.

When a static type checker is not available, which describes the situation
for most dynamic languages, other techniques are used for checking
type invariants. For example, ``design by contract'' is often used,
introduced by Meyer for the Eiffel language~\cite{Mey92},
in which the programmer defines contracts that are enforced at runtime.
Unit testing is also a popular verification technique in dynamic languages.
Clojure adopts these approaches, providing easy syntax for defining 
Eiffel-inspired pre- and post-conditions and a library for writing unit tests.

Static type systems, however, are still desirable for many situations.
Powerful type systems like ML's~\cite{Mil97} and Haskell's~\cite{Mar10} have proved particularly useful
when complicated programming styles are required. For example,
Haskell's advanced static type system helps the programmer write correct monadic code
(as detailed by Wadler~\cite{Wad95})
especially in more complicated situations like combining monads via monad transformers.

\subsection{Why implement an optional type system for Clojure?}

The initial motivation for implementing an optional type system
for Clojure was an anecdotal account of the struggles
of a Clojure programmer. In an apparently heroic effort, he managed to 
implement a Clojure library for conduits, an advanced form of ``pipes'',
using arrows, a generalisation of monads.
\footnote{CONDUIT REFERENCE TODO}
Conduits also significantly surpass monads in complexity and are usually reserved
for languages with advanced type systems like Haskell.

He highlighted a strong desire for a type system for several reasons.
Firstly, to verify the correctness of the library.
Without a static type system, it is a significant task
to verify such an implementation as correct due to heavy use
of higher-order functions.
Secondly, to aid him while writing the library.

Since then, the Clojure community has shown significant interest in this work.
I also developed this project as a Google Summer of Code 2012 project,
after it was selected by the Clojure community as a Clojure Google Summer of Code
project for 2012.
\footnote{CITE GSOC, or FOOTNOTE}
I have also been invited to speak on this project at the Clojure Conj
2012 conference in November, the main international conference related to Clojure.

\subsection{What kind of type system does Typed Clojure provide?}

% refinements check invariants of programs
% while ordinary static types check whether programs
% are basically meaningful - Rowan + Frank P
% Page 1 Rowan's PhD

There are many concepts associated with \emph{types} and \emph{type systems} in both the
literature and informal discourse.
A programmer who uses dynamically-typed languages may have a drastically different notion
of what a type is than, say, a programmer preferring languages with advanced static type systems.
There is some debate as to whether optional static type systems like Typed Clojure
can even be called a type system. We choose to
follow the terminology of Pfenning~\cite{Pfe08} and Reynolds~\cite{Rey01},
where such optional type systems are \emph{extrinsic} type systems, and more
traditional type systems are \emph{intrinsic} type systems.
This distinction has a long history, originating in work in the $\lambda$-calculus 
by Church \footnote{CHURCH FROM ROWANS DISSERTATION} and Curry \footnote{CURRY FROM ROWANS DISSERTATION},
leading to intrinsic types and extrinsic types also being called Church types and
Curry types often in the literature.

An ordinary static type system is used to check whether programs are basically
meaningful. Pfenning and Reynolds call these type systems \emph{intrinsic}. A language with an intrinsic
static type system has a run-time semantics that depends on the types associated
with variables and expressions during type checking.
For example, C, Java, ML, and Haskell have intrinsic types.
This means programs written in these languages must pass the type checker before being run.

A static type system is \emph{extrinsic} when runtime semantics
does not depend on a static type system. In other words, passing a static type checker
is not essential to running programs. A dynamically typed language can be viewed
as having a trivial static type system that supports exactly one type,
a view advocated by Harper~\cite{Har12}
and common in the literature on static types (for example, Pierce~\cite{Pie02}).

\section{Typed Clojure through Examples}

% Hello world
This section introduces Typed Clojure with example code. 
Typed Clojure is developed for the JVM implementation of Clojure, therefore
the rest of this chapter uses that implementation.
An attempt is
made to introduce some Clojure syntax and semantics to those unfamiliar or needing a refresher.
A basic knowledge of Lisp syntax is handy, but a brief tutorial is given
for newcomers.

\subsection{Preliminary: Lisp Syntax}

The core of understanding Lisp syntax when coming from a popular language
like Java or Javascript can be summarised by these points.

\begin{itemize}
  \item Operators are always in prefix position.
  \item Invocations are always wrapped in a pair of balanced parenthesis.
  \item Parenthesis start to the left of the operator.
\end{itemize}

For example, the Java expressions \emph{(1 + 2) / 3} is written in Lisp pseudocode \lstinline|(/ (+ 1 2) 3)|
and \emph{numberCrunch(1, 2)} written \lstinline|(numberCrunch 1 2)|.

Clojure also adds other syntax:

\begin{itemize}
  \item Prefixing \lstinline|:| to a symbol defines a \emph{keyword}, often used for map keys. eg. \lstinline|:my-keyword|.
  \item Square brackets delimit vector literals. eg. \lstinline|[1 2]| is a 2 place vector.
  \item Curly brackets define map literals. eg. \lstinline|{:a 1 :b 2}| is a map from 
        \lstinline|:a| to \lstinline|1| and \lstinline|:b| to \lstinline|2|.
  \item Commas are always optional and treated as whitespace, but often used to show the intended structure.
\end{itemize}

\subsection{Simple Examples}

We begin with the obligatory \emph{Hello world} example.

\begin{lstlisting}[caption=Typed Hello world, label=lst:helloworld]
(ns typed.test.hello-world
  (:require [typed.core :refer [check-ns]]]))

(println "Hello world")
\end{lstlisting}

At this point, it is worth understanding Clojure's namespacing feature.
Clojure code is always executed in a \emph{namespace}, and each file of Clojure code should 
have a \lstinline|ns| declaration with the namespace name and its dependencies,
which switches the current namespace and executes the given dependency commands.
There is one special namespace, \lstinline|clojure.core| which is
loaded with every namespace, implicitly ``referring'' all its vars in the namespace.
For example, \lstinline|ns| refers to the var \lstinline|clojure.core/ns|,
similarly \lstinline|println| refers to \lstinline|clojure.core/println|
(vars are global bindings in Clojure).

The example in listing \ref{lst:helloworld} declares a dependency to 
\lstinline|typed.core|, Typed Clojure's main namespace. It also refers the var \lstinline|typed.core/check-ns|
into scope (\lstinline|check-ns| is the top level function for type checking a namespace).
Other than this dependency, this is identical to the untyped \emph{Hello world}.

More complex code may require extra annotations to type check:

\begin{lstlisting}[caption=Annotating vars in Typed Clojure (adapted from a Typed Scheme/Racket example by Tobin-Hochstadt~\cite{Tob10})]
(ns typed.test.collatz
  (:require [typed.core :refer [check-ns ann]]))

(ann collatz [Number -> Number])
(defn collatz [n]
  (cond
    (= 1 n) 
      1
    (and (integer? n) 
         (even? n)) 
      (collatz (/ n 2))
    :else 
      (collatz (inc (* 3 n)))))
\end{lstlisting}

(\lstinline|(defn [...] body)| expands to \lstinline|(def (fn [...] body))|
where \lstinline|def| defines a new var and \lstinline|fn| creates a function value).

In this example, we define a new var \lstinline|collatz|
(when unambiguous, I omit the qualifying namespace/package for the remainder of the chapter).
Typed Clojure requires all 
used vars to be annotated. Here \lstinline|typed.core/ann|, a Typed Clojure macro for annotating vars, 
annotates \lstinline|collatz| to be a function from \lstinline|Number| to 
\lstinline|Number| (\lstinline|Number| refers to \lstinline|java.lang.Number|,
due to every Clojure namespace implicitly importing all Classes in the \lstinline|java.lang| package,
the equivalent of Java's \lstinline|import java.lang.*;|).

\subsection{Datatypes and Protocols}

As well as \lstinline|def| and \lstinline|defn| definitions,
Clojure programmers typically include datatype and protocol
definitions. 
Protocols are similar to Java interfaces\footnote{JAVA INTERFACE REFERENCE} and datatypes to classes implementing interfaces.
We can annotate datatype and protocol definitions similarly.

\begin{lstlisting}[caption=Annotating protocols and datatypes in Typed Clojure]
(ns typed.test.deftype
  (:require [typed.core 
             :refer [check-ns ann-datatype
                     tc-ignore ann-protocol AnyInteger]]))

(ann-protocol Age 
  :methods
  {age [Age -> AnyInteger]})
(tc-ignore
  (defprotocol Age
    (age [this])))

(ann-datatype Person 
  [[name :- String]
   [age :- AnyInteger]])
(deftype Person [name age]
  Age
  (age [this] age))

(age (Person. "Lucy" 34))
\end{lstlisting}

\lstinline|defprotocol| defines a new Clojure protocol\footnote{See http://clojure.org/protocols for a full description of protocols.}
with a set of methods. \lstinline|ann-protocol| is a Typed Clojure macro for annotating 
a protocol with the types of its methods.
In this example, we define a protocol \lstinline|Age| with an \lstinline|age| method,
which is really a first-class function taking the target object as the first parameter. The
type signature provided with \lstinline|ann-protocol|, here \lstinline|[Age -> AnyInteger]|
which is the type of the function taking an \lstinline|Age| and returning an \lstinline|AnyInteger|,
is for this function.

Invocations of \lstinline|defprotocol| are currently not able to be type checked
and are ignored by Typed Clojure by passing them to \lstinline|typed.core/tc-ignore|.

\lstinline|deftype|
defines a new Clojure datatype\footnote{See http://clojure.org/datatypes for a full description of datatypes.}
in the current namespace with a number of fields and methods. 
\lstinline|typed.core/ann-datatype| is a Typed Clojure form for annotating datatypes, including its field types.
In this example, we create a datatype \lstinline|typed.test.person.Person|
(datatype defintions generate a Java Class, where the current namespace is used as a starting point for its
qualifying package)
with fields \lstinline|name| and \lstinline|age| and 
extend the \lstinline|Age| protocol by implementing the \lstinline|age| method.

Java constructors are invoked in Clojure by suffixing the Class we want to instantiate with a dot.
Datatypes are implemented as Java Classes with immutable fields (by default) and a single constructor, taking as arguments its fields 
in the order they are passed to \lstinline|deftype|
\lstinline|(Person. "Lucy" 34)| constructs a new \lstinline|Person|
instance, setting the fields to their corresponding positional arguments.
Typed Clojure checks the datatype constructor to be the equivalent of 
\lstinline|[String AnyInteger -> Person]|.

Finally, Typed Clojure checks invocations of Protocol methods. It infers \lstinline|Person|
is an instance of \lstinline|Age| from the datatype definition, therefore \lstinline|(age (Person. "Lucy" 34))| is type-safe.
Since generally we only need to annotate definitions and not uses, the number of annotations is relatively small.

\subsection{Polymorphism}

Typed Clojure supports F-bounded polymorphism, first introduced by Canning, Cook, Hill and Olthoff~\cite{CCHOM89}. 
F-bounded polymorphism is an extension of bounded polymorphism, where polymorphic type variables
can be restricted by \emph{bounds}.
In particular, F-bounded polymorphism allows type variable bounds to recursively refer to the
variable being bounded. Typed Clojure supports upper and lower type variable bounds.
\footnote{F-BOUND EXAMPLE NEEDED}

For accurate type checking, Typed Clojure parameterises some of Clojure's data structures. For example,
the interface behind Clojure's \emph{seq} abstraction for sequences \lstinline|clojure.lang.Seqable| has one 
covariant parameter\footnote{Covariant means that \lstinline|(Seqable Integer)| is a subtype of \lstinline|(Seqable Number)|
because Integer is a subtype of Number.}.

\begin{lstlisting}[caption=Polymorphism in Typed Clojure]
...
(ann to-set 
     (All [x]
       [(U nil (Seqable x)) -> (clojure.lang.PersistentHashSet x)]))
(defn to-set [a]
  (set a))
...
\end{lstlisting}

In this example\footnote{When convenient, namespace declarations are omitted for the remainder of the chapter.}, 
we define \lstinline|to-set|, aliasing \lstinline|clojure.core/set|.
\lstinline|All| introduces a set of type variables to the body of a type,
here \lstinline|x| is used to define a relationship between the input type and return type.

\lstinline|(U nil (Seqable x))| is a common type in Typed Clojure, read as the union
of the type \lstinline|nil| and the type \lstinline|(Seqable x)|.
The vast majority of types for collection processing functions in the Clojure core library feature
it as an input type, where passing \lstinline|nil| either has some special behaviour 
or is synonymous with passing an empty \lstinline|Seqable|.
The ability to specify types that explicitly include or exclude \lstinline|nil| is one of the strengths
of Typed Clojure, and an aspect where it is more expressive than standard type systems like that for Java.

\subsection{Singleton Types}

Following Typed Racket, singleton types for certain values are provided
in Typed Clojure.
A singleton type is a type with a single member, like \lstinline|1|,
\lstinline|:a|, or \lstinline|"a"|\footnote{Strings are delimited by " in Clojure}.
Typed Clojure provides syntax for singleton types, either by passing
the value literal to the \lstinline|Value| primitive, or by prefixing
a quote (').

\begin{lstlisting}[caption=Singleton Types, label=lst:singletoneeg]
(ann k ':my-keyword)
(def k :my-keyword)
\end{lstlisting}

Listing \ref{lst:singletoneeg} shows a simple example of using
singleton types in Typed Clojure.
Singleton types are are discussed further in section \ref{sec:intersectionunion}.

\subsection{Heterogeneous Maps}

Clojure emphasises key-value paired, unordered \emph{maps} for many uses where other languages might use objects
Clojure provides a map literal using curly braces. For example,
\lstinline|{:a 1, :b 2}| is a map value with two key-value entries: from keyword key \lstinline|:a|
to value \lstinline|1|, and keyword key \lstinline|:b| to value \lstinline|2|. Note that commas are always
whitespace in Clojure and are included occasionally for readability.

Typed Clojure provides heterogeneous map types which captures this common
``maps as objects'' pattern. A heterogeneous map type has only \emph{positive}
information on the types of key-value entries. In other words, it conveys
whether a particular key is present, but not whether it is absent.
The implications of this is discussed in section \ref{ref:designhmap}.
Heterogeneous maps only support keys that are singleton Keyword types. This restriction is reflected
in the syntax for defining heterogeneous map types.

\begin{lstlisting}[caption=Heterogeneous map types in Typed Clojure]
...
(ann config '{:file String, :ns Symbol}))
(def config
  {:file "clojure/core.clj",
   :ns 'clojure.core})
...
\end{lstlisting}

This example checks \lstinline|config| to be a heterogeneous map
with \lstinline|:file| and \lstinline|:ns| keys, with values of
type \lstinline|String| and \lstinline|Symbol| respectively.

Heterogeneous vector and seq types are also provided and work similarly,
except for their lack of keys like maps, hence are just a sequence of types.

\subsection{Variable-Arity Functions}

Functions in Clojure are multi-arity (a very commonly used feature) which means a function
can be defined with several sets of function arguments and bodies (or \emph{arities})
and which arity is executed depends on the number of arguments passed
to the function. A function can have any number of arities with fixed parameters, and at most one arity with
variable-parameters. Each arity must have a unique number of
parameters and have a lower number of parameters than the ``variable arity'', if present.

Strickland, Tobin-Hochstadt, and Felleisen invented a calculus and corresponding
implementation in Typed Racket for variable-arity polymorphism~\cite{STF09}
that is sufficient to handle uniform and non-uniform variable-arity functions.
Typed Clojure includes an implementation of the most immediately useful parts of variable-arity
polymorphism using algorithms, nomenclature, and implementation based on this work.

\subsubsection{Uniform Variable-Arity}

A function with uniform variable parameters can treat its variable parameter
as a homogeneous list. 
Strickland et al.\ attaches a \emph{starred pre-type} \lstinline|T *| to the right of the fixed arguments
in a function type, where \emph{T} is some type, and we take an identical approach.
For example, \lstinline|+| in Clojure accepts any number of arguments
of type \lstinline|Number|, represented by the type \lstinline|[Number * -> Number]|.


\begin{lstlisting}[caption=Typing multi-arity functions, label=lst:noteq]
(ann not= (Fn [Any -> boolean]
              [Any Any -> boolean]
              [Any Any Any * -> boolean]))
(defn not=
  "Same as (not (= obj1 obj2))"
  ([x] false)
  ([x y] (not (= x y)))
  ([x y & more]
   (not (apply = x y more))))
\end{lstlisting}

It is common to find Clojure library functions that define seemingly redundant
function arities for performance reasons.
Listing \ref{lst:noteq} defines the multi-arity function \lstinline|not=|,
taken from the Clojure standard library \lstinline|clojure.core| that uses this pattern.
\lstinline|not=| has three arities in its definition, including
one that takes a variable number of arguments.
The \lstinline|Fn| type constructor builds an \emph{ordered function intersection type} from
function types. Each arity must have at least one matching function type associated with it.
The two ``fixed arities'' are given familiar function types, with one and two fixed parameters respectively.
The type given for the arity with variable arguments \lstinline|[Any Any Any * -> boolean]|
uses a starred pre-type to signify any number of arguments of type \lstinline|Any|
can be provided to the right of its fixed arguments.

\subsubsection{Non-uniform Variable-Arity Functions}

Where \emph{uniform} variable-arity function types use \emph{starred pre-types}, \emph{non-uniform}
variable-arity function types use \emph{dotted pre-types}.
Typed Clojure supports usages of \emph{non-uniform} variable-arity functions,
where the variable parameter is a heterogeneous list, represented by a \emph{dotted type variable}.

For example, the variable argument function \lstinline|clojure.core/map| takes a function and one or more sequences,
and returns the result of applying the function argument to each element of the sequences in a pair-wise fashion.
Its type is given in listing \ref{lst:maptype}.

\begin{lstlisting}[caption=Type signature for \lstinline|clojure.core/map|, label=lst:maptype]
(ann clojure.core/map
     (All [c a b ...]
       [[a b ... b -> c] (U nil (Seqable a)) (U nil (Seqable b)) ... b -> (LazySeq c)]))
\end{lstlisting}

By adding \lstinline|...| after the last type variable in an \lstinline|All| binder
we can introduce a \emph{dotted type variable} into scope, which is a placeholder for a sequence of types.
A \emph{dotted pre-type} \lstinline|T ... b| over \emph{base} \lstinline|T| (a type) and \emph{bound}
\lstinline|b| (a dotted type variable)
serves as a placeholder for this sequence of types.
Dotted pre-types must appear to the right of all fixed parameters in a function type,
and cannot be mixed with other kinds of variable parameters like starred pre-types.
When a sequence of types of length \emph{n} is associated with a dotted pre-type, 
the dotted pre-type is expanded to \emph{n} copies of \lstinline|T|.
One other special property of a dotted pre-type is that the bound \lstinline|b|
is \emph{in scope} as a \emph{normal} type variable in its base \lstinline|T|.

To demonstrate dotted pre-types, we use \lstinline|typed.core/inst| to instantiate
\lstinline|map|. \lstinline|inst| takes a polymorphic expression and a number of types
that are satisfactory for instantiating the polymorphic type and returns an expression
of the instantiated polymorphic type.

The instantiation

\begin{lstlisting}
(inst map Number boolean String)
\end{lstlisting}

returns an expression of type

\begin{lstlisting}
[[boolean String Integer -> Number] (U (Seqable boolean) nil) (U (Seqable String) nil) (U (Seqable Integer) nil) -> (LazySeq Number)]
\end{lstlisting}

However, if sufficient types are given, the instantiation for \lstinline|map| can be inferred.
The invocation

\begin{lstlisting}
(map (ann-form (fn [a b c] c) 
               [boolean String Number -> Number]) 
     [true false] ["mystr" "astr"] [1 4])
\end{lstlisting}

uses \lstinline|typed.core/ann-form| to assign an expected to
the first argument,
which is sufficient to infer the result type

\begin{lstlisting}
(LazySeq Number)
\end{lstlisting}


\subsection{Occurrence Typing}

It is common in Clojure, like other dynamically-typed languages, to
encode implicit type invariants in conditional tests.
In listing \ref{lst:numvec2} conditional tests are used
to refine the type of the bindings \lstinline|a| and \lstinline|b|.
Occurrence typing\cite{TF10} is a technique useful for capturing these kinds
of type invariants (occurrence typing is discussed in further detail in section \ref{sec:OccurrenceTyping}).

\begin{lstlisting}[caption=Example of occurrence typing in Typed Clojure, label=lst:numvec2]
(ann num-vec2 
     [(U nil Number) (U nil Number) -> (Vector* Number Number)])
(defn num-vec2 [a b]
  [(if a a 0) 
   (if b b 0)])
\end{lstlisting}

To check this example, occurrence typing infers type information based on the result of each test.
In Clojure, \lstinline|nil| and \lstinline|false| are false values and all other values are true.
The test \lstinline|(if a a 0)| follows the \emph{then} branch is \lstinline|a| is not \lstinline|nil|
or \lstinline|false|. When checking this branch, we can safely refine the type of \lstinline|a| to \lstinline|Number| from
\lstinline|(U nil Number)|. Similarly, following the \emph{else} branch refines the type of \lstinline|a|
to \lstinline|nil| from \lstinline|(U nil Number)|.

\subsection{Java Interoperability}
\label{sec:javainterop}

Typed Clojure supports integration with Java's type system during
interoperability from Clojure. This offers many of the same advantages
as using Java's type system from Java, for example it is a type
error to pass arguments of the wrong type to methods.
Typed Clojure also makes different decisions to Java's type system,
particularly in the treatment of Java's \emph{null}.

From the perspective of static types in Java, \emph{null} is included in all reference types.
\emph{null} is represented in Clojure by the value \lstinline|nil|. Unlike Java's type system
Typed Clojure explicitly separates \emph{null} and reference types giving Typed Clojure 
more accurate types: we can directly express \emph{nullable} and \emph{not nullable} types.
(A type is \emph{nullable} if it may also be \emph{null},
which is expressed in Typed Clojure by creating a union of the type and \lstinline|nil|.
A type is \emph{not nullable} if it doesn't include \emph{null}).

\begin{lstlisting}[caption=Java interoperability with Typed Clojure]
(ns typed.test.interop
  (:import (java.io File))
  (:require [typed.core :refer [ann non-nil-return check-ns]]))

(ann f File)
(def f (File. "a"))

(ann prt (U nil String))
(def prt (.getParent ^File f))

(non-nil-return java.io.File/getName :all)
(ann nme String)
(def nme (.getName ^File f))

\end{lstlisting}

This example shows how Typed Clojure handles \emph{null} while creating and
using an instance of \emph{java.io.File}.

Typed Clojure checks calls to Java constructors by requiring the provided
arguments be acceptable input to at least one constructor for that Class.
In this case, \emph{java.io.File} has a constructor accepting a \emph{java.lang.String}
argument, so \lstinline|(File. "a")| is type safe. Java constructors never
return \emph{null}, so Typed Clojure assigns the return type to be \lstinline|File|.
This constructor is equivalent to \lstinline|[String -> File]| in Typed Clojure.

Next, we see how Typed Clojure's default behaviour treats method return positions as nullable.
The \emph{java.io.File} instance method \emph{getParent}
is equivalent to \lstinline|[-> (U nil String]| in Typed Clojure. This happens to be
a valid approximation of the method as \emph{getParent} returns \emph{null} 
``if the pathname does not name a parent directory''\footnote{See Javadoc for \emph{java.io.File}: http://docs.oracle.com/javase/1.4.2/docs/api/java/io/File.html}.
On the other hand, the instance method \emph{getName} always returns an
instance of \emph{java.lang.String}, so we set the return position of
\emph{getName} to non-nil with \lstinline|typed.core/non-nil-return|
(the second parameter to \lstinline|non-nil-return| specifies which arities to assume non-nil 
returns, accepting either a set of parameter counts of the relevant arities, or \lstinline|:all|
to override all arities of that method).

Such annotations are in a sense assumptions; should they turn out to be wrong, Typed Clojure
will infer incorrect types. Typed Racket generates runtime contracts based on static types to ensure
that type errors are caught at runtime, and the original source of the error is blamed~\cite{WF09}.
A similar system for verifying that method annotations like \lstinline|non-nil-return|
actually describe correct types is planned as future work
(see section \ref{sec:contractsblame}).

% TODO
% What does this Lit review cover?
% Why things are included/not included?
% Background
% Heterogeneous Maps/record types
% Soft typing
% See Sam's dissertation
% Lisp type system background (197x)
% Why was it so hard?
% - bidirectional checking vs. global inference

\chapter{Literature Review}

In this chapter we fit the design of Typed Clojure with literature from related fields.
Forward-references are often given to link the discussed literature with this later parts
of this dissertation
where there is some relevance or influence.

%Typed Clojure is related to several existing systems for both typed and untyped languages.

\section{Dynamic Typing in Typed Languages}

Typed Clojure adds static type checking to an existing untyped language.
Coming from the other direction, Abadi, Cardelli, Pierce and R\'{e}my~\cite{ACPR95}
extend a static language to be more dynamic.
This work centres around adding a type \emph{Dynamic}, which represents
a dynamic type in a statically typed language.

Rossberg~\cite{Ros07} describes the advantages of having a type
\emph{Dynamic} over other approaches to making statically typed languages
more like dynamic ones and surveys work in this area. 
He says that type \emph{Dynamic}
allows aspects of dynamic style to be represented in statically
typed languages without compromising the type system.
However, he claims type \emph{Dynamic} is rarely used in practical languages
because it is too inconvenient in real programs.
He claims to have fixed these issues, but a practical language has yet to integrated 
his changes.

% cite: Dynamic Typing in Polymorphic languages (1994)
% - coming from the other direction
% - extending static languages to be more dynamic

% A more recent reference in the same area is "Typed Open Programming" a PhD dissertation from 2007 - at least have a look at 1.3.2 for a good summary of work on dynamics in statically typed languages since 1991 - probably it's good to cite this as a good recent survey of the area.
\section{Static Typing for Untyped Languages}

This section briefly reviews work on designing static type systems for untyped languages.

\subsection{Soft Typing}

Soft typing~\cite{CF91}
is an approach for ensuring type safety in untyped languages.
A soft type system infers types for programs, distinguishing between degrees
of potential type safety.
A soft type checker uses this information to 
preserve type safety by inserting appropriate checks 
and informs the programmer of potential inconsistencies.
For example, if the soft type system detects a portion of code is sometimes not type safe,
the soft type checker inserts a check that throws a runtime error upon unsafe usages.
Thus, soft type systems differ from traditional type systems in that type inference 
never fails and an inconsistency always results in a runtime check.

Wright and Cartwright~\cite{WC97} developed Soft Scheme, a
soft type system for Scheme. 
It extended earlier work by Cartwright and Fagan~\cite{CF91}
and Fagan~\cite{Fag91}, adding support for practical features such as
first-class continuations and variable-arity functions.
Soft Scheme does not require any extra type annotations.

Type systems for Scheme have since moved away from soft typing.
Alas, the only reasonably complete account of this transition 
appears to be slides from a talk by Felleisen~\cite{Fell09},
a leader in this area of research for over 20 years.
Felleisen comments that while Soft Scheme discovered type problems, 
it suffers from incomprehensible
error reporting that required PhD-level expertise to decipher. 

\subsection{Program Analysis}

Scheme then moved to program analysis techniques like Set Based Analysis (SBA).
They, however, suffered from modularity issues due to the lack of declared intended types.
MrSpidey\footnote{\url{http://www.plt-scheme.org/software/mrspidey/}}
is a static type checker for Racket that uses uses SBA.
Felleisen states the main problem with these kinds of systems: ``Mr Spidey and friends infer brittle and large
types; errors remain difficult to explain and fix''~\cite{Fell09}.

\subsection{Gradual Typing}

Gradual typing combines static and runtime type checking so programmers
can the most appropriate one for the situation.
A key feature of gradual type systems is its use of \emph{bidirectional checking}
like~\cite{PT00}, which requires type annotations in natural places like functions.
Type errors in gradual type systems are often more manageable than in Soft Typing and SBA,
often more comparable to type errors in statically typed languages.

Gradually typed languages have different degrees of runtime checking.
Typed Racket was developed as a path for module-by-module
porting of existing untyped Racket modules to a typed sister language~\cite{Tob10}.
Once a module is ported and type checked, it is protected from untyped modules
by inserting runtime checks.
Typescript~\cite{TypeS2012}, a gradual type system for Javascript, 
does not add runtime checks to untyped interactions.
Runtime checks affect performance, but give better errors and ensure
type errors result in the program failing quicker.

\section{Interlanguage Interoperability}

This section compares several existing languages that feature interlanguage interoperability.

Clojure is a dynamic functional language hosted on the Java Virtual Machine. It provides 
interoperability with Java libraries. As Clojure is a dynamically typed language, it does
not give strong type type guarantees at compile time whether interactions with Java
are type safe.

Scala is a statically typed language on the Java Virtual Machine offering integrated interoperability with Java, a typed language.
Scala objects and classes can ``inherit from Java classes and implement Java interfaces''~\cite{OCD+}
with the usual static type guarantees normal Scala code enjoys.
Scala offers an Option type to safely eliminate null pointers.
Java Generics are also fully supported by Scala, accounting for Scala support existential types

Typed Racket includes safe interoperability between any combination of typed and untyped 
Racket modules~\cite{Tob10,TF08}. 
Interactions with untyped modules are protected by adding runtime checks based on expected types.
Typed Racket implements a sophisticated blame calculus, based on Wadler~\cite{WF09}. It ensures 
error messages correctly \emph{blame} the source of type errors,
which can be difficult to determine in the presence of higher-order functions. 

\section{Record Types}
\label{sec:recordtypes}

O'Caml-style extensible record types has been the subject of extensive research 
(eg. Wand~\cite{Wan89}, Cardelli and Mitchell~\cite{CM91}, Harper and Pierce~\cite{HP91}).
Typed Clojure's heterogeneous map types show some resemblance to extensible
record types, but a survey of the literature did not reveal research flexible
enough to capture common usages of maps in Clojure.
This most likely implies that designing such a system for Typed Clojure would be hard.
For this reason, heterogeneous map types are kept as simple as possible while still being useful.

\section{Intersection, Union, and Singleton Types}
\label{sec:intersectionunion}

Intersection and union types are interesting type constructs relevant to capturing the complicated
types common in dynamic languages.
An expression of type \lstinline|(I a b)|, an intersection type including types \lstinline|a|
and \lstinline|b| in Typed Clojure, can be used safely in both positions expecting type \lstinline|a|,
and positions expecting type \lstinline|b|.
An expression of type \lstinline|(U a b)|, a union type including types \lstinline|a| in Typed Clojure,
and expressions of this type can be used safely in positions 
that expect a type that is \emph{either} type \lstinline|a|
or \lstinline|b|.
For example, \lstinline|(U Number Integer)| cannot be used in positions expecting \lstinline|Integer|,
but \lstinline|(I Number Integer)| can be used in positions expecting \lstinline|Integer|.

Intersection, union, and singleton types are best considered as extrinsic types.
Hayashi~\cite{Hay91} describes two type theories, ATT which includes
intersection, union, and singleton types, and ATTT which further extends ATT
to include refinement types, which can be classified as extrinsic types.
Typed Clojure can also be considered an extrinsic type system which supports
union, intersection, and singleton types, but in less sophisticated forms
than ATT and ATTT.
In particular, singleton types in ATTT a much more advanced than Typed Clojure,
which offers only singleton types for values like \lstinline|:a| and \lstinline|nil|
and, unlike ATTT, no way to parameterise over singletons.

Several interesting projects have used intersection or union types.
Intersection types were originally introduced by Coppo, Dezani-Ciancaglini, and Venneri~\cite{CDV81}
for the $\lambda$-calculus.
The use of union types to type check dynamically typed languages with soft typing systems
dates back to Cartwright and Fagan~\cite{CF91}.

Forsythe, a modern ALGOL dialect by Reynolds~\cite{Rey81,Rey96} was the first wide-spread 
programming language to use intersection types.
Uses of intersections in Forsythe include representing extensible record types
and function overloading.

Refinement types add a level of extrinsic types refining an existing intrinsic type system
in order to type check more detailed properties and invariants than standard static type systems
(described by Freeman and Pfenning~\cite{FP91}).
Refinement types have similarities with Typed Clojure, in that Typed Clojure
adds a level of extrinsic types on top of Clojure.
Intersection types are critical here to allow more than one property or invariant
to be expressed for a function.
SML CIDRE is a refinement type checker for Standard ML by Davies~\cite{Dav05}.

St-Amour, Tobin-Hochstadt, Flatt, and Felleisen
describe the \emph{ordered intersection types} used in Typed Racket~\cite{St12}
that provide a kind of function overloading.
Typed Clojure takes a similar approach for its representation of functions.
Typed Racket uses union types which are used in Typed Clojure in very similar ways,
like for creating ML-like ``datatypes''.

\section{Java Interoperability in Statically Typed Languages}

Typed Clojure features integration with Java's static type system
to help verify interoperability with Java as correct.
Section \ref{sec:interactionnull} presents Typed Clojure's treatment of
Java's \emph{null} pointer. 
Section \ref{designarrays} shows how Typed Clojure deals with shortcomings
of Java's static type system relating to primitive arrays.

Other statically typed languages support Java interoperability with
similar goals, most completely, Scala.
Scala~\cite{OCD+} is tightly integrated with Java. It manages interactions
with Java's \emph{null} pointer by using an \emph{Option} type, which (if
used correctly) provides strong elimination for \emph{null}.
\emph{null} is still not expressible as a type in Scala, however.
Arrays are parameterised by a non-variant (or invariant) parameter,
which are checked using \emph{conservative approximation}~\cite{OSV08}. For example,
this checks that covariant parameters are only used in covariant positions.

\section{Function Types}

There are several different approaches to representing functions in programming languages.

In typed languages like Haskell, functions are as simple as possible, taking a single argument.
A function with multiple arguments is represented by chaining several single-argument function
together, or by using lists, or using tuples. This first style is characterised by direct syntactic function currying, where applying
a function to less than its maximum number of arguments results in another function
that takes the remaining arguments.
Using tuples instead requires that the same number of arguments be supplied,
while using lists allows any number, but requires that all arguments have the same type.

In untyped languages like Scheme, functions can take any number of arguments. It is an
error to supply less than the minimum number of arguments to a function.
This style features sophisticated support for functions with variable-arity. For example,
functions can dispatch on the number of arguments provided, and supports a \emph{rest} parameter
as its last parameter which can accept any number of arguments.

In this regard, Clojure takes an approach identical to Scheme, and supports all the features
mentioned in the previous paragraph, and none characterised by Haskell-style functions.
For this reason, we ignore the tradeoffs associated between the two approaches 
and move directly to literature applicable to typing Scheme-style functions.

\subsection{Variable-Arity Polymorphism}
\label{sec:variablearity}

Strickland et al.\ invented a type system supporting variable-arity polymorphism ~\cite{STF09}
a version of which is included in the current implementation of Typed Racket.
Their main innovation centres around \emph{dotted type variables}, which represent a heterogeneous sequence
of types. Dotted type variables allow \emph{non-uniform} variable-arity function types,
which are used to check definitions and usages of functions with non-trivial rest parameters

For example in Clojure, the function \lstinline|map| takes a function and one or more sequences,
and returns the result of applying the function argument to each element of the sequences pair-wise.

\begin{lstlisting}[caption=An application of the non-uniform variable-arity function \lstinline|map|, label=lst:map]
(map + [1 2] [2.1 3.2]) 
;=> (3.1 5.2)
\end{lstlisting}

(Line comments in Clojure begin with \lstinline|;| and comments to the end of the line. We use \lstinline|;=>| to mean \emph{evaluates to}).

To statically check calls to \lstinline|map|, we must enforce the provided function argument can accept as many
arguments as there are sequence arguments to \lstinline|map|, and the parameter types of the provided function can accept
the pair-wise application of the elements in each sequence. This is a complex relationship between the variable parameters and
the rest of the function.
Listing \ref{lst:map} requires the first argument to \lstinline|map| to be a function of 2 parameters because
there are two sequence parameters. \lstinline|+| takes any number of \lstinline|Number| parameters, 
and applying pair-wise arguments of \lstinline|(Vector Long)| and \lstinline|(Vector Double)| 
results in types \lstinline|Long| and \lstinline|Double| being applied to \lstinline|+|. These are subtypes
of \lstinline|Number|, so the expression is well typed.

\section{Type Inference}

\subsection{Local Type Inference}

Typed Racket uses Local Type Inference~\cite{PT00}
as an inference and checking tool. Pierce and Turner~\cite{PT00}
divide Local Type Inference into
two complementary algorithms. \emph{Local type argument synthesis}
synthesises type arguments to polymorphic applications, and \emph{bidirectional
propagation} propagates type information both down and up the source tree,
known as \emph{checking} and \emph{synthesis} mode respectively.

\begin{lstlisting}[caption=Bidirectional checking algorithm with Typed Clojure pseudocode, label=lst:bidir]
(map (fn [[a :- Long] [b :- Float]]
       (+ a b))
     [1 2]
     [2.1 3.2])
;=> (3.1 5.2)
\end{lstlisting}

The pseudocode in listing \ref{lst:bidir} show both algorithms in action. Local type argument synthesis is able
to infer the type arguments to \lstinline|map| by observing the argument types of the first
argument to \lstinline|map| and the types of subsequent sequence arguments. Bidirectional checking
then \emph{synthesises} the resulting type of the expression by \emph{checking} each element
of \lstinline|[1 2]| is a subtype of \lstinline|a|, and each element of \lstinline|[2.1 3.2]| is a subtype of
\lstinline|b|. The result of the anonymous function argument is \emph{synthesised} from the type of
\lstinline|(+ a b)| as \lstinline|Float|. We now have sufficient information to 
synthesise the type of listing \ref{lst:bidir} to be \lstinline|(List Float)|.

Pierce and Turner split \emph{local type argument synthesis} into two further
algorithms: bounded, and unbounded quantification~\cite{PT00}. 
Typed Racket 
supports unbounded polymorphism~\cite{Tob10}, implementing the latter algorithm by Pierce and Turner.
Scala supports bounded quantification with F-bounded polymorphism~\cite{CCHOM89},
basing its type argument synthesis on the bounded quantification algorithm.

Pierce and Turner explicitly forbid~\cite{PT00}
attempting to synthesise type variables with interdependent bounds, including
F-bounds, having failed to devise an algorithm to infer these cases.
Scala's type argument synthesis implementation deviates from Pierce and Turner and supports these features.
I am not aware of papers specifically describing Scala's modifications, but they are at least inspired by
Scala's spiritual ancestors Generic Java~\cite{BOSW98} and Pizza~\cite{OW97}.

Hosoya and Pierce~\cite{HP99} reiterate two common problems with Local Type Inference:
``hard-to-synthesise arguments'' and ``no best type argument''. The first problem occurs because
both local type argument synthesis and bidirectional propagation cannot perform synthesis
simultaneously. 

\begin{lstlisting}[caption=Hard-to-synthesise expression, label=lst:hts]
(map (fn [a b] 
       (+ a b)) 
     [1 2] 
     [2.1 3.2])
\end{lstlisting}

Listing \ref{lst:hts} shows an example of this limitation,
here caused by both not providing type arguments to \lstinline|map| and not providing the parameter types of \lstinline|(fn [a b] (+ a b))|.
 Cases where both algorithms can simultaneously recover new type information are usually ``hard-to-synthesise''.
``No best type argument'' describes the situation where the results of local
type argument synthesis yield more than one type, and no type is better than the other. Sometimes we cannot recover and synthesis
fails.

\subsection{Colored Local Type Inference}

Scala's type checking uses Colored Local Type Inference~\cite{OZZ01},
a variant of Local Type Inference~\cite{PT00} specifically designed to
improve inference with certain kinds of Scala pattern matching expressions. It allows
 \emph{partial} type information to propagate down the syntax tree, instead of only full type information
as required by Local Type Inference.

\emph{Colored} types contain extra contextual information, including the propagation direction
and missing parts of the type. They are generally useful
for describing ``information flow in polymorphic type systems with propagation-based type inference''~\cite{OZZ01}. 
Colored Local Type Inference is a candidate for 

\section{Bounded and Unbounded Polymorphism}

Local type inference by Pierce and Turner~\cite{PT00}
describe two implementations of type variables, for bounded
and unbounded type variables. The bounded implementation is presented
as an optional extension  to the unbounded implementation, which preserves all
properties described in the Local Type Inference algorithm.

An unbounded type variable does not have subtype constraints.
Bounded type variables can have subtype constraints, and 
subsume unbounded type variables~\cite{PT00}, 
as a unbounded variable can be represented as a variable bounded
by the Top type.

Still, unbounded type variables have an advantage: their implementations are
are simpler in the presence of a \emph{Bottom} type. 
The constraint resolution algorithm for bounded variables
is more subtle, due to ``some surprising interactions between bounded quantifiers
and the \emph{Bot} type''~\cite{PT00}, described fully
by Pierce~\cite{Pie97}.

Typed Racket~\cite{TF08}
supports unbounded polymorphism, while Scala~\cite{OCD+}
supports an extended form of bounded polymorphism called
F-bounded polymorphism~\cite{CCHOM89}, which allows the
bound variable to occur in its own bound.
F-bounded polymorphism is useful in the context of object-oriented abstractions,
as demonstrated by Odersky~\cite{OCD+}.
This is one possible explanation why Typed Racket, which is not built on abstractions like Scala,
does not support bounded quantification. Unfortunately, no Typed Racket paper mentions 
bounded quantification, so the rationale is not clear.

Clojure, like Scala, is built on object-oriented abstractions. Clojure protocols
and Java interfaces (interfaces are supported by Clojure) are good candidates
for bounds in bounded or F-bounded polymorphism.

\section{Typed Racket}

Typed Racket is a statically typed sister language of Racket. It
attempts to preserve existing Racket idioms and aims type check
existing Racket code by simply adding top level type annotations~\cite{Tob10}.

Typed Racket fully expands all macro calls before type checking~\cite{Tob10},
avoiding the complex semantics of type checking macro definitions, an ongoing research area summarised
 by Herman~\cite{Her10}.
Typed Clojure follows a similar strategy; only the fully macro-expanded form
will be type checked. Type checking macro definitions are outside the scope of this project.

Along with a full static type system, Typed Racket 
also uses runtime contracts to enforce type invariants at runtime
at the interface with untyped code ~\cite{TF08}.
Utilising runtime contracts to aid type checking is outside the scope of this project, but would be 
considered desirable and accessible future work.

Two other Typed Racket features that will be explored are recursive types and refinement types  
~\cite{Tob10}. Recursive types allow a type definition to refer to itself, enabling structurally
recursive types like binary trees. Refinement types let the programmer define
new types that are subsets of existing types, such as the type for even integers, a subset of all integers.
Both these features would fit well in a future implementation of this project.

\section{Occurrence Typing}
\label{sec:OccurrenceTyping}

Dynamically typed languages use an ad-hoc combination of type predicates,
selectors, and conditionals to steer execution flow and reason about runtime types of variables.
Typed Racket uses occurrence typing to capture these ad-hoc type refinements.
For example, listing \ref{lst:occ1} shows occurrence typing following the implications 
of the type predicate \emph{number?} and the selector \emph{first}, and utilises those implications to refine
the type of \lstinline|x|. If the test at line 3 succeeds, occurrence typing refines the
type of \lstinline|(first x)| to be \lstinline|Number|, which allows \lstinline|(+ 1 (first x))|
to be well typed. Similarly at line 4, we can be sure that \lstinline|(first x)| is
a \lstinline|String|, since we have ruled out the case of being a \lstinline|Number|.

\begin{lstlisting}[caption=A well typed form utilising occurrence typing with Clojure syntax, label=lst:occ1]
(let [x (list (number-or-string))]
  (cond 
    (number? (first x)) (+ 1 (first x))
    :else               (str (first x))))
\end{lstlisting}

Occurrence typing~\cite{TF08,TF10} extends the type 
system with a \emph{proposition environment} that represents 
the information on the types of bindings down conditional branches.
These propositions are then used to update the types associated
with bindings in the \emph{type environment} down branches
so binding occurrences are given different types 
depending on the branches they appear in, and the conditionals
that lead to that branch.
(See section \ref{sec:occurenceimpl} for details on how Typed Clojure uses occurrence typing).

For occurrence typing to infer propositions from type predicate usages, it requires 
two extra annotations: a ``then'' proposition
when the result is a true value, and an ``else'' proposition for a false value.
For example, \lstinline|number?| has a ``then'' proposition that says its argument
is of type \lstinline|Number|, and an ``else'' proposition that says its argument is not of type \lstinline|Number|.

An exciting application of occurrence typing as yet unexplored is facilitating null-safe interoperability with Java.
By declaring \lstinline|nil| (Clojure's value of Java's \lstinline|null|) to \emph{not} be a subtype of reference types,
we can begin to statically disallow potentially inconsistent usages of \lstinline|nil| as part of the type system.

\begin{lstlisting}[caption=Observing nil-checks using occurrence typing, label=lst:nil]
(let [a (ObjectFactory/getObject)]
  (when a
    (expects-non-nil a)))
\end{lstlisting}

Listing \ref{lst:nil} infers from the Java signature \lstinline|Object getObject()| that
that \lstinline|a| is of type \lstinline|(U nil Object)|. This is equivalent to Java's
\lstinline|Object| static type, as \lstinline|null| is a member of all reference types. By surrounding
the call \lstinline|(expects-non-nil a)| with \lstinline|(when a ...)|, we guarantee that
\lstinline|a| is non-nil when passed to \lstinline|expects-non-nil|. Occurrence typing infers
this by observing \lstinline|nil| is a false value in Clojure, therefore \lstinline|a| cannot
be \lstinline|nil| the body of the \lstinline|when|, refining \lstinline|a|'s type to \lstinline|Object|
from \lstinline|(U nil Object)|.

Occurrence typing is a relatively simple technique used successfully 
in Typed Racket. Clojure is similar enough to Racket for occurrence typing to work
without issues, and has good potential to help programmers avoid using
\lstinline|nil| incorrectly

\section{Statically Typed Multimethods}

Clojure provides multimethods as a core language feature. This section discusses systems that statically
verify type safety for multimethods.

Millstein and Chambers~\cite{MS02}
describe Dubious, a simple statically typed core language including multimethods that
dispatch on the type of its arguments. They tackle a key challenge for statically typing
multimethods: ``it is possible for two modules containing arbitrary multimethods to typecheck
successfully in isolation but generate type errors when linked together.''~\cite{MS02}

After some investigation, typing multimethods with Typed Clojure is assigned as future work
(see section \ref{sec:mmfuture}).

\section{Higher Kinded Programming}

Many advanced type systems provide support for \emph{higher kinds},
which are ``another level up'' from types.
For example, a type \lstinline|Number| is distinguished from a
\emph{type constructor} \lstinline|Monad| (see section \ref{sec:monads}), which is a type level function
(similar to the difference between values and functions on the value level, but on the type level).

F$_\omega$ is a typed $\lambda$-calculus with support for higher kinds,
specifically type constructors (see Pierce~\cite{Pie02}).
Haskell~\cite{Mar10} distinguishes between types and type constructors,
the latter of which is an essential part of its monad library 
(see section \ref{sec:monads} for a monad library ported to Typed Clojure
that uses type constructors similarly).

\section{Conclusion}

Many related components must come together in the design of a
static type system. Typed Racket achieves a satisfying balance of 
occurrence typing, local type inference and variable-arity polymorphism.
Scala features F-bounded polymorphism, a class hierarchy
that is compatible with Java, and colored local type inference.
Typed Clojure takes inspiration from these, and similar, projects.

SML CIDRE \cite{Dav05} was also a secondary influence - particularly it's use of bidirectional 
checking of extrinsic types including subtyping, intersection types, and it's aim to capture program invariants.  
However, SML CIDRE starts from the intrinsically typed language SML which already has a rich static 
type system, which leads to many different considerations compared to starting from a dynamically typed language.  
Much of this influence was via a student-supervisor relationship, making it particularly hard to pin down the specifics.

\chapter{Design Choices}

Typed Clojure is designed to be of practical use to Clojure programmers.
Many of the design goals are similar to Tobin-Hochstadt's \cite{Tob10}
for Typed Racket.

\section{Preserve Idioms}


\section{Typed Racket}

The majority of the design and implementation of Typed Clojure is based on Typed Racket.

\subsection{Occurrence Typing}
% interesting paths
% - count path element
% - class path element
% - keyword path element

\subsection{Variable-arity Polymorphism}
% Problematic areas
% - flattened paired arguments. assoc, hash-maps, array-map
% - complex argument dependencies. comp, partial

\section{Safer Host Interoperability}
% Args non-nil, return nilable
% Constructors and other special methods handled appropriately (eg. Constructor are non-nil return)
% TODO refine language Clojure vs. Clojure JVM

% what is this paragraph saying?
Dialects of Clojure provide fast and unrestricted access to their host platform. 
This is a intentional source of incompatibility between dialects.
For example, Clojure is hosted on the Java Virtual Machine (JVM), ClojureCLR on the Common Language Runtime (CLR),
and ClojureScript on Javascript VMs. These are very different platforms, and it is unlikely
a Clojure dot call will be portable. There is no attempt at reconciling
host interoperability differences, and it is up to the programmer to decide
how to best abstract over different hosts.

Typed Clojure targets the JVM hosted Clojure.
Clojure embraces the JVM as a host by sharing its runtime type system and encouraging direct
interaction with libraries written in other JVM languages. Most commonly, Clojure programmers
reach to Java libraries. Java is statically typed language. It follows that any interaction with
Java will already have an annotated Java type. Typed Clojure can use these annotations
to statically type check these interactions.

Typed Clojure attempts to improve some shortcomings of Java's type system.

Java's static type system does not provide a type-safe construct for eliminating
the \lstinline|null| pointer. Instead, Java programmers often rely on either testing
for \lstinline|null| or prior domain knowledge.

\begin{lstlisting}[caption=null elimination in Java]
...
Object a = nextObject();
if (a != null)
  parseNextObject(a);
else
  throw new Exception("null Pointer Found!);
...
\end{lstlisting}



In Java, \lstinline|null| is a subtype to all reference
types, but \lstinline|null| is not expressible as a static type. In other words,
when a Java type claims a reference type, you should also assume it can also be of 
type \lstinline|null|.



Scala provides the Option type for safe \lstinline|null| elimination, forcing pattern matching
where \lstinline|null| might occur.

\begin{lstlisting}[caption=null elimination in Scala]
TODO
\end{lstlisting}

Clojure 

\subsection{Primitive Arrays}
% Primitive arrays are covariant, which is well-known to be statically unsound.
% We cannot trust the type signature of any Java method or field.
% Separate read and write types into two parameters.

\subsection{Interaction with null}
% null is explicitly expressible in my type system
% In Java, null is subtype of all reference types, any reference type must be tested to prevent NPE
% 

\subsection{Generic Java}
% Existential types are needed to fully express Generic Java types in Typed Clojure
% F-bounded polymorphism supported

\section{Multimethods}

% INTRODUCTION TO CLOJURE MULTIMETHODS

%A multimethod in Clojure differs from most other
%abstractions of the same name. Clojure multimethods
%dispatch on the result of an arbitrary function
%rather than the types or structure of its arguments.
%
%\begin{lstlisting}[caption=Clojure Multimethods, label=lst:mm1]
%(defmulti ToString class)
%
%(defmethod ToString java.lang.Integer
%  [a]
%  (str "Integer: " a))
%
%(defmethod ToString java.lang.Number
%  [a]
%  (str "Number: " a))
%\end{lstlisting}
%
%Even with this extra flexibility, multimethods are often used to dispatch on the type of its arguments.
%Listing \ref{lst:mm1} illustrates a common usage of multimethods: dispatching on the class of the first argument.
%We use \lstinline|defmulti| to define a new multimethod.
%The function \lstinline|class| is used as the dispatch function, a one-argument function returning the class of its argument
%as an instance of \lstinline|java.lang.Class|.
%
%A multimethod invocation redirects control flow to exactly one of its \emph{methods} with unresolved ambiguities resulting in
%a runtime error. The method is selected by first comparing the result of the dispatch function to
%the dispatch values defined by each method. If one or more methods match, multimethods provide
%the ability to prefer one method over another.

\section{Clojure Class Hierarchy}
\section{Protocols and Datatypes}
\section{Optional Type Checking}
\section{Local Type Inference}

Similar to Typed Racket, type inference is based on Local Type Inference
by Pierce and Turner.

\section{F-bounded Polymorphism}

Typed Clojure includes support for F-bounds on type variables, as an extension
to Local Type Inference. 

An extra environment from type variables to bounds is kept until after inference,
after which each unground bound is substituted with the inferred types and the
subtyping relationships are checked.

\section{Heterogeneous Maps}
% keyword keys for now, generally encouraged, fast
% Some heterogeneous maps have string args, eg. from web frameworks

% Operational semantics

\subsection{Operational Semantics}
 
%$$
%\begin{tdisplay}{Test1}
%\begin{altgrammar}
%  \s{},\t{} &::= & \int \alt \bool \alt {\proctype {\s{}} {\t{}}} 
%  &\mbox{Types}\\
%  \M{}, \N{}, \P{}, \dots &::=& \x{} \alt \abs{\x{}}{\M{}} \alt
%  \comb{\M1}{\M2}  &\mbox{Raw Terms} \\
%\end{altgrammar}
%\end{tdisplay} 
%$$

 
$$
\begin{tdisplay}{Syntax of Terms}
\begin{altgrammar}
  \exp{} &::=& (\c{} \overrightarrow{\exp{}})         % function application, multiple arguments
             \alt \{ \overrightarrow{\exp{}\ \exp{}} \} % map literal
             &\mbox{Expressions} \\ 
  \v{} &::=& \k{} \alt \nil{}
              \alt \{ \overrightarrow{\v{}\ \v{}} \}   % map literal
              &\mbox{Values} \\
  \c{} &::=& \assoc{} \alt \dissoc{} \alt \get{}
              &\mbox{Constants}
\end{altgrammar}
\end{tdisplay}
$$

$$
\begin{tdisplay}{Evaluation Contexts}
  \begin{altgrammar}
    \E{} &::=& [ ] % application rules
              \alt (\c{}\ \overrightarrow{\v{}}\ \E{}\ \overrightarrow{\exp{}}) % eval arguments left-to-right
              % map rules
              \alt \{\overrightarrow{\v{}\ \v{}}\ \E{}\ \exp{}\ \overrightarrow{\exp{}\ \exp{}} \} % key first
              \alt \{\overrightarrow{\v{}\ \v{}}\ \v{}\ \E{}\ \overrightarrow{\exp{}\ \exp{}} \}   % value next
              &\mbox{Evaluation Contexts}
  \end{altgrammar}
\end{tdisplay}
$$ 
 
% \mapsto for updating maps
$$
\begin{tdisplay}{Operational Semantics}
\begin{array}{c}
{\inferrule[E-Assoc]
  { {v_1 = \{ \overrightarrow{\v{}\ \v{}} \} } \\
    {v_4 = v_1\ with\ entry\ v_2\ to\ v_3} }
    {{(\assoc{}\ v_1\ v_2\ v_3) \hookrightarrow v_4} }} \\
\\
{\inferrule[E-Dissoc]
  { {v_1 = \{ \overrightarrow{\v{}\ \v{}} \} } \\
    {v_3 = v_1\ without\ entry\ indexed\ by\ v_2} }
  {(dissoc\ v_1\ v_2) \hookrightarrow v_3} } \\
\\
{\inferrule[E-GetMapExist]
  { {v_1 = \{ \overrightarrow{\v{}\ \v{}} \} } \\
    {v_2\ v_3\ in\ v_1} }
%    {v_1\ has\ entry\ with\ key\ v_2} \\
%    {v_3 = corresponding\ value\ in\ entry\ with\ key\ v_2\ in\ v_1}
  { {(get\ v_1\ v_2) \hookrightarrow v_3} }} \\
\\
{\inferrule[E-GetMapNotExist]
  { {v_1 = \{ \overrightarrow{\v{}\ \v{}} \} } \\
    {v_1\ has\ no\ entry\ with\ key\ v_2} \\
    {v_3 = \nil{}} }
  { {(get\ v_1\ v_2) \hookrightarrow v_3} }}
\end{array} 
\end{tdisplay} 
$$

$$
\begin{tdisplay}{Type Syntax}
\begin{altgrammar}
  \T{} &::=& \Top \alt \Bottom \alt \Nil \alt \k{} \alt \{ \overrightarrow{\k{}\ \T{}} \}
             \alt {\IPersistentMap {\T{}} {\T{}}} \alt ( \vee\ \overrightarrow{\T{}} )
             \alt ( \wedge\ \overrightarrow{\T{}} )
\end{altgrammar}
\end{tdisplay}
$$

$$
\begin{tdisplay}{Core Type Rules}
\begin{array}{c}
{\inferrule[T-AssocHMap]
  { {\Gamma \vdash {\hastype {{\exp{}}_{m}} {{\T{}}_{m}}} } \\
    {\Gamma \vdash {\hastype {{\exp{}}_{k}} {{\k{}}}} } \\
    {\Gamma \vdash {\hastype {{\exp{}}_{v}} {{\T{}}_{v}}} } \\
    {\vdash {\issubtype {{\T{}}_{m}} {\TopHMap} }} }
  { {\Gamma \vdash {\hastype {( \assoc{}\ {\exp{}}_{m}\ {\exp{}}_{k}\ {\exp{}}_{v})} 
        {(update\_hmap\ {{\T{}}_{m}}\ {{\k{}}}\ {{\T{}}_{v}})}}}}} \\
\\
{\inferrule[T-AssocPromote]
  { {\Gamma \vdash {\hastype {{\exp{}}_{m}} {{\T{}}_{m}}} } \\
    {\Gamma \vdash {\hastype {{\exp{}}_{k}} {{\T{}}_{k}}} } \\
    {\Gamma \vdash {\hastype {{\exp{}}_{v}} {{\T{}}_{v}}} } \\
    {\vdash {\issubtype {{\T{}}_{m}} {\TopIPersistentMap}}} \\
    {\T{} = (promote\_hmap\ {\T{}}_{m}\ {\T{}}_{k}\ {\T{}}_{v})} }
  { {\Gamma \vdash {\hastype {( \assoc{}\ {\exp{}}_{m}\ {\exp{}}_{k}\ {\exp{}}_{v})} {\T{}}}}}} \\
\end{array}
\end{tdisplay}
$$

$$
\begin{tdisplay}{Type Metafunctions}
\begingroup
\fontsize{10pt}{12pt}
\begin{altgrammar}
  (update\_hmap\ \{ \overrightarrow{{\T{}}_{k}\ {\T{}}_{v}} \}\ {\k{}}_1\ {\T{}}_1) = 
                 \{ \overrightarrow{{\T{}}_{k}\ {\T{}}_{v}}\ {\k{}}_1\ {\T{}}_1 \}\ \\
  (update\_hmap\ (\wedge\ \overrightarrow{{\T{}}})\ {\T{}}_{k}\ {\T{}}_{v}) = 
                               (\wedge\ \overrightarrow{(update\_hmap\ {\T{}}\ {\T{}}_{k}\ {\T{}}_{v})}) \\
  (update\_hmap\ (\vee\ \overrightarrow{{\T{}}})\ {\T{}}_{k}\ {\T{}}_{v}) = 
                               (\vee\ \overrightarrow{(update\_hmap\ {\T{}}\ {\T{}}_{k}\ {\T{}}_{v})})
  \\ \\
  (promote\_hmap\ \{ \overrightarrow{{\T{}}_{k}\ {\T{}}_{v}} \}\ {\T{}}_{kn}\ {\T{}}_{vn})
                                    = {\IPersistentMap {(\vee\ \overrightarrow{{\T{}}_{k}}\ {\T{}}_{kn})}
                                                       {(\vee\ \overrightarrow{{\T{}}_{v}}\ {\T{}}_{vn})}} \\
  (promote\_hmap\ {\IPersistentMap {{\T{}}_{k}} {{\T{}}_{v}}} \ {\T{}}_{kn}\ {\T{}}_{vn})
                                    = {\IPersistentMap {(\vee\ {\T{}}_{k}\ {\T{}}_{kn})}
                                                       {(\vee\ {\T{}}_{v}\ {\T{}}_{vn})}}
\end{altgrammar}
\endgroup
\end{tdisplay}
$$

$$
\begin{tdisplay}{Subtyping}
\begin{array}{c}
\inferrule[S-Refl] { } {\vdash {\issubtype {\T{}} {\T{}}}} \\
\inferrule[S-Any] { } {\vdash {\issubtype {\T{}} {\Top}}}
\inferrule[S-UnionSuper] 
  {\ } 
  {\vdash {\issubtype {\T{}} {\Top}}}
\end{array}
\end{tdisplay}
$$

%\subsection{Higher-order Variable-arity Polymorphism (SKETCH)}
%
%Practical Variable-arity Polymorphism introduces the concept of a dotted type parameter,
%representing a sequence of types, allowing non-uniform variable parameters.
%This enables \lstinline|map| and other function with complex arguments to be typed.
%Strickland, Tobin-Hochstadt, and Felleisen's approach is solves almost all variable-arity applications in
%Typed Racket, and is a key feature of Typed Racket's implementation.
%
%Idiomatic variable-arity functions in Clojure's are even more sophisticated than Typed Racket, requiring extensions
%to Typed Racket's approach to type check satisfactorily. The Clojure core library favours functions with
%variable parameters, especially in a higher-order context. For example, \lstinline|assoc|
%associates key-value pairs to a target map. It takes one or more key-value pairs and
%applies them left-to-right to the target map. The key problem from a typing perspective
%is \lstinline|assoc|'s arguments are flattened, resulting in \lstinline|assoc| only
%accepting an even number of rest arguments. \lstinline|assoc| is also commonly used as a
%higher-order function.
%\lstinline|assoc| cannot be expressed in Typed Racket. For instance,
%dotted parameters represent an unrestricted number of types, where we require an even number.
%
%The nature of Clojure's idiomatic variable-arity functions suggest we must introduce
%new kinds of dotted parameters. Three main issues must be considered in the context of Typed Clojure.
%Firstly, subtyping between different kinds of dotted and rest
%\footnote{A rest parameter consumes zero or more types, rather than being a sequence of types.}
%parameters can be tricky, only made trickier by adding new kinds of dotted parameters.
%Secondly, elimination rules for dotted parameters in many frequently used Clojure core functions
%difficult to express.
%Thirdly, higher-order functions taking functions with dotted arguments commonly 
%The first issue requires careful design and implementation.
%The second and third points require new constructs, in the form of \emph{projections}
%and abstractions over \emph{dotted} parameters, respectively.





% A syntactic approach to type soundness (Wright and Felleisen)
% operational semantics
% - See Sam TH's operational semantics
% TAPL 392

% potential proofs/rules
% - hmaps
% - intersection types (protocols + types)
% - subtyping
% - dot methods
% - f-bounded polymorphism

\chapter{Implementation}

This chapter discusses the implementation of the Typed Clojure prototype type 
system (we refer to this implementation as \emph{Typed Clojure} for the remainder of this chapter)
in some detail concentrating on significant challenges that were identified and overcome.
Many aspects of Typed Clojure's design follow the implementation of Typed Racket,
which is reflected in this chapter. 
(The code is available on Github at \url{https://github.com/frenchy64/typed-clojure}).

\section{Type Checking Procedure}

Typed Clojure and Typed Racket differ significantly in how type checking
is integrated into their programming environments.
Typed Racket is implemented as a language on the
Racket platform, which provides highly sophisticated and extensible macro facilities.
Interestingly, this allows Typed Racket to be entirely implemented with macros.
Instead, Typed Clojure is implemented as a library that utilises abstract syntax trees (AST)
generated by analyze~\cite{Analyze2012}, a library I developed for this project.
This strategy follows common practice for Clojure projects, which favours providing modular libraries 
over modifying the language.

This section goes into details on the implementation of Typed Clojure.
First, a high-level overview is given on the type checking procedure.
Then the interfaces to particular high-level functions are discussed.

\subsection{General Overview}

There are several stages to type checking in Typed Clojure.
Type checking is typically initiated at the read-eval-print-loop prompt (REPL),
for example \lstinline|(check-ns 'my.ns)| checks the namespace \lstinline|my.ns|.
Before type checking begins, all global type definitions in the namespace
are added to the global type environment by compiling the namespace.
These type definitions include

\begin{itemize}
  \item type alias definitions
  \item global variable, protocol, datatype, and Java Class annotations
  \item Java method annotations, such as \lstinline|nilable-param| and \lstinline|non-nil-return|
\end{itemize}

An AST is then generated from the code contained in the target namespace.
This AST is then recursively descended and is type checked using local type inference.
Currently only one error is reported at a time, and type checking stops if a type error
is found.

\subsection{Bidirectional Checking}

The interface to the bidirectional checking algorithm is \lstinline|typed.core/check|,
which takes an expression, represented as an AST generated from \emph{analyze}, and an optional expected type for
the given expression. If the expected type is present, the bidirectional algorithm \emph{checks}
that the expected type matches the actual type of the expression.
If the expected type is omitted, a type is instead \emph{synthesised} for the expression.
The algorithm is based on Pierce and Turner's  Local Type Inference~\cite{PT00}
and the implementation is similar in form to Typed Racket's~\cite{TF08}, in that
one function with an extra ``expected type'' argument is preferred over two complementary
functions, one for checking and one for synthesis.

The main difference between Typed Clojure's and Typed Racket's bidirectional checking
algorithm is the representation for expressions. Typed Racket relies on pre-existing
Racket features like syntax objects for expression representation. Clojure instead
leans towards abstract syntax tree representation, despite its Lisp heritage.
In terms of the bidirectional checking, the difference is mostly cosmetic.


% go into more detail
% - update env
% - what is an environment? Bindings + propositions
% - LOC
% - link to github
% - how occurrence typing works
%   - update env
%   - how to calculate reachability 

\subsection{Occurrence typing}
\label{sec:occurenceimpl}

Typed Clojure's implementation of occurrence typing is ported and extended from Typed Racket.
Occurrence typing plays several roles in Typed Clojure.
First, occurrence typing is used to update the type environment at every conditional branch.
Second, it is used to calculate whether branches are reachable.
Third, paths are used extensively in Typed Clojure. 
The implementation of these features are discussed in this section.

The basic idea of occurrence typing involves keeping a separate environment of \emph{propositions}
that relate bindings to types. These propositions are then used to update the type environment.
In Typed Clojure there are several types of propositions, referred to as \emph{filters}.
They are based on the theory by Tobin-Hochstadt and Felleisen~\cite{TF10}, and are ported directly
from Typed Racket.


\begin{itemize}
  \item TopFilter and BotFilter represent the trivially true and trivially false propositions respectively.
  \item TypeFilter and NotTypeFilter represent a positive or negative association of a binding name
        to a type. The syntax \lstinline|(is t name)| means a proposition that the binding called
        \lstinline|name| is of type \lstinline|t| (corresponding to TypeFilter).
        The syntax \lstinline|(! t name)| means a proposition that the binding called \lstinline|name|
        is \emph{not} of type \lstinline|t| (corresponding to NotTypeFilter).
  \item AndFilter and OrFilter represent logical conjunction of propositions 
    (written \lstinline[mathescape]|(& $\overrightarrow{p}$)|), and
    logical disjunction of propositions
    (written \lstinline[mathescape]{(| $\overrightarrow{p}$)}) for propositions \lstinline|p|.
\end{itemize}

Propositions can optionally carry \emph{path} information
represented by a sequence of \emph{path elements}, which signify which part
of the binding's type to update. For example, Typed Racket uses \emph{car} and \emph{cdr} path elements
to track which component of a cons type to update.
Paths are particularly useful in Typed Clojure. There are path elements for traversing heterogeneous map types (KeyPE),
inferring length information (CountPE), and \lstinline|first| and \lstinline|rest| paths for sequences.
These additions do not appear to introduce any major new complexities related to paths.

\section{Polymorphic Type Inference}

The polymorphic type inference algorithm is directly ported from Typed Racket with slight extensions
for bounded variables,
and is directly based on Pierce and Turner's Local Type Inference~\cite{PT00}.
A common entry point for inferring type variables for polymorphic function invocations
is \lstinline|typed.core/infer|.

\lstinline|infer| is invoked like \lstinline|(infer X Y S T R expected)|, where

\begin{itemize}
  \item \lstinline|X| is a map from type variable names
        to their bounds (representing the type variables in scope),
  \item \lstinline|Y| is a map from type variable names
        to their bounds (representing the \emph{dotted} type variables in scope),
  \item \lstinline|S| and \lstinline|T| are sequences of types of equal length,
        (usually the types of the actual arguments provided and the types of the parameters
        of the polymorphic function),
  \item \lstinline|R| is a result type (usually the return type of the polymorphic function),
  \item \lstinline|expected| is the expected type for R, or the value \lstinline|nil|,
\end{itemize}

and returns a \emph{substitution} that satisfies the following conditions:

\begin{itemize}
  \item Pairwise, each \lstinline|S| is a below \lstinline|T|,
  \item \lstinline|R| is below \lstinline|expected|, if \lstinline|expected| is provided.
\end{itemize}

A substitution maps type variables to types.
It is valid to replace all occurrences of the type variables named in the substitution 
with their associated type.
For example, substitution are often applied to \lstinline|R| by the caller of \lstinline|infer|
to eliminate the type variables in \lstinline|X| and \lstinline|Y|.

\lstinline|infer| is almost always used when invoking polymorphic functions like,
for example, \lstinline|constantly|, which has type \lstinline|(All [x y] [x -> [y * -> x]])|
(read as a function taking \lstinline|x| and returning a function that takes any number
of \lstinline|y|'s and returns \lstinline|x|).
For instance, \lstinline|((constantly true) 'any 'number)|, results in the value \lstinline|true|.

Type checking the invocation \lstinline|(constantly true)|
calls \lstinline|infer| roughly like

\begin{lstlisting}
(infer {'x no-bounds 'y no-bounds}
       {}
       [(parse-type 'true)]
       [(make-F 'x)]
       (with-frees [(make-F 'x) (make-F 'y)]
         (parse-type '[y * -> x]))
       nil)
\end{lstlisting}

where the internal Typed Clojure bindings

\begin{itemize}
  \item \lstinline|no-bounds| is the type variable bounds with upper bound as \lstinline|Any|
    and lower bound as \lstinline|Nothing|,
  \item \lstinline|parse-type| is a function that takes type \emph{syntax} and converts it to a type (its argument must be quoted), 
  \item \lstinline|make-F| is a function that takes a name symbol and returns a type variable type of that name, and
  \item \lstinline|with-frees| is a macro that brings the type variables named in its first argument into scope in its second argument.
\end{itemize}

This returns a substitution that replaces occurrences of \lstinline|x| with \lstinline|true|
and \lstinline|y| with \lstinline|Any|.
This helps infer a result type of \lstinline|(constantly true)|
as \lstinline|[Any * -> true]|.

\section{F-Bounded Polymorphism}

A feature not present in Typed Racket is bounded polymorphism.
Several changes were needed to support bounded polymorphism. In every position
where a set of type variables was required, it was replaced by a map
of type variables to bounds.

Bounds consist of an upper and lower type bound, or a \emph{kind bound}.
Kind bounds are experimental following the inclusion of user definable type constructors
(motivated in section \ref{sec:monads}).
They are a stub for a more comprehensive treatment of higher-kinded operators
such as that described by Moors, Piessens, and Odersky for Scala~\cite{MPO08}.
At present, a type variable can only be instantiated to a type between
its upper and lower bounds, or, if a kind bound is defined instead, 
to a kind below the kind bound.

F-bounded polymorphism allows type variables to refer to themselves in their type bounds.
Bounds are checked after a substitution is generated, guaranteeing no substitution
can violate type variable bounds. To support F-bounds, the substitution being checked is applied to
the lower and upper bounds for each type variable, and the type associated with the type variable 
in the substitution is checked to be between these bounds.

\section{Variable-arity Polymorphism}
\label{sec:impvariablearity}

Variable-arity polymorphism in Typed Clojure is directly ported from Typed Racket.
This was the most complicated part of the prototype. At the center of the implementation
is manipulating dotted type variables, which can represent a sequence of types.

It also required changes to the polymorphic type inference, where each
reference to a type variable required a special case for a dotted type variable.
For example, the constraint-generation algorithm for Local Type Inference
features extra kinds of constraints for dotted variables.
Strickland, Tobin-Hochstadt, and Felleisen elaborate on the particular changes
required for the Typed Racket implementation~\cite{STF09}.

Porting Typed Racket's variable-arity polymorphism implementation was
tedious because some of the relevant internal functions interact
in strange ways with the rest of Typed Racket. My reaction was that Typed Racket
was initially designed without variable-arity polymorphism and was added
without major changes to other components. Typed Clojure was
developed with the same design so full variable-arity polymorphism
implementation could be ported without change.

\section{Portability to other Clojure Dialects}
\label{sec:portability}

Typed Clojure was built for the Clojure programming language, whose compiler
and data structures are implemented in Java. ClojureScript is the first major Clojure dialect
to be written in Clojure, and it is likely future dialects of Clojure will follow this example. 
Where Clojure uses Java Classes and Interfaces, 
ClojureScript's compiler and data structures are written in terms of Clojure's
two core abstractions: protocols and datatypes.
It would be desirable to port Typed Clojure to such Clojure dialects while keeping
the core of the implementation constant.

A significant portion of Typed Clojure is theoretically platform independent but there are
challenges to targeting new dialects, including incompatible host interoperability and
non-standardised abstract syntax trees.
The first issue is predictable due to a core philosophy of Clojure dialects: 
host interoperability is non-portable\footnote{See the complete Clojure rationale: http://clojure.org/rationale}.
Each dialect of Clojure has a unique host interoperability story and Typed Clojure should cater for them
separately.
The second issue is potentially resolvable either by enforcing a standard representation for 
abstract syntax trees across Clojure implementations, or developing a library that provided
a common interface to abstract syntax trees for each Clojure implementation.

\section{Proposition Inference for Functions}
\label{sec:filterneg}

A feature not yet implemented in Typed Racket is the ability
to infer new propositions based on existing propositions of a function.
This feature was added to Typed Clojure to support filtering
a sequence based on negative information, such as filtering values
that are \emph{not} \lstinline|nil|.

\begin{lstlisting}[caption=Type annotation for filter, label=lst:filterann]
(ann clojure.core/filter 
  (All [x y]
    [[x -> Any :filters {:then (is y 0)}] (U nil (Seqable x)) -> (Seqable y)]))
\end{lstlisting}

To better understand the problem, listing \ref{lst:filterann} presents the type of \lstinline|filter|,
which takes a function \lstinline|f| and a sequence \lstinline|s| as arguments, and returns a sequence that 
contains each element in \lstinline|s| such that applying \lstinline|f| to the element returns a true value.
The \lstinline|:filters| syntax requires some explanation. 
Function types support an optional \emph{filter set} attached to its return type, written as a map
with \lstinline|:then| and/or \lstinline|:else| keys (if omitted, they default to the trivially true proposition which has no effect). 
The ``then-proposition'' and ``else-proposition'' are added
to the type environment when the return value is a true and false value respectively.
For example, the filter set \lstinline|{:then (is y 0)}| is read ``if the return value is a true value, then the first argument
must be of type y, otherwise if it is a false value, nothing interesting is enforced''
The type given for \lstinline|filter| works because the type variable
\lstinline|y| occurs in both the ``then-proposition'' of the first argument and the return type 
\lstinline|(Seqable y)|.

\begin{lstlisting}[caption=Troublesome filter, label=lst:filtertrouble]
(filter (fn [a] (not (nil? a))) coll)|
\end{lstlisting}

The difficulty starts with something like listing \ref{lst:filtertrouble},
where the inferred filter set for the first argument to \lstinline|filter|
is \lstinline|{:then (! nil 0) :else (is nil 0)}|\footnote{(! nil 0) is the proposition that the first argument is \emph{not}
of type \lstinline|nil|.}. The ``then-proposition'' \lstinline|(! nil 0)| does not fit with
\lstinline|(is y 0)| that \lstinline|filter| expects.

We can sometimes get around this if we already have a predicate with a positive
``then-proposition''. For example, if we are filtering out \lstinline|nil| values
from a sequence of type \lstinline|(Seqable (U Number nil))|, we can replace
listing \ref{lst:filtertrouble} with \lstinline|(filter number? coll)|,
where \lstinline|number?| has the filter set \lstinline|{:then (is Number 0) :else (! Number 0)}|.
This does not work, however, when filtering a sequence of type like \lstinline|(Seqable (U x nil))|
for some unspecified \lstinline|x| because there is no built-in predicate with ``then-proposition''
\lstinline|(is x 0)|.

\begin{lstlisting}[caption=Filtering with negative propositions, label=lst:filtergood]
  (filter (ann-form 
            #(not (nil? %))
            [(U nil x) -> boolean :filters {:then (is x 0)}])
          mvs)
\end{lstlisting}

Instead, we generate new propositions using a technique 
suggested by Tobin-Hochstadt~\cite{Tob12},
that follows from the occurrence typing calculus defined 
by Tobin-Hochstadt and Felleisen~\cite{TF10}.
First the filter set for functions are inferred as usual.
To collect new propositions, the ``then-proposition'' is applied to the type environment (which maps
local bindings to types). Any types associated with bindings that are changed after this
are represented as new propositions, which are added to the ``then-proposition'' for this filter set.
The same procedure is followed for the ``else-proposition''.

Using this technique, the anonymous function in listing 
\ref{lst:filtergood}\footnote{The Clojure syntax \lstinline|#(not (nil? \%))| 
is equivalent to \lstinline|(fn [a] (not (nil? a)))|}
has the filter set \lstinline|{:then (& (! nil 0) (is x 0)) :else (is nil 0)}|
which is good enough to infer the \lstinline|filter|ed result as \lstinline|(Seqable x)|.

Further work in this area is needed when filtering on a non-anonymous function.
For example, it is not clear how to infer the common idiom \lstinline|(filter identity coll)|
as returning a sequence of non-nil elements, for any sequence \lstinline|coll|.
Inferring new propositions for already existing functions like \lstinline|identity|
does not fit with Tobin-Hochstadt and Felleisen's calculus~\cite{TF10},
confirmed by Tobin-Hochstadt~\cite{Tob12} as future work in this area.

%\section{Non-hygienic Macro Expansion and Filters}


% need to remove bindings as they fall out of scope
% Need hygiene to ensure correct propagation of filters
%(fn [a]
%  (if (= a 1)
%    (let [a 'foo] ; here this shadows the argument, impossible to recover filters
%      a)          ; in fact any new filters about a will be incorrectly assumed to be the argument
%      false))


\chapter{Experiments}

In this chapter we look at several example of using Typed Clojure and
how well the current prototype handles them. 
We intentially chose examples that could be challenging to type check
or were particularly useful to the everyday Clojure programmer.

\section{Java Interoperability}
\label{sec:experimentinterop}

As Typed Clojure is intended to be useful for practical purposes, it was important
to port existing code that utilized Java interoperability.
My porting effort attempted to follow how a real programmer might port code to Typed Clojure
and I describe my approach for each situation.

I ported a function from clojure.contrib.reflect\footnote{https://github.com/cemerick/pomegranate}, 
a Clojure library relying heavily on Java interoperability.
I chose \lstinline|call-method| for several reasons: it chains several Java calls together,
it uses primitive arrays, it exposes a wart of Clojure with \emph{Named} things,
and \lstinline|null| is a valid value in one place.

Before showing the implementation, there is a brief explanation of the relevant syntax.
Java methods are called using the \emph{dot} operator. If \lstinline|o| is an object \lstinline|(. o m a*)|
calls its method named \lstinline|m| passing \lstinline|a*| arguments. The method can be named first
using the equivalent sugar \lstinline|(.m o a*)|. Also, \lstinline|doto| is convenient 
notation for muliple method calls on the same object, presumably for side effects, and returns the original
object. For example, \lstinline|(doto o (.m1 a*) (m2 a*))| calls methods \lstinline|m1| and \lstinline|m2|
on \lstinline|o| and returns \lstinline|o|.

\begin{lstlisting}[caption=call-method, label=lst:callmethod]
;; call-method pulled from clojure.contrib.reflect, (c) 2010 Stuart Halloway & Contributors
(defn call-method
  "Calls a private or protected method.

  params is a vector of classes which correspond to the arguments to
  the method e

  obj is nil for static methods, the instance object otherwise.

  The method-name is given a symbol or a keyword (something Named)."
  [^Class klass method-name params obj & args]
  (let [method (doto (.getDeclaredMethod klass 
                                         (name method-name)
                                         (into-array Class params))
                 (.setAccessible true))]
    (.invoke method obj (into-array Object args))))
\end{lstlisting}

The original function is modified slightly for readability and is presented in listing \ref{lst:callmethod}.

\begin{lstlisting}[caption=call-method Type Annotation, label=lst:callmethodtype]
(ann call-method 
     [Class Named (IPersistentVector Class) (U nil Object) (U nil Object) * -> (U nil Object)])
\end{lstlisting}

Thankfully this function has up-to-date documentation, and from it we can derive an expected type
(listing \ref{lst:callmethodtype}).

Before running the type checker we must convert our array constructors into ones that Typed Clojure
can understand. Array types in Typed Clojure are represented by \lstinline|(Array c t)|, where
the Java class \lstinline|c| is the Java component type,
and the Typed Clojure type \lstinline|t| is the Typed Clojure component type. 
We can pass this array to Java methods accepting
type \lstinline|c[]|, and we can read and write type \lstinline|t| to the array from Typed Clojure.

\begin{itemize}
\item \lstinline|(into-array Class params)| becomes \lstinline|(into-array> Class Class params)|
      which is of type \lstinline|(Array Class Class)|.
\item \lstinline|(into-array Object args)| becomes \lstinline|(into-array> Object (U nil Object) args)|
      which is of type \lstinline|(Array Object (U nil Object))|.
\end{itemize}

The second place in the \lstinline|Array| type constructor allows fine grained control over what is allowed
in the array. The first point above must be type \lstinline|(Array Class Class)| because the 
\lstinline|getDeclaredMethod| method on \lstinline|java.lang.Class| instances requires an array of non-null
\lstinline|Class| objects. On the other hand, the \lstinline|invoke| method takes an array that allows
\emph{null} members, so its type is \lstinline|(Array Object (U nil Object))|.

Now we run the type checker, which produces a type error.

\begin{lstlisting}
#<Exception java.lang.Exception: 29: Cannot call instance method java.lang.reflect.AccessibleObject/setAccessible on type (U nil java.lang.reflect.Method)>
\end{lstlisting}

Because Typed Clojure assumes all methods return nilable Objects, the call to \lstinline|getDeclaredMethod|
has return type \lstinline|(U nil java.lang.reflect.Method)|. It is not type safe to call \lstinline|setAccessible|
on this type, so we get a type error.

In this case, Typed Clojure is too conservative: according to its documentation \lstinline|getDeclaredMethod|
never returns \emph{null}. We add this rule with \lstinline|non-nil-return|.

\begin{lstlisting}
(non-nil-return java.lang.Class/getDeclaredMethod :all)
\end{lstlisting}

Running the type checker produces a different type error.

\begin{lstlisting}
#<Exception java.lang.Exception: Type Error, REPL:32 - (U java.lang.Object nil) is not a subtype of: java.lang.Object>
\end{lstlisting}

This concerns passing \lstinline|obj| as the first argument to the \lstinline|invoke| method.
Typed Clojure conservatively defaults method parameter types as non-nullable. 
Therefore the first parameter of \lstinline|invoke| is \lstinline|Object| by default; \lstinline|obj|
is \lstinline|(U java.lang.Object nil)|. Again, this is too conservative as the first argument can
be \emph{null} for static methods, and we use \lstinline|nilable-param| to specify the first argument
of \lstinline|invoke| may be nil, for the arity of two parameters.

\begin{lstlisting}
(nilable-param java.lang.reflect.Method/invoke {2 #{0}})
\end{lstlisting}

The final successfully type checked code is presented in listing \ref{lst:callmethodfinal}.

\begin{lstlisting}[caption=Type Annotated code for call-method, label=lst:callmethodfinal]
(non-nil-return java.lang.Class/getDeclaredMethod :all)
(nilable-param java.lang.reflect.Method/invoke {2 #{0}})

(ann call-method [Class Named (IPersistentVector Class) (U nil Object) (U nil Object) * -> (U nil Object)])

;; call-method pulled from clojure.contrib.reflect, (c) 2010 Stuart Halloway & Contributors
(defn call-method
  "Calls a private or protected method.

  params is a vector of classes which correspond to the arguments to
  the method e

  obj is nil for static methods, the instance object otherwise.

  The method-name is given a symbol or a keyword (something Named)."
  [^Class klass method-name params obj & args]
  (let [method (doto (.getDeclaredMethod klass 
                                         (name method-name)
                                         (into-array> Class Class params))
                 (.setAccessible true))]
    (.invoke method obj (into-array> Object (U nil Object) args))))
\end{lstlisting}

\section{Red-black trees}

This experiment involved porting an implementation of red-black trees
used as an experiment for SML CIDRE by Davies~\cite{Dav05}.
SML CIDRE is a sort-checker for Standard ML that supports refinement
types.
The red-black tree implementation was ported to work with SML CIDRE
to statically check red-black tree invariants.
It is a particularly good experiment for SML CIDRE because it generates an unusually large
amount of intersection types.

Intersection types in SML CIDRE are heavily optimised via memoisation.
Because of this, SML CIDRE is able to check this experiment with little trouble.
The current Typed Clojure prototype is not as successful, and appears to hang
during type checking.
The reasons are not entirely clear, but it is likely the number of intersection types
are too large for the current prototype to handle, which has no optimisations
around intersection types.

This experiment was intended to stress test Typed Clojure to an extreme, and
it is unsurprising that Typed Clojure was not up to the task.
One interesting point was noticed when porting it.
The SML CIDRE version of the red-black tree experiment
uses SML CIDRE datasorts to represent the red-black tree invariants.
The Typed Clojure version entirely uses plain hash-maps, a common Clojure idiom.
It was noticed that the heterogeneous map types that Typed Clojure provides would be sufficient to
represent the red-black tree invariants.

\section{Monads}
\label{sec:monads}

Monads are an interesting control structure used in functional programming languages,
as described by Wadler~\cite{Wad95}.
They are most recognisable from its inclusion in the statically-typed language Haskell~\cite{Mar10},
where monads are relied on for many features including global state, file output, and exceptions.

I chose to port the Clojure Contrib library \emph{algo.monads}~\cite{Hic12} to Typed Clojure. The library
provides several kinds of monads, monad transformers, and monadic functions.
Macros are used to provide pleasant syntax for consumers of the library.

\subsection{Monad Definitions}
\label{sec:monaddef}

This library represents a monad as a hash-map with four keys: \lstinline|:m-bind|, \lstinline|:m-result|,
\lstinline|:m-zero|, and \lstinline|:m-plus|. A valid monad must provide the first two, and the latter
two may optionally be mapped to the keyword \lstinline|::undefined|\footnote{Keywords prefixed with \lstinline|::|
are qualified in the current namespace.}.

\begin{lstlisting}[caption=Untyped definition for the identity monad, label=lst:identitymdef]
(defmonad identity-m
   "Monad describing plain computations. This monad does in fact nothing
    at all. It is useful for testing, for combination with monad
    transformers, and for code that is parameterized with a monad."
  [m-result identity
   m-bind   (fn m-result-id [mv f]
              (f mv))
  ])
\end{lstlisting}

A monad is defined using the macro \lstinline|defmonad|. The macro expands to code that binds a var to a hash-map
with the aforementioned keys.

\begin{lstlisting}[caption=Type for identity monad, label=lst:identitymtype]
(def-alias Undefined '::undefined)
(ann identity-m
     '{:m-bind (All [x y]
                 [x [x -> y] -> y])
       :m-result (All [x]
                   [x -> x])
       :m-zero Undefined
       :m-plus Undefined})
\end{lstlisting}

A direct typing of the identity monad would look like listing \ref{lst:identitymtype}.
This signature assigns the expected type for the var \lstinline|identity-m| 
to be a heterogeneous map type with the minimum entries for monads: \lstinline|:m-bind|
and \lstinline|:m-result|. The syntax for heterogeneous map types requires a \lstinline|'| prefix,
and any number of key-value pairs are given between curly braces. Heterogeneous map keys must be keywords,
so the syntax is made more convenient by omitting the usually required \lstinline|'| prefix
for keyword types.

Monad types in languages with advanced type systems are often abstracted using type constructors.
This allows reasoning about monadic code while keeping the particular monad abstract, which is a desirable result,
so Typed Clojure was extended to support user definable type constructors.

\begin{lstlisting}[caption=An abstract definition of a monad., label=lst:monadalias]
(def-alias Monad 
  (TFn [[m :kind (TFn [[x :variance :covariant]] Any)]]
    '{:m-bind (All [x y]
                [(m x) [x -> (m y)] -> (m y)])
      :m-result (All [x]
                  [x -> (m x)])
      :m-zero Undefined
      :m-plus Undefined}))
\end{lstlisting}

Listing \ref{lst:monadalias} captures the abstract definition of a monad.
A \lstinline|Monad| is a type constructor parameterised by \lstinline|m|, which is
a type constructor taking a single argument \lstinline|x| (a type) and returning a type 
(written \lstinline|Any|)\footnote{Haskell-like syntax is helpful here, the kind \lstinline|(TFn [[x :variance :covariant]] Any)|
is approximately \lstinline|* -> *|. A more streamlined syntax is planned for future work.}.
The body of the type constructor uses \lstinline|m| abstractly. Typed Clojure
ensures the correct number of arguments are passed and recognises the declared variances
for each parameter, In this case the first argument of \lstinline|m| is declared
a covariant position. When instantiated, \lstinline|m| must also be a type operator
of one covariant argument returning a type.

\begin{lstlisting}[caption=Identity monad using user defined type constructors, label=lst:identityctors]
(ann identity-m (Monad (TFn [[x :variance :covariant]] x)))
\end{lstlisting}

We can now express the type of the identity monad more abstractly (listing \ref{lst:identityctors}).
The purpose of the monad seems more apparent just from reading its type.
In this case, the fact that identity monad has no effect is reflected
by its type constructor returning exactly its argument.

\begin{lstlisting}[caption=Type checked identity monad definition, label=lst:identitydeftyped]
(defmonad identity-m
   "Monad describing plain computations. This monad does in fact nothing
    at all. It is useful for testing, for combination with monad
    transformers, and for code that is parameterized with a monad."
  [m-result identity
   m-bind   (ann-form
              (fn m-result-id [mv f]
                (f mv))
              (All [x y]
                [x [x -> y] -> y]))
  ])
\end{lstlisting}

The final type checked definition for the identity monad is given in listing \ref{lst:identitydeftyped}.
In this case, just the monadic bind required annotation.

\begin{lstlisting}[caption=Several monad types, label=lst:severalmonads]
; Maybe monad
(ann maybe-m (MonadPlusZero
               (TFn [[x :variance :covariant]] 
                 (U nil x))))

; Sequence monad (called "list monad" in Haskell)
(ann sequence-m (MonadPlusZero 
                  (TFn [[x :variance :covariant]]
                    (Seqable x))))

; State monad
(def-alias State
  (TFn [[r :variance :covariant]
        [s :variance :invariant]]
    [s -> '[r s]]))
(ann state-m (All [s]
               (Monad (TFn [[x :variance :covariant]]
                        (State x s)))))
\end{lstlisting}

Listing \ref{lst:severalmonads} shows several monad types. The alias \lstinline|MonadPlusZero|
is identical to \lstinline|Monad|, except both \lstinline|:m-zero| and \lstinline|:m-plus| are defined.
The definitions were type checked by adding type annotations in appropriate places, in a similar fashion
to the identity monad. A significant obstacle was discovered in while type checking monad definitions,
relating to filtering sequences with negative type information, discussed in section \ref{sec:filterneg}.

\subsection{Monad Transformer Definitions}

The initial motivation for adding user defined type constructors to Typed Clojure
(as discussed in section \ref{sec:monaddef}) was to type check
monad transformer definitions.
Monad transformers are always parameterised by a monad type constructor, and should work for all
monad type constructors. Keeping the monad type constructor abstract allows us to determine
whether a monad transformer works \emph{for any} monads.

Listing \ref{lst:maybet} shows the full code for type checking the definition
of the maybe monad transformer.
Typing this example was a matter of assigning types to anonymous functions
(similar to standard monad definitions) and manually instantiating 
polymorphic types with \lstinline|inst|
that parameterise over higher kinds, which cannot be inferred by Typed Clojure.

\begin{lstlisting}[caption={Maybe monad transformer in Typed Clojure}, label=lst:maybet]
(ann maybe-t
     (All [[m :kind (TFn [[x :variance :covariant]] Any)]]
       (Fn 
         [(AnyMonad m) -> (MonadPlusZero (TFn [[y :variance :covariant]]
                                           (m (U nil y))))]
         [(AnyMonad m) nil -> (MonadPlusZero (TFn [[y :variance :covariant]]
                                               (m (U nil y))))]
         [(AnyMonad m) nil (U ':m-plus-default ':m-plus-from-base)
          -> (MonadPlusZero (TFn [[y :variance :covariant]]
                              (m (U nil y))))])))
(defn maybe-t
  "Monad transformer that transforms a monad m into a monad in which
   the base values can be invalid (represented by nothing, which defaults
   to nil). The third argument chooses if m-zero and m-plus are inherited
   from the base monad (use :m-plus-from-base) or adopt maybe-like
   behaviour (use :m-plus-from-transformer). The default is :m-plus-from-base
   if the base monad m has a definition for m-plus, and
   :m-plus-from-transformer otherwise."
  ([m] ((inst maybe-t m) m nil :m-plus-default))
  ([m nothing] ((inst maybe-t m) m nothing :m-plus-default))
  ([m nothing which-m-plus]
   (monad-transformer m which-m-plus
     [m-result (with-monad m m-result)
      m-bind   (with-monad m
                 (ann-form
                   (fn m-bind-maybe-t [mv f]
                     (m-bind
                       mv
                       (ann-form
                         (fn [x]
                           (if (nil? x)
                             (m-result nothing) 
                             (f x)))
                         [(U nil a) -> (m (U nil b))])))
                   (All [a b]
                     [(m (U nil a)) [a -> (m (U nil b))] -> (m (U nil b))])))
      m-zero   (with-monad m (m-result nothing))
      m-plus   (with-monad m
                 (ann-form
                   (fn m-plus-maybe-t [& mvs]
                     (if (empty? mvs)
                       (m-result nothing)
                       (m-bind (first mvs)
                               (ann-form
                                 (fn [v]
                                   (if (= v nothing)
                                     (apply m-plus-maybe-t (rest mvs))
                                     (m-result v)))
                                 [(U x nil) -> (m (U x nil))]))))
                   (All [x] [(m (U x nil)) * -> (m (U x nil))])))
      ])))

\end{lstlisting}

\section{Conduit}

Conduits are an advanced form of ``pipes'' using arrows, and a generalisation of monads~\cite{ConduitRef}.
\emph{Conduit} is a Clojure library developed by Duey and 
contributors~\footnote{Github home of Conduit: \url{https://github.com/jduey/conduit}} 
that supports programming with conduits.

The library had to be modified and simplified considerably to be able to type check.
Many existing conduits had troublesome variable-parameters, reminiscent of the
issues of assigning types to \lstinline|assoc| and \lstinline|partial|
(which is future work, see section \label{sec:futurearity}).
In these cases, simplified versions of the conduits were created that 
took fixed arguments.

An interesting property of the types assigned to conduits (which
are internally functions of a single argument) is that it is
a recursive type. This is not in itself interesting, but Typed Clojure
implements conduit types as \emph{type functions} with variance,
which required the variance of the conduit types to be known in advance.

Listing \ref{lst:conduittypes} shows the type constructor for conduit,
\lstinline|==>|. Notice the body of \lstinline|==>| includes a reference
to itself in a position where variance must be known in advance.
\lstinline|declare-alias-kind| is used to declare the kind of an alias.
Once an alias with an already declared kind is defined with \lstinline|def-alias|,
the defined kind must match the declared kind, otherwise a type error is thrown.
(The specifics of the types in listing \ref{lst:conduittypes} are not relevant to the rest of this discussion,
so they are not explained).

\begin{lstlisting}[caption=Types for Conduits, label=lst:conduittypes]
...

(def-alias Result
  (TFn [[x :variance :covariant]]
    (U nil ;stream is closed
       '[] ;abort/skip
       '[x];consume/continue
       )))

(def-alias Cont
  (TFn [[in :variance :covariant]
        [out :variance :invariant]]
    [(U nil [(Result in) -> (Result out)]) -> (Result out)]))


(declare-alias-kind ==> (TFn [[in :variance :contravariant]
                              [out :variance :invariant]] Any))

(def-alias ==>
  (TFn [[in :variance :contravariant] 
        [out :variance :invariant]]
    [in -> '[(U nil (==> in out)) (Cont out out)]]))

...
\end{lstlisting}


The resulting port of \emph{Conduit} was unsatisfying for real world use.
The simplifications made in order to type check the library reversed
any effort \emph{Conduit} made to conform to common Clojure idioms.
The future work on improving variable-argument types should reveal whether
this library can be ported satisfactorily.


\chapter{Future Work}

\section{Variable-Arity Polymorphism}
\label{sec:futurearity}

It is not clear how to handle several key Clojure functions
that accept an even number of arguments.
For example, valid usages of \lstinline|assoc| are
\lstinline[mathescape]{(assoc m k v $\overrightarrow{k v}$)},
which takes three arguments and any number of paired arguments.

Strickland, Tobin-Hochstadt, and Felleisen's calculus~\cite{STF09}
is insufficient to express this pattern.
Devising and integrating types that can express this pattern 
is set as future work.

\section{Contracts and Blame}
\label{sec:contractsblame}

A key part of Typed Racket~\cite{Tob10} is its contract and blame system.
They enable safe interoperability with untyped Racket by generating runtime
contracts in key places based on static types. Typed Racket also 
includes a sophisticated blame system based on Wadler\cite{WF09}
to provide more accurate error reporting that ensures typed modules
``can't be blamed''.

There are two kinds of interoperability in Typed Clojure
that could utilize these features. First, the set of ``assumptions''
given by the programmer about Java interoperability like those introduced by \lstinline|non-nil-return| 
and \lstinline|nilable-param| (see section \ref{sec:experimentinterop})
could be checked by adding runtime contracts.
Secondly, when typed namespaces import untyped functions, the static type
assigned to the untyped function could be enforced by wrapping the
function in a runtime contract. Typed Racket uses this second approach
when importing untyped Racket functions, and it results in better error messages.

\section{Multimethods}
\label{sec:mmfuture}

Multimethods in Clojure are an idiomatic and often used feature.
It would be essential for an optional static type system for Clojure to support
The major hurdle for type checking multimethod definitions is the
interaction between the multimethod dispatch mechanism and occurrence typing.

\lstinline|clojure.core/isa?| is the core of multimethod dispatch.
It is a function that takes two arguments and is \lstinline|true| if
the first argument is a member of the second argument, otherwise \lstinline|false|.
It gets quite complicated quickly even with common usages of \lstinline|isa?|.
Simple calls involving only \emph{Class} objects return true if the left class
includes the right class in a Java type relationship, eg. \lstinline|(isa? Integer Number)| is \lstinline|true|.
For example, \lstinline|(isa? [Integer Double] [Number Number])| is \lstinline|true|
because the left and right arguments are vectors of classes of the same length
that are subtypes by \lstinline|isa?| pairwise.

It is also unclear whether it is feasible to perform comprehensiveness and
ambiguity prediction tests on multimethod usages. I predict that they will be
very difficult to pull off satisfactorily because of the ``openess'' of multimethods,
and we will fall back on the runtime errors that multimethods already throw.

\section{Records}

A Clojure record is a composite of datatypes and maps; it is a datatype
with fields that can be treated like a map: records support map operations like \lstinline|get| (lookup),
\lstinline|assoc| (add entries), and \lstinline|dissoc| (remove entries).
One interesting property from the perspective of static typing is that if
a entry corresponding to a field of the underlying datatype is removed
with \lstinline|dissoc|, then the resulting type is a plain map: it loses
its record type.
Correspondingly, a record keeps its type if any other entry is removed.

Records are used frequently in Clojure code, so it is desirable
to support them to some degree. Future work is planned to investigate solutions
to satisfactorily capture their subtle semantics in a practical way.

\subsection{Porting to other Clojure implementations}

It would be desirable to port Typed Clojure to other Clojure implementations.
Each implementation would bring its own set of challenges with interoperability.
The core of Typed Clojure would preserved with each port changing at least 
its interoperability with its host platform. Section \label{sec:portability}
discusses potential problems with the porting process, and how
they might be solved.

%\subsection{Extensions to heterogeneous map types}

%\subsection{Java Generics}

%\subsection{Linter}


\chapter{Conclusion}

Whether a language is dynamically typed or statically typed is not always a
dividing classification. Recent languages have managed to support the advantages of both styles:
the safety of a statically typed language while retaining the idioms found in
dynamically typed languages.

This dissertation describes an optional static typing system
that attempts to bring the advantages of static typing to the dynamically typed language
Clojure, running on the Java Virtual Machine. 
By reusing many of the design choices made in similar projects
like Typed Racket, we are able to design and implement a prototype type system
\emph{Typed Clojure} that can statically check many Clojure idioms.

Typed Clojure is intended to be of practical use to Clojure programmers.
We show that Typed Clojure helps verify the absence of errors in Clojure code written in sophisticated
programming styles by porting most of a wide-spread Clojure library for monadic programming.
Similarly, we show that Typed Clojure helps verify code using Java interoperability is correct, and is
a useful tool to show misuse of Java's \emph{null}.

There is still significant future work in order to type check all Clojure idioms,
but the work already carried out suggests that an optional type system for Clojure
like Typed Clojure is both practical and useful tool.

\printbibliography[title=References]

\appendix
\chapter{Heterogeous map type theory prototype}
\label{sec:hmaptheory}

\section{Operational Semantics}
 
%$$
%\begin{tdisplay}{Test1}
%\begin{altgrammar}
%  \s{},\t{} &::= & \int \alt \bool \alt {\proctype {\s{}} {\t{}}} 
%  &\mbox{Types}\\
%  \M{}, \N{}, \P{}, \dots &::=& \x{} \alt \abs{\x{}}{\M{}} \alt
%  \comb{\M1}{\M2}  &\mbox{Raw Terms} \\
%\end{altgrammar}
%\end{tdisplay} 
%$$

 
$$
\begin{tdisplay}{Syntax of Terms}
\begin{altgrammar}
  \exp{} &::=& (\c{} \overrightarrow{\exp{}})         % function application, multiple arguments
             \alt \{ \overrightarrow{\exp{}\ \exp{}} \} % map literal
             &\mbox{Expressions} \\ 
  \v{} &::=& \k{} \alt \nil{}
              \alt \{ \overrightarrow{\v{}\ \v{}} \}   % map literal
              &\mbox{Values} \\
  \c{} &::=& \assoc{} \alt \dissoc{} \alt \get{}
              &\mbox{Constants}
\end{altgrammar}
\end{tdisplay}
$$

$$
\begin{tdisplay}{Evaluation Contexts}
  \begin{altgrammar}
    \E{} &::=& [ ] % application rules
              \alt (\c{}\ \overrightarrow{\v{}}\ \E{}\ \overrightarrow{\exp{}}) % eval arguments left-to-right
              % map rules
              \alt \{\overrightarrow{\v{}\ \v{}}\ \E{}\ \exp{}\ \overrightarrow{\exp{}\ \exp{}} \} % key first
              \alt \{\overrightarrow{\v{}\ \v{}}\ \v{}\ \E{}\ \overrightarrow{\exp{}\ \exp{}} \}   % value next
              &\mbox{Evaluation Contexts}
  \end{altgrammar}
\end{tdisplay}
$$ 
 
% \mapsto for updating maps
$$
\begin{tdisplay}{Operational Semantics}
\begin{array}{c}
{\inferrule[E-Assoc]
  { {v_1 = \{ \overrightarrow{\v{}\ \v{}} \} } \\
    {v_4 = v_1\ with\ entry\ v_2\ to\ v_3} }
    {{(\assoc{}\ v_1\ v_2\ v_3) \hookrightarrow v_4} }} \\
\\
{\inferrule[E-Dissoc]
  { {v_1 = \{ \overrightarrow{\v{}\ \v{}} \} } \\
    {v_3 = v_1\ without\ entry\ indexed\ by\ v_2} }
  {(dissoc\ v_1\ v_2) \hookrightarrow v_3} } \\
\\
{\inferrule[E-GetMapExist]
  { {v_1 = \{ \overrightarrow{\v{}\ \v{}} \} } \\
    {v_2\ v_3\ in\ v_1} }
%    {v_1\ has\ entry\ with\ key\ v_2} \\
%    {v_3 = corresponding\ value\ in\ entry\ with\ key\ v_2\ in\ v_1}
  { {(get\ v_1\ v_2) \hookrightarrow v_3} }} \\
\\
{\inferrule[E-GetMapNotExist]
  { {v_1 = \{ \overrightarrow{\v{}\ \v{}} \} } \\
    {v_1\ has\ no\ entry\ with\ key\ v_2} \\
    {v_3 = \nil{}} }
  { {(get\ v_1\ v_2) \hookrightarrow v_3} }}
\end{array} 
\end{tdisplay} 
$$

$$
\begin{tdisplay}{Type Syntax}
\begin{altgrammar}
  \T{} &::=& \quad \Top \alt \Bottom \alt \Nil \alt \k{} \alt \{ \overrightarrow{\k{}\ \T{}} \} 
             \alt {\IPersistentMap {\T{}} {\T{}}} \\
             && \alt ( \vee\ \overrightarrow{\T{}} )
                \alt ( \wedge\ \overrightarrow{\T{}} )
\end{altgrammar}
\end{tdisplay}
$$

% what about get? Core Type Rules

$$
\begin{tdisplay}{Core Type Rules}
\begin{array}{c}
{\inferrule[T-AssocHMap]
  { {\Gamma \vdash {\hastype {{\exp{}}_{m}} {{\T{}}_{m}}} } \\
    {\Gamma \vdash {\hastype {{\exp{}}_{k}} {{\k{}}}} } \\
    {\Gamma \vdash {\hastype {{\exp{}}_{v}} {{\T{}}_{v}}} } \\
    {\vdash {\issubtype {{\T{}}_{m}} {\TopHMap} }} }
  { {\Gamma \vdash {\hastype {( \assoc{}\ {\exp{}}_{m}\ {\exp{}}_{k}\ {\exp{}}_{v})} 
        {(update\_hmap\ {{\T{}}_{m}}\ {{\k{}}}\ {{\T{}}_{v}})}}}}} \\
\\
{\inferrule[T-AssocPromote]
  { {\Gamma \vdash {\hastype {{\exp{}}_{m}} {{\T{}}_{m}}} } \\
    {\Gamma \vdash {\hastype {{\exp{}}_{k}} {{\T{}}_{k}}} } \\
    {\Gamma \vdash {\hastype {{\exp{}}_{v}} {{\T{}}_{v}}} } \\
    {\vdash {\issubtype {{\T{}}_{m}} {\TopIPersistentMap}}} \\
    {\T{} = (promote\_hmap\ {\T{}}_{m}\ {\T{}}_{k}\ {\T{}}_{v})} }
  { {\Gamma \vdash {\hastype {( \assoc{}\ {\exp{}}_{m}\ {\exp{}}_{k}\ {\exp{}}_{v})} {\T{}}}}}} \\
\end{array}
\end{tdisplay}
$$

$$
\begin{tdisplay}{Type Metafunctions}
\begingroup
\fontsize{10pt}{12pt}
\begin{altgrammar}
  (update\_hmap\ \{ \overrightarrow{{\T{}}_{k}\ {\T{}}_{v}} \}\ {\k{}}_1\ {\T{}}_1) = 
                 \{ \overrightarrow{{\T{}}_{k}\ {\T{}}_{v}}\ {\k{}}_1\ {\T{}}_1 \}\ \\
  (update\_hmap\ (\wedge\ \overrightarrow{{\T{}}})\ {\T{}}_{k}\ {\T{}}_{v}) = 
                               (\wedge\ \overrightarrow{(update\_hmap\ {\T{}}\ {\T{}}_{k}\ {\T{}}_{v})}) \\
  (update\_hmap\ (\vee\ \overrightarrow{{\T{}}})\ {\T{}}_{k}\ {\T{}}_{v}) = 
                               (\vee\ \overrightarrow{(update\_hmap\ {\T{}}\ {\T{}}_{k}\ {\T{}}_{v})})
  \\ \\
  (promote\_hmap\ \{ \overrightarrow{{\T{}}_{k}\ {\T{}}_{v}} \}\ {\T{}}_{kn}\ {\T{}}_{vn})
                                    = {\IPersistentMap {(\vee\ \overrightarrow{{\T{}}_{k}}\ {\T{}}_{kn})}
                                                       {(\vee\ \overrightarrow{{\T{}}_{v}}\ {\T{}}_{vn})}} \\
  (promote\_hmap\ {\IPersistentMap {{\T{}}_{k}} {{\T{}}_{v}}} \ {\T{}}_{kn}\ {\T{}}_{vn})
                                    = {\IPersistentMap {(\vee\ {\T{}}_{k}\ {\T{}}_{kn})}
                                                       {(\vee\ {\T{}}_{v}\ {\T{}}_{vn})}}
\end{altgrammar}
\endgroup
\end{tdisplay}
$$

%$$
%\begin{tdisplay}{Subtyping}
%\begin{array}{c}
%\inferrule[S-Refl] { } {\vdash {\issubtype {\T{}} {\T{}}}} \\
%\inferrule[S-Any] { } {\vdash {\issubtype {\T{}} {\Top}}}
%\inferrule[S-UnionSuper] 
%  {\ } 
%  {\vdash {\issubtype {\T{}} {\Top}}}
%\end{array}
%\end{tdisplay}
%$$

\chapter{Dissertation Proposal}

\section*{Background} 

% In this section you should give some background to your
% research area. What is the problem you are tackling, and why is it
% worthwhile solving? Who has already done some work in this area,
% and what have they achieved?

Dynamically typed languages (also known as monotyped) are designed to be convenient for
writing programs quickly, and aspire to get out of the programmer's
way as much as possible. When programs grow large and stabilize,
some features of static languages are missed, specifically static
type checking. 

Tobin-Hochstadt notes that ``untyped scripts are difficult to 
maintain over the long run''~\cite{Tob10} because types
contain valuable design information.
He developed Typed Scheme~\cite{Tob10} 
to safely and incrementally port existing Scheme code, a dynamic language, to
Typed Scheme, a static language.

Clojure is a dynamically typed, functional language with implementations for the Java Virtual
Machine and Common Language Runtime, and compiling to Javascript. Clojure
is also a Lisp, which makes it a good candidate to test the generality of
ideas developed for Typed Scheme~\cite{Tob10}.

\section*{Aim} 

% Now state explicitly the hypothesis you aim to
% test. Make references to the items listed in the Reference section
% that back up your arguments for why this is a reasonable
% hypothesis to test, for example the work of Knuth~\cite{knuth}.
% Explain what you expect will be accomplished by undertaking this
% particular project.  Moreover, is it likely to have any other
% applications?

My goal is to develop a prototype optional static type system for Clojure, 
eventually intended for practical use.

It will be based on the lessons learnt throught the development
of Typed Scheme, and as a response to Tobin-Hochstadt's~\cite{Tob10}
suggestion to add type systems to other existing dynamic languages.

There are several challenges to creating a satisfactory type system for Clojure.

Multimethods play a significant role in Clojure, and a satisfactory
type system for Clojure should understand them to some degree.
There is some experience statically typing multimethods~\cite{MS02}, 
but this is a challenge and may not be addressed in the
timeframe proposed, beyond initial consideration of how it would
fit with the rest of the system.

Clojure's core library includes variable-arity functions which
are not easily typed with current static type systems. 
Providing satisfactory types for functions 
like Clojure's 'map', 'filter', and 'reduce' require
support for non-uniform variable arity polymorphism. 
Strickland, Tobin-Hochstadt and Felleisen~\cite{STF09}.
describe their approach for non-uniform variable arity polymorphism, as used
in Typed Racket. 
At a minimum, the proposed project will include broadly identifiying 
what new issues arise when adapting this approach to Clojure, 
with a prototype design if these issues are not too major.

Occurence typing is type checking technique developed for Typed Scheme~\cite{TF08}
and improved for Typed Racket~\cite{TF10}.
It helps the type checker understand common programming idioms 
with minimal type annotations.
The proposed project will include a comparison of these approaches in
the context of Clojure, and a prototype design if a satisfactory approach 
is designed and no major issues are identifed.

Ensuring vigorous type safety is an important aspect of Typed Scheme~\cite{Tob10},
especially when interacting between untyped and typed modules.
I do not expect to concentrate on every combination of cross-module
interaction. In particular, safely using typed code from untyped code
will not be designed. Safe interaction between typed modules
however is needed for a practical type checker. I
will attempt to design a viable strategy or identify issues that
require further consideration.

\section*{Method}

% In this section you should outline how you intend to go
% about accomplishing the aims you have set in the previous
% section. Try to break your grand aims down into small,
% achievable tasks. Try to estimate how long you will
% spend on each task, and draw up a timetable for each
% sub-task.

The milestones for this project are broken into release milestones
for the prototype library.

My goal is to finish all features listed in releases below
and to begin work on the challenges, listed last.

For each of these it's not yet fully clear what novel issues may 
arise in adapting existing techniques to Clojure.  For each item 
below the project will involve either implemention or identification 
of the novel issues that make implementation difficult.  
The classification below reflects the level of uncertainity 
regarding such issues.

\begin{description}
\item[0.1]
  \begin{itemize}
  \item Union types
  \item Basic Local Type Inference algorithm
  \item Fixed arity function types
  \item Typed deftype (class definitions)
  \item Uniform variable-arity function types
  \item Enough functions annotated from clojure.core for proof-of-concept
  \item Bounded Type variables
  \end{itemize}
\item[Challenges]
  \begin{itemize}
  \item Devise strategy for type inference (eg. occurence typing)
  \item Manage interactions between typed namespaces
  \item Mutable reference types
  \item Non-uniform variable arity polymorphism
  \item Typing Multimethods
  \item Fine Grained Hash-Map Types
  \end{itemize}

\end{description}
\section*{Software and Hardware Requirements}
% Outline what your specific requirements will be with regard
% to software and hardware, but note that any special requests
% might need to be approved by your supervisor and the Head of
% Department.

Linux environment with Java, git, and maven installed.
No issues are anticipated with regards to access to these.

% Overall, you should aim to produce roughly a two page document
% (and certainly no more than four pages)
% outlining your plan for the year.


\end{document}
