\documentclass{cshonours}

\usepackage{mmm}
\usepackage{mathpartir}
\usepackage{clj-grammar}
\usepackage[style=numeric]{biblatex}
\addbibresource{bibliography.bib}

\usepackage{listings}
\lstset{ %
  language=Lisp,                % choose the language of the code
  columns=fixed,basewidth=.5em,
  basicstyle=\small\ttfamily,       % the size of the fonts that are used for the code
  %numbers=left,                   % where to put the line-numbers
  %numberstyle=\small\ttfamily,      % the size of the fonts that are used for the line-numbers
  %stepnumber=1,                   % the step between two line-numbers. If it is 1 each line will be numbered
  %numbersep=5pt,                  % how far the line-numbers are from the code
  %backgroundcolor=\color{white},  % choose the background color. You must add \usepackage{color}
  %showspaces=false,               % show spaces adding particular underscores
  showstringspaces=false,         % underline spaces within strings
  %showtabs=false,                 % show tabs within strings adding particular underscores
  %frame=single,           % adds a frame around the code
  %tabsize=2,          % sets default tabsize to 2 spaces
  captionpos=t,           % sets the caption-position to bottom
  breaklines=true,        % sets automatic line breaking
  breakatwhitespace=true,    % sets if automatic breaks should only happen at whitespace
  %escapeinside={\%*}{*)},          % if you want to add a comment within your code
}

\title{Typed Clojure: A Practical Optional Type System for Clojure}
\author{Ambrose Bonnaire-Sergeant}

\keywords{Type systems, Clojure}
% TODO What are categories? cshonours requires them
\categories{}

\begin{document}

\maketitle

\begin{abstract}
  % FIXME long tradition for attempts
  There is a long tradition to create static type systems for dynamically-typed
  languages. Static type checking promises to eliminate many common user errors
  that are otherwise unnoticed or are caught when the program is run. 

  Recent attempts have seen several promising and innovative
  type systems built for dynamically typed languages. Typed Racket is able to
  statically type check idioms ..
  We take the lessons learnt from these
  projects and apply them to the dynamically-typed language Clojure in the form of Typed Clojure.
\end{abstract}

%\tableofcontents

\chapter{Introduction}

\section{Thesis}

\emph{It is practical and useful to design and implement an optional typing system 
for the Clojure programming language that allows Clojure programmers to continue 
using idioms and style found in current Clojure code.}

\section{Motivation}

% Type Systems for lisp?
% Why type systems?
% Why Clojure?

In the last decade it has become increasingly common to enhance
dynamically typed languages with static type systems. This idea not new,
but recent attempts are noteworthy for their successful use of bidirectional checking.
Instead of always attempting to infer types the algorithm relies on programmer annotations
appearing in some natural places, such as giving the type of each top-level function.

The Clojure programming language is a dynamically typed language running on the Java Virtual Machine. 
It emphasises functional programming with immutable data structures
and provides direct interoperability with languages on the Java Virtual Machine.
Clojure also is also a family of languages with similar semantics, referred to as Clojure dialects.
There are several Clojure dialects, notably the aforementioned Clojure for the Java Virtual Machine,
ClojureCLR which targets the Common Language Runtime, and ClojureScript which compiles to Javascript.
We refer to Clojure the language as \emph{Clojure}, and the family of Clojure-like languages as
\emph{Clojure dialects}.

Clojure dialects are widely used for several reasons.
Performance is a key feature, with most dialects 
offering access to host-level performance 

Recently languages have been created to or been modified to support aspects
of both static and dynamic typing.
Dart is fundamentally dynamically typed but offers a form of optional
static typing. Haskell recently supports the ability to delay static type errors 
to runtime, removing the need for the program to type check before running.
Scala can emulate some features of dynamic languages with Dynamic Scala.

\subsection{What kind of type system does Typed Clojure provide?}

% What are types? No real agreement .. but here is my definition..
% Every programmer thinks they know what they are. More than one concept
% in the literature.
% Static vs. dynamic
% intrinsic vs. extrinsic

% is it even a type system?
% type refinements vs ordinary types
% extrinsic vs intrinsic type systems

% refinements check invariants of programs
% while ordinary static types check whether programs
% are basically meaningful - Rowan + Frank P
% Page 1 Rowan's PhD
\begin{verbatim}Following the terminology of [Pfenning, Reynolds ..]\end{verbatim}
An ordinary static type system is used to check whether programs are basically
meaningful. Such type systems are \emph{intrinsic}. A language with an intrinsic
static type system has run-time semantics that depends on the types of associated
variables and expressions during type checking.
For example, C, Java and ML have intrinsic types.
This means such programs must pass the type checker before being run.

A static type system is \emph{extrinsic} when runtime semantics
does not depend on the type system. In other words, passing a static type checker
is not essential to running programs. A dynamically typed language can be viewed
as having a trivial static type system that supports exact one type,
\begin{verbatim} .. a view strongly advocated by Harper[..]
and common in the literature on static types (see, eg. Pierce[..])
\end{verbatim}


A type checker ensures invariants hold complains if code
does not conform to its documented type. Powerful type systems like
Haskell's and Scala's are useful when reasoning about higher-order code.

When a static type checker is not available, which describes the situation
for most dynamic languages, other techniques are used for checking
type invariants. Runtime contracts and unit testing are a popular alternatives.
Clojure adopts these approachs, providing easy syntax for defining pre-
and post-conditions \footnote{Inspired by Eiffel} and a library for writing unit tests.
\begin{verbatim} but.. (why a static type system is still desirable\end{verbatim}

% clojure is dynamically typed
% currently few tools to statically reason about complex higher-order code
% conduit based on streams, uses continuations
% core.logic based on monads
% Static types are useful for reasoning for such things
% Strong statically typed languages like Haskell use monads regularly

% strength of Clojure is concision
% typically, concise static languages are built concurrently with their type system
% Eg. Haskell, ML flavoured languages
% Counterpoint: Typed Racket preserves racket idioms, adds static typing

\section{Typed Clojure through Examples}

% Hello world
This section introduces Typed Clojure with example code. An attempt is
made to introduce some Clojure syntax and sematics to those unfamiliar or needing a refresher.
A basic knowledge of Lisp syntax is assumed.

We begin with the obligatory \emph{Hello world} example.

\begin{lstlisting}[caption=Hello world]
(ns typed.test.hello-world
  (:require [typed.core :refer [check-ns]]]))

(println "Hello world")
\end{lstlisting}

The \lstinline|clojure.core/ns|
\footnote{All vars from the \lstinline|clojure.core| namespace are referred implicitly in new namespaces (equivalent to \lstinline|(:require [clojure.core :refer :all])|).}
invocation declares the current namespace to be \lstinline|typed.test.hello-world|
and requires the namespace \lstinline|typed.core| to be loaded as a dependency, referring the var
\lstinline|typed.core/check-ns| to be referenced locally. Other than this
dependency, this is identical to the untyped \emph{Hello world}.
% footnote: check-ns used to initiate type checking usually at REPL

More complex code require extra annotations to type check:

\begin{lstlisting}
(ns typed.test.collatz
  (:require [typed.core :refer [check-ns ann]]))

(ann collatz [Number -> Number])
(defn collatz [n]
  (cond
    (= 1 n) 
      1
    (and (integer? n) 
         (even? n)) 
      (collatz (/ n 2))
    :else 
      (collatz (inc (* 3 n)))))
\end{lstlisting}
\footnote{\emph{Example adapted from Sam's dissertation TODO}}

In this example, we define a new var \lstinline|typed.test.collatz/collatz|. Typed Clojure requires all 
used vars to be annotated. Here \lstinline|typed.core/ann| annotates \lstinline|typed.test.collatz/collatz|
to be a function from \lstinline{java.lang.Number} to 
\lstinline{java.lang.Number}
\footnote{All Classes in the \lstinline|java.lang| package
are automatically imported in every Clojure namespace (the equivalent of Java's \lstinline|import java.lang.*;|).}.

\subsection{Datatypes and Protocols}

We can annotate datatype and protocol definitions similarly.

\begin{lstlisting}
(ns typed.test.deftype
  (:require [typed.core 
             :refer [check-ns ann-datatype
                     tc-ignore ann-protocol AnyInteger]]))

(ann-protocol Age 
  :methods
  {age [Age -> AnyInteger]})
(tc-ignore
  (defprotocol Age
    (age [this])))

(ann-datatype Person 
  [[name :- String]
   [age :- AnyInteger]])
(deftype Person [name age]
  Age
  (age [this] age))

(age (Person. "Lucy" 34))
\end{lstlisting}

\lstinline|clojure.core/defprotocol| defines a new Clojure protocol\footnote{See http://clojure.org/protocols for a full description of protocols}
with a set of methods. \lstinline|typed.core/ann-protocol| annotates a protocol with the types of its methods.
In this example, we define a protocol \lstinline|typed.test.person/Age| with an \lstinline|age| method.
The call to \lstinline|clojure.core/defprotocol| also defines a new var \lstinline|typed.test.person/age|, a first-class function
wrapping the \lstinline|age| method, but taking the target Object as the first parameter. The
type signature provided with \lstinline|typed.core/ann-protocol|, here \lstinline|[typed.test.person/Age -> typed.core/AnyInteger]|, 
is for this function.

Invocations of \lstinline|clojure.core/defprotocol| are currently not able to be type checked
and are ignored by Typed Clojure by passing them to \lstinline|typed.core/tc-ignore|.

\lstinline|clojure.core/deftype|
defines a new Clojure datatype\footnote{See http://clojure.org/datatypes for a full description of datatypes}
in the current namespace with a number of fields and methods. 
\lstinline|typed.core/ann-datatype| annotates a datatype with its field types.
In this example, we create a datatype \lstinline|typed.test.person.Person|
with fields \lstinline|name| and \lstinline|age| and implement the \lstinline|age|
method from protocol \lstinline|typed.test.person/Age|.

Java constructors are invoked in Clojure by suffixing the Class we want to instantiate with a dot.
Datatypes are implemented as Java Classes with immutable fields (by default) and a single constructor, taking as arguments its fields 
in the order they are passed to \lstinline|deftype|
\footnote{When unambiguous, I omit the qualifying namespace/package for the remainder of the chapter.}.
\lstinline|(Person. "Lucy" 34)| constructs a new \lstinline|Person|
instance, setting the fields to their corresponding positional arguments.
Typed Clojure checks the datatype constructor to be the equivalent of 
\lstinline|[String AnyInteger -> Person]|.

Finally, Typed Clojure checks invocations of Protocol methods. It infers \lstinline|Person|
is an instance of \lstinline|Age| from the datatype definition, therefore \lstinline|(age (Person. "Lucy" 34))| is type-safe.

\subsection{Polymorphism}

Typed Clojure supports F-bounded polymorphism. All type variables may have
upper and lower bounds.

Typed Clojure parameterises some of Clojure's data structures. For example,
the interface behind Clojure's \emph{seq} abstraction \lstinline|clojure.lang.Seqable| has one 
covariant parameter\footnote{\lstinline|(Seqable Integer)| being a subtype of \lstinline|(Seqable Number)|
because Integer is a subtype of Number.}.

\begin{lstlisting}
...
(ann to-set 
     (All [x]
       [(U nil (Seqable x)) -> (clojure.lang.PersistentHashSet x)]))
(defn to-set [a]
  (set a))
...
\end{lstlisting}

In this example\footnote{When convenient, namespace declarations are omitted for the remainder of the chapter.}, 
we define \lstinline|to-set|, aliasing \lstinline|clojure.core/set|.
\lstinline|All| introduces a set of type variables to the body of a type,
here \lstinline|x| is used to define a relationship between the input type and return type.

\lstinline|(U nil (Seqable x))| is a common type in Typed Clojure, read as the union
of the type \lstinline|nil| and the type \lstinline|(Seqable x)|.
The vast majority of types for collection processing functions in the Clojure core library feature
it as an input type, where passing \lstinline|nil| either has some special behaviour 
or is synonymous with passing an empty \lstinline|Seqable|.

\subsection{Heterogeneous Maps}

Where particularly object-oriented languages reach for objects, Clojure
utilises maps. Clojure provides a map literal using curly braces. For example,
\lstinline|{:a 1, :b 2}| is a map with two key-value entries: from Keyword key \lstinline|:a|
to value \lstinline|1|, and Keyword key \lstinline|:b| to value \lstinline|2|. Note that commas are always
whitespace in Clojure and are included for readability.

Typed Clojure provides a heterogeneous map type, restricted to 
maps with singleton Keyword keys. This restriction is reflected
in the syntax for defining heterogeneous map types.

\begin{lstlisting}
...
(ann config
     (HMap {:file String,
            :ns Symbol}))
(def config
  {:file "clojure/core.clj",
   :ns 'clojure.core})
...
\end{lstlisting}

This example checks \lstinline|config| to be a heterogeneous map
with \lstinline|:file| and \lstinline|:ns| keys, with values of
type \lstinline|String| and \lstinline|Symbol| respectively.

Heterogeneous vector and seq types are also provided.

\subsection{Multi-Arity Functions}

Clojure function objects support multiple arities, allowing dispatch
on the number of supplied arguments.

\begin{lstlisting}
(ns typed.test.poly
  (:require [typed.core :refer [ann AnyInteger check-ns cf]])
  (:import [clojure.lang Seqable]))

(ann repeatedly'
     (All [x]
       (Fn [[-> x] -> (Seqable x)]
           [AnyInteger [-> x] -> (Seqable x)])))
(defn repeatedly'
  "Takes a function of no args, presumably with side effects, and
  returns an infinite (or length n if supplied) lazy sequence of calls
  to it"
  ([f] (lazy-seq (cons (f) (repeatedly' f))))
  ([n f] (take n (repeatedly' f))))
\end{lstlisting}

This example defines \lstinline|repeatedly'|\footnote{Identical to \lstinline|clojure.core/repeatedly|},
a multi-arity function taking either a zero-argument function, or an integer and a zero-argument function as arguments.
The type variable \lstinline|x| links the return type of the generator function to
the resulting \lstinline|Seqable| instance.

When checking the definition of multi-arity functions, Typed Clojure
finds the matching function types for a particular arity by number of arguments
and requires each relevant type to check successfully.

\subsection{Variable-Arity Functions}

Clojure functions can take an optional \emph{rest} argument.

\subsubsection{Uniform Variable-Arity}

\subsubsection{Non-uniform Variable-Arity}

\subsection{Occurrence Typing}

Typed Clojure uses occurrence typing to improve inference\footnote{TODO occurrence typing reference}.
Occurrence typing is used to refine types for immutable local bindings 
as programs follow conditional branches.

\begin{lstlisting}
(ann num-vec2 
     [(U nil Number) (U nil Number) -> (Vector* Number Number)])
(defn num-vec2 [a b]
  [(if a a 0) 
   (if b b 0)])
\end{lstlisting}

To check this example, occurrence typing infers type information based on the result of each test.
In Clojure, \lstinline|nil| and \lstinline|false| are false values and all other values are true.
The test \lstinline|(if a a 0)| follows the \emph{then} branch is \lstinline|a| is not \lstinline|nil|
or \lstinline|false|. When checking this branch, we can safely refine the type of \lstinline|a| to \lstinline|Number| from
\lstinline|(U nil Number)|. Similarly, following the \emph{else} branch refines the type of \lstinline|a|
to \lstinline|nil| from \lstinline|(U nil Number)|.

\subsection{Java Interoperability}

Typed Clojure currently provides safe interop\footnote{Interop is short for interoperability} with (non-Generic) Java
\footnote{Compatibility with Generic Java is planned future work}.

The key to safe Java interop is the treatment of Java's \emph{null}.
\emph{null} is a subtype to all reference types
represented in Clojure by \lstinline|nil|. Unlike Java's type system
Clojure explicitly separates \emph{null} and reference types, allowing
Typed Clojure to express
\emph{nullable}
\footnote{A type is \emph{nullable} if it may also be an instance of \emph{null},
which is expressed in Typed Clojure by creating a union of the reference type and \lstinline|nil|.}
positions.

\begin{lstlisting}
(ns typed.test.interop
  (:import (java.io File))
  (:require [typed.core :refer [ann non-nil-return check-ns]]))

(ann f File)
(def f (File. "a"))

(ann prt (U nil String))
(def prt (.getParent ^File f))

(non-nil-return java.io.File/getName :all)
(ann nme String)
(def nme (.getName ^File f))

\end{lstlisting}

This example shows how Typed Clojure handles \emph{null} while creating and
using an instance of \emph{java.io.File}\footnote{See Javadoc: http://docs.oracle.com/javase/1.4.2/docs/api/java/io/File.html}.

Typed Clojure checks calls to Java constructors by requiring the provided
arguments be acceptable input to at least one constructor for that Class.
In this case, \emph{java.io.File} has a constructor accepting a \emph{java.lang.String}
argument, so \lstinline|(File. "a")| is type safe. Java constructors never
return \emph{null}, so Typed Clojure assigns the return type to be \lstinline|File|.
This constructor is equivalent to \lstinline|[String -> File]| in Typed Clojure.

Next, we see how Typed Clojure's default behaviour treats method return positions as nullable.
The \emph{java.io.File} instance method \emph{getParent}
is equivalent to \lstinline|[-> (U nil String]| in Typed Clojure. This happens to be
a valid approximation of the method as \emph{getParent} returns \emph{null} 
``if the pathname does not name a parent directory''\footnote{See Javadoc: http://docs.oracle.com/javase/1.4.2/docs/api/java/io/File.html}.
On the other hand, the instance method \emph{getName} always returns an
instance of \emph{java.lang.String}, so we set the return position of
\emph{getName} to non-nil with \lstinline|typed.core/non-nil-return|
\footnote{The second parameter to \lstinline|non-nil-return| specifies which arities to assume non-nil 
returns, accepting either a set of parameter counts of the relevant arities, or \lstinline|:all|
to override all arities of that method.}.

% TODO
% What does this Lit review cover?
% Why things are included/not included?
% Background
% Heterogeneous Maps/record types
% Soft typing
% See Sam's dissertation
% Lisp type system background (197x)
% Why was it so hard?
% - bidirectional checking vs. global inference

\chapter{Literature Review}

In this chapter we fit the design of Typed Clojure with literature from related fields.
Forward-references are given to link the discussed literature with this project
where there is some relevance or influence.

%Typed Clojure is related to several existing systems for both typed and untyped languages.

\section{Static Type Systems}

% Haskell
% OCaml
% Forsythe
% Java C#
% Scala

\section{Static Types for Untyped Languages}

There is a history of designing static type systems for untyped languages. 

\subsection{Soft Typing}

Soft typing\cite{CF91}
is an approach for ensuring type safety in untyped languages.
A soft type system infers types for programs, distinguishing between degrees
of potential type safety.
A soft type checker uses this information to 
preserve type safety by inserting appropriate checks 
and informs the programmer of potential inconsistencies.
For example, if the soft type system detects a portion of code is sometimes not type safe,
the soft type checker inserts a check that throws a runtime error upon unsafe usages.
Thus, soft type systems differ from traditional type systems in that type inference 
never fails and an inconsistency always results in a runtime check.

Wright and Cartwright\cite{WC97} developed Soft Scheme, a practical
soft type system for Scheme. 
It extended earlier work by Cartwright and Fagan\cite{CF91}
and Fagan \cite{Fag91}, adding support for practical features such as
first-class continuations and variable-arity functions.
Soft Scheme does not require any extra type annotations.

Type systems for Scheme have since moved away from soft typing
to to other approaches like gradual typing.
Alas, the best reference for commentary on this transition 
appears to be slides from a talk by Felleisen\cite{Fell09},
a leader in this area of research for over 20 years.
Felleisen says that while Soft Scheme discovered type problems, 
it suffers from incomprehensible
errors that required PhD-level expertise to decipher. 

\begin{verbatim}
  Bidirectionality
\end{verbatim}

\subsection{Gradual Typing}

Gradual typing combines static and runtime type checking so programmers
can the most appropriate one for the situation.

Typed Racket was developed as a path for module-by-module
porting of existing untyped Racket modules to a typed sister language\cite{Tob10}.
Once a module is ported and type checked, it is protected from untyped modules
by inserting runtime checks.

\begin{verbatim}
  Different levels of Gradual Typing discussed by Siek..
\end{verbatim}

\section{Interlanguage Interoperability}

This section compares several existing languages that feature interlanguage interoperability.

Clojure is a dynamic functional language hosted on the Java Virtual Machine. It provides 
interoperability with Java libraries. As Clojure is a dynamically typed language, it does
not give strong type type guarantees at compile time whether interactions with Java
are type safe.

Scala is a statically typed language on the Java Virtual Machine offering integrated interoperability with Java, a typed language.
Scala objects and classes can ``inherit from Java classes and implement Java interfaces''\cite{OCD+}
with the usual static type guarantees normal Scala code enjoys.
Scala offers an Option type to safely eliminate null pointers.
Java Generics are also fully supported by Scala, accounting for Scala support existential types

Typed Racket includes safe interoperability between any combination of typed and untyped 
Racket modules\cite{Tob10}\cite{TF08}. 
Interactions with untyped modules are protected by adding runtime checks based on expected types.
Typed Racket implements a sophisticated blame calculus. It ensures 
error messages correctly \emph{blame} the source of type errors,
which can be difficult to determine in the presence of higher-order functions. 
\begin{verbatim}references, Walder etc.\end{verbatim}

\section{Record Types}

This section summarises research in typing OCaml-style records
which relate to Clojure records and heterogeneous hash maps.

\section{Intersection, Union, and Singleton Types}
\label{sec:intersection_types}

% Hiyashi Singleton types
% Intersection types - CDV81
% Unions in soft typing CF91
% - Static Type Inference in a Dynamically Typed Lanugage
%  - cited by CF91, which has unions

Intersection and union types are interesting type constructs relevant to capturing the complicated
types common in dynamic languages.
An expression of type \lstinline|(I a b)|, an intersection type including types \lstinline|a|
and \lstinline|b| in Typed Clojure, can be used safely in both positions expecting type \lstinline|a|,
and positions expecting type \lstinline|b|.
An expression of type \lstinline|(U a b)|, a union type including types \lstinline|a| in Typed Clojure,
and expressions of this type can be used safely in positions 
that expect a type that is \emph{either} type \lstinline|a|
or \lstinline|b|.
For example, \lstinline|(U Number Boolean)| cannot be used in positions expecting \lstinline|Number|.
% TODO Intersection example

Several interesting projects have used intersection or union types.

Forsythe, a modern ALGOL dialect by Reynolds\cite{Rey96} was the first wide-spread 
programming language to use intersection types.
Uses of intersections in Forsythe include representing extensible record types
and function overloading.

Refinement types add a level of types refining an existing type system
in order to type check more detailed properties and invariants than standard static type systems
(described by Freeman and Pfenning\cite{FP91}).
Intersection types are critical here to allow more than one property or invariant
to be expressed for a function.
SML CIDRE is a refinement type checker for Standard ML by Davies \cite{Dav05}.
\begin{verbatim}
Refinement types are similar to this project ... how?
\end{verbatim}

St-Amour, Tobin-Hochstadt, Flatt, and Felleisen
describe the \emph{ordered intersection types} used in Typed Racket\cite{St12}
that provide a kind of function overloading.
Typed Clojure takes a similar approach
\begin{verbatim}
described in chapter
\end{verbatim}
Typed Racket also uses disjoint union types, 

\section{Java Interoperability}

Scala and Clojure are both languages with a focus on host interoperability, specifically
to the Java Virtual Machine. Their differing treatment of the Java Virtual Machine's \lstinline|null| is
relevant, particularly considering the potential of using occurrence typing \cite{TF10}
to prevent erroneous usage of \lstinline|null|.
\begin{verbatim}
eg.
\end{verbatim}

\section{Function Types}

There are several different approaches to representing functions in programming languages.

In typed languages like Haskell, functions are as simple as possible, taking a single argument.
A function with multiple arguments is represented by chaining several single-argument function
together, or by using lists, or using tuples. This style is characterised by direct syntactic function currying, where applying
a function to less than its maximum number of arguments results in another function
that takes the remaining arguments.

In untyped languages like Scheme, functions can take any number of arguments. It is an
error to supply less than the minimum number of arguments to a function.
This style features sophisticated support for functions with variable-arity. For example,
functions can dispatch on the number of arguments provided, and supports a \emph{rest} parameter
as its last parameter which can accept any number of arguments.

In this regard, Clojure takes an approach identical to Scheme, and supports all the features
mentioned in the previous paragraph, and none characterised by Haskell-style functions.
For this reason, we ignore the tradeoffs associated between the two approaches 
and move directly to literature applicable to typing Scheme-style functions.

\subsection{Variable-Arity Polymorphism}

Strickland et al. invented a type system supporting variable-arity polymorphism\cite{STF09}
a version of which is included in the current implementation of Typed Racket.
Their main innovation centres around \emph{dotted type variables}, which represent a heterogeneous sequence
of types. Dotted type variables allow \emph{non-uniform} variable-arity function types,
which are used to check definitions and usages of functions with non-trivial rest parameters

For example in Clojure, the function \lstinline|map| takes a function and one or more sequences,
and returns the result of applying the function argument to each element of the sequences pair-wise.

\begin{lstlisting}[caption=An application of the non-uniform variable-arity function \lstinline|map|, label=lst:map]
(map + [1 2] [2.1 3.2]) 
;=> (3.1 5.2)
\end{lstlisting}\footnote{Line comments in Clojure begin with \lstinline|;| and comments to the end of the line. We use \lstinline|;=>| to mean \emph{evaluates to}.}

To statically check calls to \lstinline|map|, we must enforce the provided function argument can accept as many
arguments as there are sequence arguments to \lstinline|map|, and the parameter types of the provided function can accept
the pair-wise application of the elements in each sequence. This is a complex relationship between the variable parameters and
the rest of the function.
Listing \ref{lst:map} requires the first argument to \lstinline|map| to be a function of 2 parameters because
there are two sequence parameters. \lstinline|+| takes any number of \lstinline|Number| parameters, 
and applying pair-wise arguments of \lstinline|(Vector Long)| and \lstinline|(Vector Double)| 
results in types \lstinline|Long| and \lstinline|Double| being applied to \lstinline|+|. These are subtypes
of \lstinline|Number|, so the expression is well typed.

\section{Type Inference}

\subsection{Local Type Inference}

Typed Racket uses Local Type Inference \cite{PT00}
as an inference and checking tool. Pierce and Turner
\cite{PT00} divide Local Type Inference into
two complementary algorithms. \emph{Local type argument synthesis}
synthesises type arguments to polymorphic applications, and \emph{bidirectional
propagation} propagates type information both down and up the source tree,
known as \emph{checking} and \emph{synthesis} mode respectively.

\begin{lstlisting}[caption=Bidirectional checking algorithm with Typed Clojure pseudocode, label=lst:bidir]
(map (fn [[a :- Long] [b :- Float]]
       (+ a b))
     [1 2]
     [2.1 3.2])
;=> (3.1 5.2)
\end{lstlisting}

The pseudocode in listing \ref{lst:bidir} show both algorithms in action. Local type argument synthesis is able
to infer the type arguments to \lstinline|map| by observing the argument types of the first
argument to \lstinline|map| and the types of subsequent sequence arguments. Bidirectional checking
then \emph{synthesises} the resulting type of the expression by \emph{checking} each element
of \lstinline|[1 2]| is a subtype of \lstinline|a|, and each element of \lstinline|[2.1 3.2]| is a subtype of
\lstinline|b|. The result of the anonymous function argument is \emph{synthesised} from the type of
\lstinline|(+ a b)| as \lstinline|Float|. We now have sufficient information to 
synthesise the type of listing \ref{lst:bidir} to be \lstinline|(List Float)|.

Pierce and Turner split \emph{local type argument synthesis} into two further
algorithms: bounded, and unbounded quantification \cite{PT00}. 
Typed Racket 
supports unbounded polymorphism \cite{Tob10}, implementing the latter algorithm by Piece et al.
Scala supports bounded quantification with F-bounded polymorphism \cite{CCHOM89},
basing its type argument synthesis on the bounded quantification algorithm.

Pierce and Turner explicitly forbid \cite{PT00}
attempting to synthesise type variables with interdependent bounds, including
F-bounds, having failed to devise an algorithm to infer these cases.
Scala's type argument synthesis implementation deviates from Pierce and Turner and supports these features.
I am not aware of papers specifically describing Scala's modifications, but they are at least inspired by
Scala's spiritual ancestors Generic Java \cite{BOSW98} and Pizza \cite{OW97}.

Hosoya and Pierce \cite{HP99} reiterate two common problems with Local Type Inference:
``hard-to-synthesise arguments'' and ``no best type argument''. The first problem occurs because
both local type argument synthesis and bidirectional propagation cannot perform synthesis
simultaneously. 

\begin{lstlisting}[caption=Hard-to-synthesise expression, label=lst:hts]
(map (fn [a b] 
       (+ a b)) 
     [1 2] 
     [2.1 3.2])
\end{lstlisting}

Listing \ref{lst:hts} shows an example of this limitation,
here caused by both not providing type arguments to \lstinline|map| and not providing the parameter types of \lstinline|(fn [a b] (+ a b))|.
 Cases where both algorithms can simultaneously recover new type information are usually ``hard-to-synthesise''.
``No best type argument'' describes the situation where the results of local
type argument synthesis yield more than one type, and no type is better than the other. Sometimes we cannot recover and synthesis
fails.

\subsection{Colored Local Type Inference}

Scala's type checking uses Colored Local Type Inference \cite{OZZ01},
a variant of Local Type Inference \cite{PT00} specifically designed to
improve inference with certain kinds of Scala pattern matching expressions. It allows
\emph{partial} type information to propagate down the syntax tree, instead of only full type information
as required by Local Type Inference.

\emph{Colored} types contain extra contextual information, including the propagation direction
and missing parts of the type. They are generally useful
for describing ``information flow in polymorphic type systems with propagation-based type inference''
\cite{OZZ01}. If the scope of my research permits it, I plan to investigate
using colored types to solve the polymorphic higher-order-function limitation of Local Type Inference
(see listing \ref{lst:hts}).

\section{Bounded and Unbounded Polymorphism}

Local type inference by Pierce and Turner\cite{PT00}
describe two implementations of type variables, for bounded
and unbounded type variables. The bounded implementation is presented
as an optional extension  to the unbounded implementation, which preserves all
properties described in the Local Type Inference algorithm.

An unbounded type variable does not have subtype constraints.
Bounded type variables can have subtype constraints, and 
subsume unbounded type variables \cite{PT00}, 
as a unbounded variable can be represented as a variable bounded
by the Top type.

Still, unbounded type variables have an advantage: their implementations are
are simpler in the presence of a \emph{Bottom} type. 
The constraint resolution algorithm for bounded variables
is more subtle, due to ``some surprising interactions between bounded quantifiers
and the \emph{Bot} type'' \cite{PT00}, described fully
by Pierce \cite{Pie97}.

Typed Racket \cite{TF08}
supports unbounded polymorphism, while Scala \cite{OCD+}
supports an extended form of bounded polymorphism called
F-bounded polymorphism \cite{CCHOM89}, which allows the
bound variable to occur in its own bound.
F-bounded polymorphism is useful in the context of object-oriented abstractions,
as demonstrated by Odersky \cite{OCD+}.
This is one possible explanation why Typed Racket, which is not built on abstractions like Scala,
does not support bounded quantification. Unfortunately, no Typed Racket paper mentions 
bounded quantification, so the rationale is not clear.

Clojure, like Scala, is built on object-oriented abstractions. Clojure protocols
and Java interfaces (interfaces are supported by Clojure) are good candidates
for bounds in bounded or F-bounded polymorphism.

\section{Typed Racket}

Typed Racket is a statically typed sister language of Racket. It
attempts to preserve existing Racket idioms and aims type check
existing Racket code by simply adding top level type annotations \cite{Tob10}.

Typed Racket fully expands all macro calls before type checking \cite{Tob10} to
avoid the complex semantics of type checking macro definitions, an ongoing research area summarised
 by Herman \cite{Her10}.
The design of my Clojure type system will follow a similar strategy; only the fully macro-expanded form
will be type checked. Type checking macro definitions are outside the scope of this project, and would
be exceptionally hard.

Along with a full static type system, Typed Racket 
also uses runtime contracts to enforce type invariants at runtime \cite{TF08}.
Utilising runtime contracts to aid type checking is outside the scope of this project, but would be 
considered desirable and accessible future work.

Two other Typed Racket features that will be explored are recursive types and refinement types  
\cite{Tob10}. Recursive types allow a type definition to refer to itself, enabling structurally
recursive types like binary trees. Refinement types let the programmer define
new types that are subsets of existing types, such as the type for even integers, a subset of all integers.
Both these features would fit well in a future implementation of this project.

\section{Occurrence Typing}
\label{sec:OccurrenceTyping}

Dynamically typed languages use an ad-hoc combination of type predicates,
selectors, and conditionals to steer execution flow and reason about runtime types of variables.
Typed Racket uses occurrence typing to capture these ad-hoc type refinements.
For example, listing \ref{lst:occ1} shows occurrence typing following the implications 
of the type predicate \emph{number?} and the selector \emph{first}, and utilises those implications to refine
the type of \lstinline|x|. If the test at line 3 succeeds, occurrence typing refines the
type of \lstinline|(first x)| to be \lstinline|Number|, which allows \lstinline|(+ 1 (first x))|
to be well typed. Similarly at line 4, we can be sure that \lstinline|(first x)| is
a \lstinline|String|, since we have ruled out the case of being a \lstinline|Number|.

\begin{lstlisting}[caption=A well typed form utilising occurrence typing with Clojure syntax, label=lst:occ1]
(let [x (list (number-or-string))]
  (cond 
    (number? (first x)) (+ 1 (first x))
    :else               (str (first x))))
\end{lstlisting}

Occurrence typing \cite{TF08}
\cite{TF10} extends the type 
system with a \emph{proposition environment} that represents 
the type refinements inferred down a particular path.
The 2010 paper \cite{TF10}
reformulated occurrence typing, improving the original 2008 paper
\cite{TF08}
after revealing that ``three years of practical experience has revealed
serious shortcoming of our type system.''\cite{TF10}
The new implementation utilizes a simple proof system to solve
propositional logic statements, in terms of type predicates and selectors.

For occurrence typing to infer propositions from type predicate usages, it requires 
two extra annotations: a ``then'' proposition
when the result is a true value, and an ``else'' proposition for a false value.
For example, \lstinline|number?| has a ``then'' proposition that says its argument
is of type \lstinline|Number|, and an ``else'' proposition that says its argument is not of type \lstinline|Number|.

An exciting application of occurrence typing as yet unexplored is facilitating null-safe interop with Java.
By declaring \lstinline|nil| (Clojure's value of Java's \lstinline|null|) to \emph{not} be a subtype of reference types,
we can begin to statically disallow potential inconsistent usages of \lstinline|nil| as part of the type system.

\begin{lstlisting}[caption=Observing nil-checks using occurrence typing, label=lst:nil]
(let [a (ObjectFactory/getObject)]
  (when a
    (expects-non-nil a)))
\end{lstlisting}

Listing \ref{lst:nil} infers from the Java signature \lstinline|Object getObject()| that
that \lstinline|a| is of type \lstinline|(U nil Object)|. This is equivalent to Java's
\lstinline|Object| static type, as \lstinline|null| is a subtype of all reference types. By surrounding
the call \lstinline|(expects-non-nil a)| with \lstinline|(when a ...)|, we guarantee that
\lstinline|a| is non-nil when passed to \lstinline|expects-non-nil|. Occurrence typing infers
this by observing \lstinline|nil| is a false value in Clojure, therefore \lstinline|a| cannot
be \lstinline|nil| the body of the \lstinline|when|, refining \lstinline|a|'s type to \lstinline|Object|
from \lstinline|(Union nil Object)|.

Occurrence typing is a relatively simple, time worn technique used successfully 
in Typed Racket. Clojure is similar enough to Racket for occurrence typing to work
without issues, and has good potential to enable null-safe Java interop.

\section{Statically Typed Multimethods}

Clojure provides multimethods as a core language feature. This section discusses systems that statically
verify type safety for multimethods.

Millstein and Chambers\cite{MS02}
describe Dubious, a simple statically typed core language including multimethods that
dispatch on the type of its arguments. They tackle a key challenge for statically typing
multimethods: ``it is possible for two modules containing arbitrary multimethods to typecheck
successfully in isolation but generate type errors when linked together.''\cite{MS02}

\section{Higher Kinded Programming}

\begin{verbatim}
Many advanced type systems provide support for ..
Forward reference to algo.monad experiment.
\end{verbatim}

\section{Conclusion}

Many related components must come together in the design of a
static type system. Typed Racket achieves a satisfying balance of 
occurrence typing, local type inference and variable-arity polymorphism.
Similarly, Scala features F-bounded polymorphism, a class hierarchy
that is compatible with Java, and colored local type inference.

\chapter{Design Choices}

Typed Clojure is designed to be of practical use to Clojure programmers.
Many of the design goals are similar to Tobin-Hochstadt's \cite{SAMTH:dissertation}
for Typed Racket.

\section{Typed Racket}

The majority of the design and implementation of Typed Clojure is based on Typed Racket.

\subsection{Occurrence Typing}
% interesting paths
% - count path element
% - class path element
% - keyword path element

\subsection{Variable-arity Polymorphism}
% Problematic areas
% - flattened paired arguments. assoc, hash-maps, array-map
% - complex argument dependencies. comp, partial

\section{Safer Host Interoperability}
% Args non-nil, return nilable
% Constructors and other special methods handled appropriately (eg. Constructor are non-nil return)
% TODO refine language Clojure vs. Clojure JVM

% what is this paragraph saying?
Dialects of Clojure provide fast and unrestricted access to their host platform. 
This is a intentional source of incompatibility between dialects.
For example, Clojure is hosted on the Java Virtual Machine (JVM), ClojureCLR on the Common Language Runtime (CLR),
and ClojureScript on Javascript VMs. These are very different platforms, and it is unlikely
a Clojure dot call will be portable. There is no attempt at reconciling
host interoperability differences, and it is up to the programmer to decide
how to best abstract over different hosts.

Typed Clojure targets the JVM hosted Clojure.
Clojure embraces the JVM as a host by sharing its runtime type system and encouraging direct
interaction with libraries written in other JVM languages. Most commonly, Clojure programmers
reach to Java libraries. Java is statically typed language. It follows that any interaction with
Java will already have an annotated Java type. Typed Clojure can use these annotations
to statically type check these interactions.

Typed Clojure attempts to improve some shortcomings of Java's type system.

Java's static type system does not provide a type-safe construct for eliminating
the \lstinline|null| pointer. Instead, Java programmers often rely on either testing
for \lstinline|null| or prior domain knowledge.

\begin{lstlisting}[caption=null elimination in Java]
...
Object a = nextObject();
if (a != null)
  parseNextObject(a);
else
  throw new Exception("null Pointer Found!);
...
\end{lstlisting}



In Java, \lstinline|null| is a subtype to all reference
types, but \lstinline|null| is not expressible as a static type. In other words,
when a Java type claims a reference type, you should also assume it can also be of 
type \lstinline|null|.



Scala provides the Option type for safe \lstinline|null| elimination, forcing pattern matching
where \lstinline|null| might occur.

\begin{lstlisting}[caption=null elimination in Scala]
TODO
\end{lstlisting}

Clojure 

\subsection{Primitive Arrays}
% Primitive arrays are covariant, which is well-known to be statically unsound.
% We cannot trust the type signature of any Java method or field.
% Separate read and write types into two parameters.

\subsection{Interaction with null}
% null is explicitly expressible in my type system
% In Java, null is subtype of all reference types, any reference type must be tested to prevent NPE
% 

\subsection{Generic Java}
% Existential types are needed to fully express Generic Java types in Typed Clojure
% F-bounded polymorphism supported


\section{Optional Type Checking}
\section{Local Type Inference}

Similar to Typed Racket, type inference is based on Local Type Inference
by Pierce and Turner.

\section{F-bounded Polymorphism}

Typed Clojure includes support for F-bounds on type variables, as an extension
to Local Type Inference. 

An extra environment from type variables to bounds is kept until after inference,
after which each unground bound is substituted with the inferred types and the
subtyping relationships are checked.

\section{Heterogeneous Maps}
% keyword keys for now, generally encouraged, fast
% Some heterogeneous maps have string args, eg. from web frameworks

% Operational semantics

\subsection{Operational Semantics}
 
$$
\begin{tdisplay}{Test1}
\begin{altgrammar}
  \s{},\t{} &::= & \int \alt \bool \alt {\proctype {\s{}} {\t{}}} 
  &\mbox{Types}\\
  \M{}, \N{}, \P{}, \dots &::=& \x{} \alt \abs{\x{}}{\M{}} \alt
  \comb{\M1}{\M2}  &\mbox{Raw Terms} \\
\end{altgrammar}
\end{tdisplay} 
$$

 
$$
\begin{tdisplay}{Syntax of Terms}
\begin{altgrammar}
  \exp{} &::=&  \x{} \alt \{ \overrightarrow{\exp{}\ \exp{}} \} 
             \alt [ \overrightarrow{\exp{}} ] 
             &\mbox{Expressions} \\ 
  \v{} &::=& \alt \{ \overrightarrow{\v{}\ \v{}} \} 
             \alt [ \overrightarrow{\v{}} ] 
  &\mbox{Values} \\ 
\end{altgrammar}
\end{tdisplay}
$$
 
$$
\begin{tdisplay}{Operational Semantics}
\begin{array}{c}
{\inferrule[E-Assoc]
  { {\Gamma \vdash e_1 \hookrightarrow v_1} \\
    {\Gamma \vdash e_2 \hookrightarrow v_2} \\
    {\Gamma \vdash e_3 \hookrightarrow v_3} \\
    {v_4 = v_1\ with\ entry\ v_2\ to\ v_3} }
  { {\Gamma \vdash (assoc\ e_1\ e_2\ e_3) \hookrightarrow v_4} }} \\
\\
{\inferrule[E-Dissoc]
  { {\Gamma \vdash e_1 \hookrightarrow v_1} \\
    {\Gamma \vdash e_2 \hookrightarrow v_2} \\
    {v_3 = v_1\ without\ entry\ with\ key\ v_2}}
 {\Gamma \vdash (dissoc\ e_1\ e_2) \hookrightarrow v_3} } \\
\\
{\inferrule[E-Var]
  {\Gamma(x) = v}
  {\Gamma \vdash x \hookrightarrow v}}
\end{array} 
\end{tdisplay} 
$$



\subsection{Higher-order Variable-arity Polymorphism (SKETCH)}

Practical Variable-arity Polymorphism introduces the concept of a dotted type parameter,
representing a sequence of types, allowing non-uniform variable parameters.
This enables \lstinline|map| and other function with complex arguments to be typed.
Strickland, Tobin-Hochstadt, and Felleisen's approach is solves almost all variable-arity applications in
Typed Racket, and is a key feature of Typed Racket's implementation.

Idiomatic variable-arity functions in Clojure's are even more sophisticated than Typed Racket, requiring extensions
to Typed Racket's approach to type check satisfactorily. The Clojure core library favours functions with
variable parameters, especially in a higher-order context. For example, \lstinline|assoc|
associates key-value pairs to a target map. It takes one or more key-value pairs and
applies them left-to-right to the target map. The key problem from a typing perspective
is \lstinline|assoc|'s arguments are flattened, resulting in \lstinline|assoc| only
accepting an even number of rest arguments. \lstinline|assoc| is also commonly used as a
higher-order function.
\lstinline|assoc| cannot be expressed in Typed Racket. For instance,
dotted parameters represent an unrestricted number of types, where we require an even number.

The nature of Clojure's idiomatic variable-arity functions suggest we must introduce
new kinds of dotted parameters. Three main issues must be considered in the context of Typed Clojure.
Firstly, subtyping between different kinds of dotted and rest
\footnote{A rest parameter consumes zero or more types, rather than being a sequence of types.}
parameters can be tricky, only made trickier by adding new kinds of dotted parameters.
Secondly, elimination rules for dotted parameters in many frequently used Clojure core functions
difficult to express.
Thirdly, higher-order functions taking functions with dotted arguments commonly 
The first issue requires careful design and implementation.
The second and third points require new constructs, in the form of \emph{projections}
and abstractions over \emph{dotted} parameters, respectively.





% A syntactic approach to type soundness (Wright and Felleisen)
% operational semantics
% - See Sam TH's operational semantics
% TAPL 392

% potential proofs/rules
% - hmaps
% - intersection types (protocols + types)
% - subtyping
% - dot methods
% - f-bounded polymorphism

\chapter{Implementation}

This chapter discusses the implementation of the Typed Clojure prototype type 
system (we refer to this implementation as \emph{Typed Clojure} for the remainder of this chapter)
in some detail concentrating on significant challenges that were identified and overcome.
Many aspects of Typed Clojure's design follow the implementation of Typed Racket,
which is reflected in this chapter.

\section{Type Checking Procedure}

Typed Clojure and Typed Racket differ significantly in how type checking
is integrated into their programming environments.
Typed Racket is implemented as a language on the
Racket platform, which provides highly sophisticated and extensible macro facilities.
Interestingly, this allows Typed Racket to be entirely implemented with macros.
Instead, Typed Clojure is implemented as a library that utilises abstract syntax trees (AST)
generated by analyze~\cite{Analyze2012}, a library I developed for this project.
This strategy follows common practice for Clojure projects, which favours providing modular libraries 
over modifying the language.

This section goes into details on the implementation of Typed Clojure.
First, a high-level overview is given on the type checking procedure.
Then the interfaces to particular high-level functions are discussed.

\subsection{General Overview}

There are several stages to type checking in Typed Clojure.
Type checking is typically initiated at the read-eval-print-loop prompt (REPL),
for example \lstinline|(check-ns 'my.ns)| checks the namespace \lstinline|my.ns|.
Before type checking begins, all global type definitions in the namespace
are added to the global type environment by compiling the namespace.
These type definitions include

\begin{itemize}
  \item type alias definitions
  \item global variable, protocol, datatype, and Java Class annotations
  \item Java method annotations, such as \lstinline|nilable-param| and \lstinline|non-nil-return|
\end{itemize}

An AST is then generated from the code contained in the target namespace.
This AST is then recursively descended and is type checked using local type inference.
Currently only one error is reported at a time, and type checking stops if a type error
is found.

\subsection{Bidirectional Checking}

The interface to the bidirectional checking algorithm is \lstinline|typed.core/check|,
which takes an expression, represented as an AST generated from \emph{analyze}, and an optional expected type for
the given expression. If the expected type is present, the bidirectional algorithm \emph{checks}
that the expected type matches the actual type of the expression.
If the expected type is omitted, a type is instead \emph{synthesised} for the expression.
The algorithm is based on Pierce and Turner's  Local Type Inference~\cite{PT00}
and the implementation is similar in form to Typed Racket's~\cite{TF08}, in that
one function with an extra ``expected type'' argument is preferred over two complementary
functions, one for checking and one for synthesis.

The main difference between Typed Clojure's and Typed Racket's bidirectional checking
algorithm is the representation for expressions. Typed Racket relies on pre-existing
Racket features like syntax objects for expression representation. Clojure instead
leans towards abstract syntax tree representation, despite its Lisp heritage.
In terms of the bidirectional checking, the difference is mostly cosmetic.


% go into more detail
% - update env
% - what is an environment? Bindings + propositions
% - LOC
% - link to github
% - how occurrence typing works
%   - update env
%   - how to calculate reachability 

\subsection{Occurrence typing}
\label{sec:occurenceimpl}

Typed Clojure's implementation of occurrence typing is ported and extended from Typed Racket.
Occurrence typing plays several roles in Typed Clojure.
First, occurrence typing is used to update the type environment at every conditional branch.
Second, it is used to calculate whether branches are reachable.
Third, paths are used extensively in Typed Clojure. 
The implementation of these features are discussed in this section.

The basic idea of occurrence typing involves keeping a separate environment of \emph{propositions}
that relate bindings to types. These propositions are then used to update the type environment.
In Typed Clojure there are several types of propositions, referred to as \emph{filters}.
They are based on the theory by Tobin-Hochstadt and Felleisen~\cite{TF10}, and are ported directly
from Typed Racket.


\begin{itemize}
  \item TopFilter and BotFilter represent the trivially true and trivially false propositions respectively.
  \item TypeFilter and NotTypeFilter represent a positive or negative association of a binding name
        to a type. The syntax \lstinline|(is t name)| means a proposition that the binding called
        \lstinline|name| is of type \lstinline|t| (corresponding to TypeFilter).
        The syntax \lstinline|(! t name)| means a proposition that the binding called \lstinline|name|
        is \emph{not} of type \lstinline|t| (corresponding to NotTypeFilter).
  \item AndFilter and OrFilter represent logical conjunction of propositions 
    (written \lstinline[mathescape]|(& $\overrightarrow{p}$)|), and
    logical disjunction of propositions
    (written \lstinline[mathescape]{(| $\overrightarrow{p}$)}) for propositions \lstinline|p|.
\end{itemize}

Propositions can optionally carry \emph{path} information
represented by a sequence of \emph{path elements}, which signify which part
of the binding's type to update. For example, Typed Racket uses \emph{car} and \emph{cdr} path elements
to track which component of a cons type to update.
Paths are particularly useful in Typed Clojure. There are path elements for traversing heterogeneous map types (KeyPE),
inferring length information (CountPE), and \lstinline|first| and \lstinline|rest| paths for sequences.
These additions do not appear to introduce any major new complexities related to paths.

\section{Polymorphic Type Inference}

The polymorphic type inference algorithm is directly ported from Typed Racket with slight extensions
for bounded variables,
and is directly based on Pierce and Turner's Local Type Inference~\cite{PT00}.
A common entry point for inferring type variables for polymorphic function invocations
is \lstinline|typed.core/infer|.

\lstinline|infer| is invoked like \lstinline|(infer X Y S T R expected)|, where

\begin{itemize}
  \item \lstinline|X| is a map from type variable names
        to their bounds (representing the type variables in scope),
  \item \lstinline|Y| is a map from type variable names
        to their bounds (representing the \emph{dotted} type variables in scope),
  \item \lstinline|S| and \lstinline|T| are sequences of types of equal length,
        (usually the types of the actual arguments provided and the types of the parameters
        of the polymorphic function),
  \item \lstinline|R| is a result type (usually the return type of the polymorphic function),
  \item \lstinline|expected| is the expected type for R, or the value \lstinline|nil|,
\end{itemize}

and returns a \emph{substitution} that satisfies the following conditions:

\begin{itemize}
  \item Pairwise, each \lstinline|S| is a below \lstinline|T|,
  \item \lstinline|R| is below \lstinline|expected|, if \lstinline|expected| is provided.
\end{itemize}

A substitution maps type variables to types.
It is valid to replace all occurrences of the type variables named in the substitution 
with their associated type.
For example, substitution are often applied to \lstinline|R| by the caller of \lstinline|infer|
to eliminate the type variables in \lstinline|X| and \lstinline|Y|.

\lstinline|infer| is almost always used when invoking polymorphic functions.
\lstinline|clojure.core/constantly| is a polymorphic function with type \lstinline|(All [x y] [x -> [y * -> x]])|
(read as a function taking \lstinline|x| and returning a function that takes any number
of \lstinline|y|'s and returns \lstinline|x|).
For instance, \lstinline|((constantly true) 'any 'number)| results in the value \lstinline|true|.

Type checking the invocation \lstinline|(constantly true)|
calls \lstinline|infer| roughly like

\begin{lstlisting}
(infer {'x no-bounds 'y no-bounds}
       {}
       [(parse-type 'true)]
       [(make-F 'x)]
       (with-frees [(make-F 'x) (make-F 'y)]
         (parse-type '[y * -> x]))
       nil)
\end{lstlisting}

where the internal Typed Clojure bindings

\begin{itemize}
  \item \lstinline|no-bounds| is the type variable bounds with upper bound as \lstinline|Any|
    and lower bound as \lstinline|Nothing|,
  \item \lstinline|parse-type| is a function that takes type \emph{syntax} and converts it to a type (its argument must be quoted), 
  \item \lstinline|make-F| is a function that takes a name symbol and returns a type variable type of that name, and
  \item \lstinline|with-frees| is a macro that brings the type variables named in its first argument into scope in its second argument.
\end{itemize}

This returns a substitution that replaces occurrences of \lstinline|x| with \lstinline|true|
and \lstinline|y| with \lstinline|Any|.
This helps infer a result type of \lstinline|(constantly true)|
as \lstinline|[Any * -> true]|.

\section{F-Bounded Polymorphism}

A feature not present in Typed Racket is bounded polymorphism.
Several changes were needed to support bounded polymorphism. In every position
where a set of type variables was required, it was replaced by a map
of type variables to bounds.

Bounds consist of an upper and lower type bound, or a \emph{kind bound}.
Kind bounds are experimental following the inclusion of user definable type constructors
(motivated in section \ref{sec:monads}).
They are a stub for a more comprehensive treatment of higher-kinded operators
such as that described by Moors, Piessens, and Odersky for Scala~\cite{MPO08}.
At present, a type variable can only be instantiated to a type between
its upper and lower bounds, or, if a kind bound is defined instead, 
to a kind below the kind bound.

F-bounded polymorphism allows type variables to refer to themselves in their type bounds.
Bounds are checked after a substitution is generated, guaranteeing no substitution
can violate type variable bounds. To support F-bounds, the substitution being checked is applied to
the lower and upper bounds for each type variable, and the type associated with the type variable 
in the substitution is checked to be between these bounds.

\section{Variable-arity Polymorphism}

Variable-arity polymorphism in Typed Clojure is directly ported from Typed Racket.
This was the most complicated part of the prototype. At the center of the implementation
is manipulating dotted type variables, which can represent a sequence of types.

It also required changes to the polymorphic type inference, where each
reference to a type variable required a special case for a dotted type variable.
For example, the constraint-generation algorithm for Local Type Inference
features extra kinds of constraints for dotted variables.
Strickland, Tobin-Hochstadt, and Felleisen elaborate on the particular changes
required for the Typed Racket implementation~\cite{STF09}.

Porting Typed Racket's variable-arity polymorphism implementation was
tedious because some of the relevant internal functions interact
in strange ways with the rest of Typed Racket. My reaction was that Typed Racket
was initially designed without variable-arity polymorphism and was added
without major changes to other components. Typed Clojure was
developed with the same design so full variable-arity polymorphism
implementation could be ported without change.

\section{Portability to other Clojure Dialects}

Typed Clojure was built for the Clojure programming language, whose compiler
and data structures are implementated in Java. Clojurescript is the first major Clojure dialect
to be written in Clojure, and it is likely future dialects of Clojure will follow this example. 
Where Clojure uses Java Classes and Interfaces, 
Clojurescript's compiler and data structures are written in terms of Clojure's
two core abstractions: protocols and datatypes.
It would desirable to port Typed Clojure to similar  Clojure dialects while keeping
the core of the implementation constant.

A significant portion of Typed Clojure is theoretically platform independent but there are
challenges to targeting new dialects, including incompatible host interoperability and
non-standardised abstract syntax trees.
The first issue is predictable due to a core philosophy of Clojure dialects: 
host interoperability is non-portable\footnote{See the complete Clojure rationale: http://clojure.org/rationale}.
Each dialect of Clojure has a unique host interoperability story and Typed Clojure should cater for them
separately.
The second issue is potentially resolvable either by enforcing a standard representation for 
abstract syntax trees across Clojure implementations, or developing a library that provided
a common interface to abstract syntax trees for each Clojure implementation.

\section{Proposition Inference for Functions}
\label{sec:filterneg}

A feature not yet implemented in Typed Racket is the ability
to infer new propositions based on existing propositions of a function.
This feature was added to Typed Clojure to support filtering
a sequence based on negative information, such as filtering values
that are \emph{not} \lstinline|nil|.

\begin{lstlisting}[caption=Type annotation for filter, label=lst:filterann]
(ann clojure.core/filter 
  (All [x y]
    [[x -> Any :filters {:then (is y 0)}] (U nil (Seqable x)) -> (Seqable y)]))
\end{lstlisting}

To better understand the problem, listing \ref{lst:filterann} presents the type of \lstinline|filter|,
which takes a function \lstinline|f| and a sequence \lstinline|s| as arguments, and returns a sequence that 
contains each element in \lstinline|s| such that applying \lstinline|f| to the element returns a true value.
The \lstinline|:filters| syntax requires some explanation. 
Function types support an optional \emph{filter set} attached to its return type, written as a map
with \lstinline|:then| and/or \lstinline|:else| keys (if omitted, they default to the trivially true proposition which has no effect). 
The ``then-proposition'' and ``else-proposition'' are added
to the type environment when the return value is a true and false value respectively.
For example, the filter set \lstinline|{:then (is y 0)}| is read ``if the return value is a true value, then the first argument
must be of type y, otherwise if it is a false value, nothing interesting is enforced''
The type given for \lstinline|filter| works because the type variable
\lstinline|y| occurs in both the ``then-proposition'' of the first argument and the return type 
\lstinline|(Seqable y)|.

\begin{lstlisting}[caption=Troublesome filter, label=lst:filtertrouble]
(filter (fn [a] (not (nil? a))) coll)|
\end{lstlisting}

The difficulty starts with something like listing \ref{lst:filtertrouble},
where the inferred filter set for the first argument to \lstinline|filter|
is \lstinline|{:then (! nil 0) :else (is nil 0)}|\footnote{(! nil 0) is the proposition that the first argument is \emph{not}
of type \lstinline|nil|.}. The ``then-proposition'' \lstinline|(! nil 0)| does not fit with
\lstinline|(is y 0)| that \lstinline|filter| expects.

We can sometimes get around this if we already have a predicate with a positive
``then-proposition''. For example, if we are filtering out \lstinline|nil| values
from a sequence of type \lstinline|(Seqable (U Number nil))|, we can replace
listing \ref{lst:filtertrouble} with \lstinline|(filter number? coll)|,
where \lstinline|number?| has the filter set \lstinline|{:then (is Number 0) :else (! Number 0)}|.
This does not work, however, when filtering a sequence of type like \lstinline|(Seqable (U x nil))|
for some unspecified \lstinline|x| because there is no built-in predicate with ``then-proposition''
\lstinline|(is x 0)|.

\begin{lstlisting}[caption=Filtering with negative propositions, label=lst:filtergood]
  (filter (ann-form 
            #(not (nil? %))
            [(U nil x) -> boolean :filters {:then (is x 0)}])
          mvs)
\end{lstlisting}

Instead, we generate new propositions using a technique 
suggested by Tobin-Hochstadt~\cite{Tob12},
that follows from the occurrence typing calculus defined 
by Tobin-Hochstadt and Felleisen~\cite{TF10}.
First the filter set for functions are inferred as usual.
To collect new propositions, the ``then-proposition'' is applied to the type environment (which maps
local bindings to types). Any types associated with bindings that are changed after this
are represented as new propositions, which are added to the ``then-proposition'' for this filter set.
The same procedure is followed for the ``else-proposition''.

Using this technique, the anonymous function in listing 
\ref{lst:filtergood}\footnote{The Clojure syntax \lstinline|#(not (nil? \%))| 
is equivalent to \lstinline|(fn [a] (not (nil? a)))|}
has the filter set \lstinline|{:then (& (! nil 0) (is x 0)) :else (is nil 0)}|
which is good enough to infer the \lstinline|filter|ed result as \lstinline|(Seqable x)|.

Further work in this area is needed when filtering on a non-anonymous function.
For example, it is not clear how to infer the common idiom \lstinline|(filter identity coll)|
as returning a sequence of non-nil elements, for any sequence \lstinline|coll|.
Inferring new propositions for already existing functions like \lstinline|identity|
does not fit with Tobin-Hochstadt and Felleisen's calculus~\cite{TF10},
confirmed by Tobin-Hochstadt~\cite{Tob12} as future work in this area.

%\section{Non-hygienic Macro Expansion and Filters}


% need to remove bindings as they fall out of scope
% Need hygiene to ensure correct propagation of filters
%(fn [a]
%  (if (= a 1)
%    (let [a 'foo] ; here this shadows the argument, impossible to recover filters
%      a)          ; in fact any new filters about a will be incorrectly assumed to be the argument
%      false))


% Evaluation
% - successes
% - future work identified

\chapter{Experiments}

In this chapter we look at several example of using Typed Clojure and
how well the current prototype handles them. 
We intentially chose examples that could be challenging to type check
or were particularly useful to the everyday Clojure programmer.

\section{Java Interoperability}

As Typed Clojure is intended to be useful for practical purposes, it was important
to port existing code that utilized Java interoperability.
My porting effort attempted to follow how a real programmer might port code to Typed Clojure
and I describe my approach for each situation.

I ported a function from clojure.contrib.reflect\footnote{https://github.com/cemerick/pomegranate}, 
a Clojure library relying heavily on Java interoperability.
I chose \lstinline|call-method| for several reasons: it chains several Java calls together,
it uses primitive arrays, it exposes a wart of Clojure with \emph{Named} things,
and \lstinline|null| is a valid value in one place.

Before showing the implementation, there is a brief explanation of the relevant syntax.
Java methods are called using the \emph{dot} operator. If \lstinline|o| is an object \lstinline|(. o m a*)|
calls its method named \lstinline|m| passing \lstinline|a*| arguments. The method can be named first
using the equivalent sugar \lstinline|(.m o a*)|. Also, \lstinline|doto| is convenient 
notation for muliple method calls on the same object, presumably for side effects, and returns the original
object. For example, \lstinline|(doto o (.m1 a*) (m2 a*))| calls methods \lstinline|m1| and \lstinline|m2|
on \lstinline|o| and returns \lstinline|o|.

\begin{lstlisting}[caption=call-method, label=lst:callmethod]
;; call-method pulled from clojure.contrib.reflect, (c) 2010 Stuart Halloway & Contributors
(defn call-method
  "Calls a private or protected method.

  params is a vector of classes which correspond to the arguments to
  the method e

  obj is nil for static methods, the instance object otherwise.

  The method-name is given a symbol or a keyword (something Named)."
  [^Class klass method-name params obj & args]
  (let [method (doto (.getDeclaredMethod klass 
                                         (name method-name)
                                         (into-array Class params))
                 (.setAccessible true))]
    (.invoke method obj (into-array Object args))))
\end{lstlisting}

The original function is modified slightly for readability and is presented in listing \ref{lst:callmethod}.

\begin{lstlisting}[caption=call-method Type Annotation, label=lst:callmethodtype]
(ann call-method 
     [Class Named (IPersistentVector Class) (U nil Object) (U nil Object) * -> (U nil Object)])
\end{lstlisting}

Thankfully this function has up-to-date documentation, and from it we can derive an expected type
(listing \ref{lst:callmethodtype}).

Before running the type checker we must convert our array constructors into ones that Typed Clojure
can understand. Array types in Typed Clojure are represented by \lstinline|(Array c t)|, where
the Java class \lstinline|c| is the Java component type,
and the Typed Clojure type \lstinline|t| is the Typed Clojure component type. 
We can pass this array to Java methods accepting
type \lstinline|c[]|, and we can read and write type \lstinline|t| to the array from Typed Clojure.

\begin{itemize}
\item \lstinline|(into-array Class params)| becomes \lstinline|(into-array> Class Class params)|
      which is of type \lstinline|(Array Class Class)|.
\item \lstinline|(into-array Object args)| becomes \lstinline|(into-array> Object (U nil Object) args)|
      which is of type \lstinline|(Array Object (U nil Object))|.
\end{itemize}

The second place in the \lstinline|Array| type constructor allows fine grained control over what is allowed
in the array. The first point above must be type \lstinline|(Array Class Class)| because the 
\lstinline|getDeclaredMethod| method on \lstinline|java.lang.Class| instances requires an array of non-null
\lstinline|Class| objects. On the other hand, the \lstinline|invoke| method takes an array that allows
\emph{null} members, so its type is \lstinline|(Array Object (U nil Object))|.

Now we run the type checker, which produces a type error.

\begin{lstlisting}
#<Exception java.lang.Exception: 29: Cannot call instance method java.lang.reflect.AccessibleObject/setAccessible on type (U nil java.lang.reflect.Method)>
\end{lstlisting}

Because Typed Clojure assumes all methods return nilable Objects, the call to \lstinline|getDeclaredMethod|
has return type \lstinline|(U nil java.lang.reflect.Method)|. It is not type safe to call \lstinline|setAccessible|
on this type, so we get a type error.

In this case, Typed Clojure is too conservative: according to its documentation \lstinline|getDeclaredMethod|
never returns \emph{null}. We add this rule with \lstinline|non-nil-return|.

\begin{lstlisting}
(non-nil-return java.lang.Class/getDeclaredMethod :all)
\end{lstlisting}

Running the type checker produces a different type error.

\begin{lstlisting}
#<Exception java.lang.Exception: Type Error, REPL:32 - (U java.lang.Object nil) is not a subtype of: java.lang.Object>
\end{lstlisting}

This concerns passing \lstinline|obj| as the first argument to the \lstinline|invoke| method.
Typed Clojure conservatively defaults method parameter types as non-nullable. 
Therefore the first parameter of \lstinline|invoke| is \lstinline|Object| by default; \lstinline|obj|
is \lstinline|(U java.lang.Object nil)|. Again, this is too conservative as the first argument can
be \emph{null} for static methods, and we use \lstinline|nilable-param| to specify the first argument
of \lstinline|invoke| may be nil, for the arity of two parameters.

\begin{lstlisting}
(nilable-param java.lang.reflect.Method/invoke {2 #{0}})
\end{lstlisting}

Here is the final successfully type checked code.
\begin{verbatim}Is this a reasonable result?\end{verbatim}

\begin{lstlisting}
(non-nil-return java.lang.Class/getDeclaredMethod :all)
(nilable-param java.lang.reflect.Method/invoke {2 #{0}})

(ann call-method [Class Named (IPersistentVector Class) (U nil Object) (U nil Object) * -> (U nil Object)])

;; call-method pulled from clojure.contrib.reflect, (c) 2010 Stuart Halloway & Contributors
(defn call-method
  "Calls a private or protected method.

  params is a vector of classes which correspond to the arguments to
  the method e

  obj is nil for static methods, the instance object otherwise.

  The method-name is given a symbol or a keyword (something Named)."
  [^Class klass method-name params obj & args]
  (let [method (doto (.getDeclaredMethod klass 
                                         (name method-name)
                                         (into-array> Class Class params))
                 (.setAccessible true))]
    (.invoke method obj (into-array> Object (U nil Object) args))))
\end{lstlisting}

\section{Red-Black Tree}

\section{Monads}

Monads[] are an interesting control structure probably most recognisable from its inclusion
in the statically-typed language Haskell[].

I chose to port a subset of the Clojure library \emph{algo.monads} to Typed Clojure. The library
provides support for several kinds of monads.

This library represents a monad as a hash-map with four keys: \lstinline|:m-bind|, \lstinline|:m-result|,
\lstinline|:m-zero|, and \lstinline|:m-plus|. A valid monad must provide the first two, and the latter
two may optionally be mapped to the keyword \lstinline|::undefined|\footnote{Keywords prefixed with \lstinline|::|
are qualified in the current namespace.}.

\begin{lstlisting}[caption=Untyped definition for the identity monad, label=lst:identitymdef]
(defmonad identity-m
   "Monad describing plain computations. This monad does in fact nothing
    at all. It is useful for testing, for combination with monad
    transformers, and for code that is parameterized with a monad."
  [m-result identity
   m-bind   (fn m-result-id [mv f]
              (f mv))
  ])
\end{lstlisting}

A monad is defined using the macro \lstinline|defmonad|. The macro expands to code that binds a var to a hash-map
with the aforementioned keys. For example, listing \ref{lst:identitymdef} expands to 
the code in listing \ref{lst:identitymexpand}.

\begin{lstlisting}[caption=Type for identity monad, label=lst:identitymtype]
(ann identity-m
     '{:m-bind (All [x y]
                 [x [x -> y] -> y])
       :m-result (All [x]
                   [x -> x])
       :m-zero '::undefined
       :m-plus '::undefined})
\end{lstlisting}

Typed Clojure operates directly on fully macro-expanded forms, so it is unsurprising that the type
annotation for \lstinline|identity-m| reflects its macro-expansion (listing \ref{lst:identitymtype}).
The \lstinline|:m-bind| entry is comparable to Haskell's monadic \emph{bind},
and the \lstinline|:m-result| entry is comparable to Haskell's monadic \emph{return}.

\begin{lstlisting}[caption=Maybe monad definition, label=lst:maybemdef]
(defmonad maybe-m
  "Monad describing computations with possible failures. Failure is
  represented by nil, any other value is considered valid. As soon as
  a step returns nil, the whole computation will yield nil as well."
  [m-zero   nil
   m-result (fn m-result-maybe [v] 
              v)
   m-bind   (fn m-bind-maybe [mv f]
              (if (nil? mv) nil (f mv)))
   m-plus   (fn m-plus-maybe [& mvs]
              (first 
                (filter #(not (nil? a))) mvs))
   ])
\end{lstlisting}

Instead of explaining the identity monad further, we show a more interesting monad, the maybe monad.
Its untyped definition is given in listing \ref{lst:maybemdef}.

\begin{lstlisting}[caption=Typed maybe monad definition, label=lst:maybemtyped]
(ann maybe-m
    '{:m-bind (All [x y]
                [(U nil x) [x -> (U nil y)] -> (U nil y)])
      :m-result (All [x]
                  [x -> (U nil x)])
      :m-zero nil
      :m-plus (All [x]
                [(U nil x) * -> (U nil x)])})
(defmonad maybe-m
  "Monad describing computations with possible failures. Failure is
  represented by nil, any other value is considered valid. As soon as
  a step returns nil, the whole computation will yield nil as well."
  [m-zero   nil
   m-result (ann-form 
              (fn m-result-maybe [v] v)
              (All [x] 
                [x -> (U nil x)]))
   m-bind   (ann-form 
              (fn m-bind-maybe [mv f]
                (if (nil? mv) nil (f mv)))
              (All [x y]
                [(U nil x) [x -> x] -> (U nil x)]))
   m-plus   (ann-form 
              (fn m-plus-maybe [& mvs]
                (first 
                  (filter
                    (ann-form 
                      #(not (nil? %))
                      [(U nil x) -> boolean :filters {:then (is x 0)}])
                    mvs)))
              (All [x]
                [(U nil x) * -> (U nil x)]))
   ])
\end{lstlisting}



A significant difference between typing Haskell style monads in Haskell and typing monads in \emph{algo.monads} with Typed Clojure
is that monads in Haskell are represented by type classes, while Typed Clojure does not support higher-kinded
constructs.

\section{Conduit}


\chapter{Future Work}

\section{Multimethods}
% Problem - typing isa?
% - probably impossible to type higher order usages of isa?.
% Dispatch function must infer latent filters for multi-arity functions. (Typed Racket only infers single arg I think)
% Statically proving at least one relevant dispatch method is present.
% Warn potential method conflicts.
% Suppress warnings, similar to prefer-method syntax

\section{Records}
% A composition of existing features: Datatypes and Heterogeneous Maps

\subsection{Other Clojure Dialects}
% unify ASTs for easier porting

\subsection{Linter}


%
%\begin{mathpar}
%\inferrule*[right=B-Var]
%  {\rho(x) = v}
%  {\rho \vdash x \downarrow v}
%
%\inferrule*[right=B-Assoc]
%  {\rho \vdash e_m \downarrow mi \\
%   \rho \vdash e_k \downarrow k \\
%   \rho \vdash e_v \downarrow v \\
%   \rho \vdash \overrightarrow{e_j}^n \downarrow \overrightarrow{k_j}^n \\
%   \rho \vdash \overrightarrow{e_m}^n \downarrow \overrightarrow{k_m}^n \\
%   \delta(add\_entry, mi, k, v, \overrightarrow{k_j, k_m}^n) = mo}
%  {\rho \vdash (assoc\ e_m\ e_k\ e_v\ \overrightarrow{e_j\ e_m}^n) \downarrow mo}

%\infrule[B-Var]
%  {\rho(x) = v}
%  {\rho \vdash x \downarrow v}
%
%\infrule[B-Assoc]

%\end{mathpar}

\printbibliography[title=References]

\end{document}
