\chapter{Design Choices}

Typed Clojure is designed to be of practical use to Clojure programmers.
Many of the design goals are similar to Tobin-Hochstadt's \cite{Tob10}
for Typed Racket.

\section{Preserve Idioms}


\section{Typed Racket}

The majority of the design and implementation of Typed Clojure is based on Typed Racket.

\subsection{Occurrence Typing}
% interesting paths
% - count path element
% - class path element
% - keyword path element

\subsection{Variable-arity Polymorphism}
% Problematic areas
% - flattened paired arguments. assoc, hash-maps, array-map
% - complex argument dependencies. comp, partial

\section{Safer Host Interoperability}
% Args non-nil, return nilable
% Constructors and other special methods handled appropriately (eg. Constructor are non-nil return)
% TODO refine language Clojure vs. Clojure JVM

% what is this paragraph saying?
Dialects of Clojure provide fast and unrestricted access to their host platform. 
This is a intentional source of incompatibility between dialects.
For example, Clojure is hosted on the Java Virtual Machine (JVM), ClojureCLR on the Common Language Runtime (CLR),
and ClojureScript on Javascript VMs. These are very different platforms, and it is unlikely
a Clojure dot call will be portable. There is no attempt at reconciling
host interoperability differences, and it is up to the programmer to decide
how to best abstract over different hosts.

Typed Clojure targets the JVM hosted Clojure.
Clojure embraces the JVM as a host by sharing its runtime type system and encouraging direct
interaction with libraries written in other JVM languages. Most commonly, Clojure programmers
reach to Java libraries. Java is statically typed language. It follows that any interaction with
Java will already have an annotated Java type. Typed Clojure can use these annotations
to statically type check these interactions.

Typed Clojure attempts to improve some shortcomings of Java's type system.

Java's static type system does not provide a type-safe construct for eliminating
the \lstinline|null| pointer. Instead, Java programmers often rely on either testing
for \lstinline|null| or prior domain knowledge.

\begin{lstlisting}[caption=null elimination in Java]
...
Object a = nextObject();
if (a != null)
  parseNextObject(a);
else
  throw new Exception("null Pointer Found!);
...
\end{lstlisting}



In Java, \lstinline|null| is a subtype to all reference
types, but \lstinline|null| is not expressible as a static type. In other words,
when a Java type claims a reference type, you should also assume it can also be of 
type \lstinline|null|.



Scala provides the Option type for safe \lstinline|null| elimination, forcing pattern matching
where \lstinline|null| might occur.

\begin{lstlisting}[caption=null elimination in Scala]
TODO
\end{lstlisting}

Clojure 

\subsection{Primitive Arrays}
% Primitive arrays are covariant, which is well-known to be statically unsound.
% We cannot trust the type signature of any Java method or field.
% Separate read and write types into two parameters.

\subsection{Interaction with null}
% null is explicitly expressible in my type system
% In Java, null is subtype of all reference types, any reference type must be tested to prevent NPE
% 

\subsection{Generic Java}
% Existential types are needed to fully express Generic Java types in Typed Clojure
% F-bounded polymorphism supported

\section{Multimethods}

% INTRODUCTION TO CLOJURE MULTIMETHODS

%A multimethod in Clojure differs from most other
%abstractions of the same name. Clojure multimethods
%dispatch on the result of an arbitrary function
%rather than the types or structure of its arguments.
%
%\begin{lstlisting}[caption=Clojure Multimethods, label=lst:mm1]
%(defmulti ToString class)
%
%(defmethod ToString java.lang.Integer
%  [a]
%  (str "Integer: " a))
%
%(defmethod ToString java.lang.Number
%  [a]
%  (str "Number: " a))
%\end{lstlisting}
%
%Even with this extra flexibility, multimethods are often used to dispatch on the type of its arguments.
%Listing \ref{lst:mm1} illustrates a common usage of multimethods: dispatching on the class of the first argument.
%We use \lstinline|defmulti| to define a new multimethod.
%The function \lstinline|class| is used as the dispatch function, a one-argument function returning the class of its argument
%as an instance of \lstinline|java.lang.Class|.
%
%A multimethod invocation redirects control flow to exactly one of its \emph{methods} with unresolved ambiguities resulting in
%a runtime error. The method is selected by first comparing the result of the dispatch function to
%the dispatch values defined by each method. If one or more methods match, multimethods provide
%the ability to prefer one method over another.

\section{Clojure Class Hierarchy}
\section{Protocols and Datatypes}
\section{Optional Type Checking}
\section{Local Type Inference}

Similar to Typed Racket, type inference is based on Local Type Inference
by Pierce and Turner.

\section{F-bounded Polymorphism}

Typed Clojure includes support for F-bounds on type variables, as an extension
to Local Type Inference. 

An extra environment from type variables to bounds is kept until after inference,
after which each unground bound is substituted with the inferred types and the
subtyping relationships are checked.

\section{Heterogeneous Maps}
% keyword keys for now, generally encouraged, fast
% Some heterogeneous maps have string args, eg. from web frameworks

% Operational semantics

\subsection{Operational Semantics}
 
%$$
%\begin{tdisplay}{Test1}
%\begin{altgrammar}
%  \s{},\t{} &::= & \int \alt \bool \alt {\proctype {\s{}} {\t{}}} 
%  &\mbox{Types}\\
%  \M{}, \N{}, \P{}, \dots &::=& \x{} \alt \abs{\x{}}{\M{}} \alt
%  \comb{\M1}{\M2}  &\mbox{Raw Terms} \\
%\end{altgrammar}
%\end{tdisplay} 
%$$

 
$$
\begin{tdisplay}{Syntax of Terms}
\begin{altgrammar}
  \exp{} &::=& (\c{} \overrightarrow{\exp{}})         % function application, multiple arguments
             \alt \{ \overrightarrow{\exp{}\ \exp{}} \} % map literal
             &\mbox{Expressions} \\ 
  \v{} &::=& \k{} \alt \nil{}
              \alt \{ \overrightarrow{\v{}\ \v{}} \}   % map literal
              &\mbox{Values} \\
  \c{} &::=& \assoc{} \alt \dissoc{} \alt \get{}
              &\mbox{Constants}
\end{altgrammar}
\end{tdisplay}
$$

$$
\begin{tdisplay}{Evaluation Contexts}
  \begin{altgrammar}
    \E{} &::=& [ ] % application rules
              \alt (\c{}\ \overrightarrow{\v{}}\ \E{}\ \overrightarrow{\exp{}}) % eval arguments left-to-right
              % map rules
              \alt \{\overrightarrow{\v{}\ \v{}}\ \E{}\ \exp{}\ \overrightarrow{\exp{}\ \exp{}} \} % key first
              \alt \{\overrightarrow{\v{}\ \v{}}\ \v{}\ \E{}\ \overrightarrow{\exp{}\ \exp{}} \}   % value next
              &\mbox{Evaluation Contexts}
  \end{altgrammar}
\end{tdisplay}
$$ 
 
% \mapsto for updating maps
$$
\begin{tdisplay}{Operational Semantics}
\begin{array}{c}
{\inferrule[E-Assoc]
  { {v_1 = \{ \overrightarrow{\v{}\ \v{}} \} } \\
    {v_4 = v_1\ with\ entry\ v_2\ to\ v_3} }
    {{(\assoc{}\ v_1\ v_2\ v_3) \hookrightarrow v_4} }} \\
\\
{\inferrule[E-Dissoc]
  { {v_1 = \{ \overrightarrow{\v{}\ \v{}} \} } \\
    {v_3 = v_1\ without\ entry\ indexed\ by\ v_2} }
  {(dissoc\ v_1\ v_2) \hookrightarrow v_3} } \\
\\
{\inferrule[E-GetMapExist]
  { {v_1 = \{ \overrightarrow{\v{}\ \v{}} \} } \\
    {v_2\ v_3\ in\ v_1} }
%    {v_1\ has\ entry\ with\ key\ v_2} \\
%    {v_3 = corresponding\ value\ in\ entry\ with\ key\ v_2\ in\ v_1}
  { {(get\ v_1\ v_2) \hookrightarrow v_3} }} \\
\\
{\inferrule[E-GetMapNotExist]
  { {v_1 = \{ \overrightarrow{\v{}\ \v{}} \} } \\
    {v_1\ has\ no\ entry\ with\ key\ v_2} \\
    {v_3 = \nil{}} }
  { {(get\ v_1\ v_2) \hookrightarrow v_3} }}
\end{array} 
\end{tdisplay} 
$$

$$
\begin{tdisplay}{Type Syntax}
\begin{altgrammar}
  \T{} &::=& \Top \alt \Bottom \alt \Nil \alt \k{} \alt \{ \overrightarrow{\k{}\ \T{}} \}
             \alt {\IPersistentMap {\T{}} {\T{}}} \alt ( \vee\ \overrightarrow{\T{}} )
             \alt ( \wedge\ \overrightarrow{\T{}} )
\end{altgrammar}
\end{tdisplay}
$$

$$
\begin{tdisplay}{Core Type Rules}
\begin{array}{c}
{\inferrule[T-AssocHMap]
  { {\Gamma \vdash {\hastype {{\exp{}}_{m}} {{\T{}}_{m}}} } \\
    {\Gamma \vdash {\hastype {{\exp{}}_{k}} {{\k{}}}} } \\
    {\Gamma \vdash {\hastype {{\exp{}}_{v}} {{\T{}}_{v}}} } \\
    {\vdash {\issubtype {{\T{}}_{m}} {\TopHMap} }} }
  { {\Gamma \vdash {\hastype {( \assoc{}\ {\exp{}}_{m}\ {\exp{}}_{k}\ {\exp{}}_{v})} 
        {(update\_hmap\ {{\T{}}_{m}}\ {{\k{}}}\ {{\T{}}_{v}})}}}}} \\
\\
{\inferrule[T-AssocPromote]
  { {\Gamma \vdash {\hastype {{\exp{}}_{m}} {{\T{}}_{m}}} } \\
    {\Gamma \vdash {\hastype {{\exp{}}_{k}} {{\T{}}_{k}}} } \\
    {\Gamma \vdash {\hastype {{\exp{}}_{v}} {{\T{}}_{v}}} } \\
    {\vdash {\issubtype {{\T{}}_{m}} {\TopIPersistentMap}}} \\
    {\T{} = (promote\_hmap\ {\T{}}_{m}\ {\T{}}_{k}\ {\T{}}_{v})} }
  { {\Gamma \vdash {\hastype {( \assoc{}\ {\exp{}}_{m}\ {\exp{}}_{k}\ {\exp{}}_{v})} {\T{}}}}}} \\
\end{array}
\end{tdisplay}
$$

$$
\begin{tdisplay}{Type Metafunctions}
\begingroup
\fontsize{10pt}{12pt}
\begin{altgrammar}
  (update\_hmap\ \{ \overrightarrow{{\T{}}_{k}\ {\T{}}_{v}} \}\ {\k{}}_1\ {\T{}}_1) = 
                 \{ \overrightarrow{{\T{}}_{k}\ {\T{}}_{v}}\ {\k{}}_1\ {\T{}}_1 \}\ \\
  (update\_hmap\ (\wedge\ \overrightarrow{{\T{}}})\ {\T{}}_{k}\ {\T{}}_{v}) = 
                               (\wedge\ \overrightarrow{(update\_hmap\ {\T{}}\ {\T{}}_{k}\ {\T{}}_{v})}) \\
  (update\_hmap\ (\vee\ \overrightarrow{{\T{}}})\ {\T{}}_{k}\ {\T{}}_{v}) = 
                               (\vee\ \overrightarrow{(update\_hmap\ {\T{}}\ {\T{}}_{k}\ {\T{}}_{v})})
  \\ \\
  (promote\_hmap\ \{ \overrightarrow{{\T{}}_{k}\ {\T{}}_{v}} \}\ {\T{}}_{kn}\ {\T{}}_{vn})
                                    = {\IPersistentMap {(\vee\ \overrightarrow{{\T{}}_{k}}\ {\T{}}_{kn})}
                                                       {(\vee\ \overrightarrow{{\T{}}_{v}}\ {\T{}}_{vn})}} \\
  (promote\_hmap\ {\IPersistentMap {{\T{}}_{k}} {{\T{}}_{v}}} \ {\T{}}_{kn}\ {\T{}}_{vn})
                                    = {\IPersistentMap {(\vee\ {\T{}}_{k}\ {\T{}}_{kn})}
                                                       {(\vee\ {\T{}}_{v}\ {\T{}}_{vn})}}
\end{altgrammar}
\endgroup
\end{tdisplay}
$$

$$
\begin{tdisplay}{Subtyping}
\begin{array}{c}
\inferrule[S-Refl] { } {\vdash {\issubtype {\T{}} {\T{}}}} \\
\inferrule[S-Any] { } {\vdash {\issubtype {\T{}} {\Top}}}
\inferrule[S-UnionSuper] 
  {\ } 
  {\vdash {\issubtype {\T{}} {\Top}}}
\end{array}
\end{tdisplay}
$$

%\subsection{Higher-order Variable-arity Polymorphism (SKETCH)}
%
%Practical Variable-arity Polymorphism introduces the concept of a dotted type parameter,
%representing a sequence of types, allowing non-uniform variable parameters.
%This enables \lstinline|map| and other function with complex arguments to be typed.
%Strickland, Tobin-Hochstadt, and Felleisen's approach is solves almost all variable-arity applications in
%Typed Racket, and is a key feature of Typed Racket's implementation.
%
%Idiomatic variable-arity functions in Clojure's are even more sophisticated than Typed Racket, requiring extensions
%to Typed Racket's approach to type check satisfactorily. The Clojure core library favours functions with
%variable parameters, especially in a higher-order context. For example, \lstinline|assoc|
%associates key-value pairs to a target map. It takes one or more key-value pairs and
%applies them left-to-right to the target map. The key problem from a typing perspective
%is \lstinline|assoc|'s arguments are flattened, resulting in \lstinline|assoc| only
%accepting an even number of rest arguments. \lstinline|assoc| is also commonly used as a
%higher-order function.
%\lstinline|assoc| cannot be expressed in Typed Racket. For instance,
%dotted parameters represent an unrestricted number of types, where we require an even number.
%
%The nature of Clojure's idiomatic variable-arity functions suggest we must introduce
%new kinds of dotted parameters. Three main issues must be considered in the context of Typed Clojure.
%Firstly, subtyping between different kinds of dotted and rest
%\footnote{A rest parameter consumes zero or more types, rather than being a sequence of types.}
%parameters can be tricky, only made trickier by adding new kinds of dotted parameters.
%Secondly, elimination rules for dotted parameters in many frequently used Clojure core functions
%difficult to express.
%Thirdly, higher-order functions taking functions with dotted arguments commonly 
%The first issue requires careful design and implementation.
%The second and third points require new constructs, in the form of \emph{projections}
%and abstractions over \emph{dotted} parameters, respectively.



