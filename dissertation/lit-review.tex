% TODO
% What does this Lit review cover?
% Why things are included/not included?
% Background
% Soft typing
% See Sam's dissertation
% Lisp type system background (197x)
% Why was it so hard?
% - bidirectional checking vs. global inference

\chapter{Literature Review}

%\begin{abstract}
%Clojure is a practical, dynamically typed language running on the Java Virtual Machine.
%I intend to design and implement a prototype optional static type 
%system for Clojure, with practical use being a priority. Devising a static type system 
%requires a combination of components that each adds some utility
%without significantly complicating the overall implementation.
%I look at several novel inventions created as a result 
%of the Typed Racket project including
%non-uniform variable-arity polymorphism and occurrence typing.
%Typed Racket and Scala both use flavours of local type inference
%which I compare and judge for applicability to this project.
%Other relevant features of Scala are considered, such as its approach to bounded polymorphism,
%the Scala class hierarchy, and the similarities of its methods of composition to Clojure's.
%\end{abstract}

\section{Introduction}

Designing a static type system for any dynamically typed language requires
careful consideration. Idiomatic style in dynamically typed languages is usually
designed to be powerful yet non-intrusive. An effective type system
for such languages should preserve existing idioms and syntactic abstractions
until time tested idioms are devised for the new typed language \cite{SAMTH:dissertation}.

Typed Racket \cite{SAMTH:dissertation} integrates a selection of ideas to
create a typed sister language to Racket.
Both Clojure and Racket are dynamically typed Lisps and unsurprisingly have similar idioms. 
Consequently, a number of novel Typed Racket features are relevant to designing and implementing 
a type system for Clojure.
The inventions of occurrence typing \cite{Tobin-Hochstadt:2010:LTU:1932681.1863561}
and a systematic approach to variable-arity polymorphism
\cite{Strickland:2009:PVP:1532974.1532978}
are both applicable to this project.

Clojure provides two dispatching mechanisms: multimethods, a mechanism to dispatch on the
results of an arbitrary function; and protocols, a mechanism for single dispatch.
Millstien and Chambers explore statically typing multiple dispatched multimethods in Dubious \cite{Millstein02modularstatically}.
Dubious' multimethods are less powerful than Clojure's multimethods but more powerful than protocols,
and Clojure's dispatch functions can be extended at runtime, unlike Dubious' multimethods.

Clojure and the statically typed language Scala \cite{Odersky06anoverview} 
have some interesting similarities. The recently redesigned 
Scala collection hierarchy
\cite{Odersky06anoverview} \cite{Scala:Collections}
bears similarity to Clojure's encompassing ``seq''
model, revealing an opportunity to test potential typed extensions to the Clojure 
class hierarchy. 

Scala and Clojure are both languages with a focus on host interoperability, specifically
to the Java Virtual Machine. Their differing treatment of the Java Virtual Machine's \lstinline|null| is
relevant, particularly considering the potential of using occurrence typing \cite{Tobin-Hochstadt:2010:LTU:1932681.1863561}
to prevent erroneous usage of \lstinline|null|.

\section{Variable-Arity Polymorphism}

Functions with variable-arity are abundant in dynamically typed languages;
it is common to have a large standard library of functions that utilize
variable-arity, and Clojure is no different. 
Support for complex variable-arity functions is sparse in typed languages,
Strickland claims \cite{Strickland:2009:PVP:1532974.1532978}
``no typed language supports them in a systematic and principled manner''.

Strickland et al. invented \cite{Strickland:2009:PVP:1532974.1532978} 
a calculus capable of typing non-uniform variable-arity functions, a version of which
is implemented in Typed Racket. A non-uniform variable-arity function is a function that
takes a heterogeneous variable parameter list, which usually has a complex relationship 
with other components of the function.
For example, Clojure's \lstinline|map| function takes a function and one or more sequences,
and returns the result of applying the function argument to each element of the sequences pair-wise.

\begin{lstlisting}[caption=An application of the non-uniform variable-arity function \lstinline|map|, label=lst:map]
(map + [1 2] [2.1 3.2]) 
;=> (3.1 5.2)
\end{lstlisting}

To statically check calls to \lstinline|map|, we must enforce the provided function argument can accept as many
arguments as there are sequence arguments to \lstinline|map|, and the parameter types of the provided function can accept
the pair-wise application of the elements in each sequence. This is a complex relationship between the variable parameters and
the rest of the function.
Listing \ref{lst:map} requires the first argument to \lstinline|map| to be a function of 2 parameters because
there are two sequence parameters. \lstinline|+| takes any number of \lstinline|Number| parameters, 
and applying pair-wise arguments of \lstinline|(Vector Long)| and \lstinline|(Vector Double)| 
results in types \lstinline|Long| and \lstinline|Double| being applied to \lstinline|+|. These are subtypes
of \lstinline|Number|, so the expression is well typed.

Clojure presents some opportunities to extend Strickland's calculus. It is idiomatic
to utilize hash-maps as ad-hoc objects in Clojure, and it would be useful to track the keys associated
with a hash-map.  The fundamental hash-map manipulation functions \lstinline|assoc| and \lstinline|dissoc|,
which associate and dissociate entries to (immutable) hash-maps, both take a variable number of arguments.

\begin{lstlisting}[caption=Using \lstinline|dissoc| with variable arguments, label=lst:assoc1]
(dissoc {:a 1, :b 2} ; => {:a 1, :b 2}
        :a           ; => {:b 2}
        :b}          ; => {}
;=> {}
\end{lstlisting}

One approach to typing \lstinline|dissoc| could be to include in every hash-map type how to dissociate exactly one
key from itself, and then treat subsequent entries passed to \lstinline|dissoc| individually in a similar manner.
This is a complex polymorphic relationship between the parameter and result types useful for typing
several core Clojure functions.

\section{Type Inference}

\subsection{Local Type Inference}

Typed Racket uses Local Type Inference \cite{Pierce:2000:LTI:345099.345100}
as an inference and checking tool. Pierce and Turner
\cite{Pierce:2000:LTI:345099.345100} divide Local Type Inference into
two complementary algorithms. \emph{Local type argument synthesis}
synthesises type arguments to polymorphic applications, and \emph{bidirectional
propagation} propagates type information both down and up the source tree,
known as \emph{checking} and \emph{synthesis} mode respectively.

\begin{lstlisting}[caption=Bidirectional checking algorithm with Typed Clojure pseudocode, label=lst:bidir]
(map (fn [a :- Long, b :- Float]
       (+ a b))
     [1 2]
     [2.1 3.2])
;=> (3.1 5.2)
\end{lstlisting}

The pseudocode in listing \ref{lst:bidir} show both algorithms in action. Local type argument synthesis is able
to infer the type arguments to \lstinline|map| by observing the argument types of the first
argument to \lstinline|map| and the types of subsequent sequence arguments. Bidirectional checking
then \emph{synthesises} the resulting type of the expression by \emph{checking} each element
of \lstinline|[1 2]| is a subtype of \lstinline|a|, and each element of \lstinline|[2.1 3.2]| is a subtype of
\lstinline|b|. The result of the anonymous function argument is \emph{synthesised} from the type of
\lstinline|(+ a b)| as \lstinline|Float|. We now have sufficient information to 
synthesise the type of listing \ref{lst:bidir} to be \lstinline|(List Float)|.

Pierce and Turner split \emph{local type argument synthesis} into two further
algorithms: bounded, and unbounded quantification \cite{Pierce:2000:LTI:345099.345100}. 
Typed Racket 
supports unbounded polymorphism \cite{SAMTH:dissertation}, implementing the latter algorithm by Piece et al.
Scala supports bounded quantification with F-bounded polymorphism \cite{Canning:1989:FPO:99370.99392},
basing its type argument synthesis on the bounded quantification algorithm.

Pierce and Turner explicitly forbid \cite{Pierce:2000:LTI:345099.345100}
attempting to synthesise type variables with interdependent bounds, including
F-bounds, having failed to devise an algorithm to infer these cases.
Scala's type argument synthesis implementation deviates from Pierce and Turner and supports these features.
I am not aware of papers specifically describing Scala's modifications, but they are at least inspired by
Scala's spiritual ancestors Generic Java \cite{Bracha98gj:extending} and Pizza \cite{Odersky97pizzainto}.

Hosoya and Pierce \cite{Hosoya99howgood} reiterate two common problems with Local Type Inference:
``hard-to-synthesise arguments'' and ``no best type argument''. The first problem occurs because
both local type argument synthesis and bidirectional propagation cannot perform synthesis
simultaneously. 

\begin{lstlisting}[caption=Hard-to-synthesise expression, label=lst:hts]
(map (fn [a b] 
       (+ a b)) 
     [1 2] 
     [2.1 3.2])
\end{lstlisting}

Listing \ref{lst:hts} shows an example of this limitation,
here caused by both not providing type arguments to \lstinline|map| and not providing the parameter types of \lstinline|(fn [a b] (+ a b))|.
 Cases where both algorithms can simultaneously recover new type information are usually ``hard-to-synthesise''.
``No best type argument'' describes the situation where the results of local
type argument synthesis yield more than one type, and no type is better than the other. Sometimes we cannot recover and synthesis
fails.

\subsection{Colored Local Type Inference}

Scala's type checking uses Colored Local Type Inference \cite{Odersky:2001:CLT:373243.360207},
a variant of Local Type Inference \cite{Pierce:2000:LTI:345099.345100} specifically designed to
improve inference with certain kinds of Scala pattern matching expressions. It allows
\emph{partial} type information to propagate down the syntax tree, instead of only full type information
as required by Local Type Inference.

\emph{Colored} types contain extra contextual information, including the propagation direction
and missing parts of the type. They are generally useful
for describing ``information flow in polymorphic type systems with propagation-based type inference''
\cite{Odersky:2001:CLT:373243.360207}. If the scope of my research permits it, I plan to investigate
using colored types to solve the polymorphic higher-order-function limitation of Local Type Inference
(see listing \ref{lst:hts}).

\section{Bounded and Unbounded Polymorphism}

The Local Type Inference paper \cite{Pierce:2000:LTI:345099.345100}
describes two implementations of type variables, for bounded
and unbounded type variables. The bounded implementation is presented
as an optional extension  to the unbounded implementation, which preserves all
properties described in the Local Type Inference algorithm.

An unbounded type variable does not have subtype constraints.
Bounded type variables can have subtype constraints, and 
subsume unbounded type variables \cite{Pierce:2000:LTI:345099.345100}, 
as a unbounded variable can be represented as a variable bounded
by the Top type.

Still, unbounded type variables have an advantage: their implementations are
are simpler in the presence of a \emph{Bottom} type. 
The constraint resolution algorithm for bounded variables
is more subtle, due to ``some surprising interactions between bounded quantifiers
and the \emph{Bot} type'' \cite{Pierce:2000:LTI:345099.345100}, described fully
by Pierce \cite{Pierce97boundedquantification}.

Typed Racket \cite{Tobin-Hochstadt:2008:DIT:1328897.1328486}
supports unbounded polymorphism, while Scala \cite{Odersky06anoverview}
supports an extended form of bounded polymorphism called
F-bounded polymorphism \cite{Canning:1989:FPO:99370.99392}, which allows the
bound variable to occur in its own bound.
F-bounded polymorphism is useful in the context of object-oriented abstractions,
as demonstrated by Odersky \cite{Odersky06anoverview}.
This is one possible explanation why Typed Racket, which is not built on abstractions like Scala,
does not support bounded quantification. Unfortunately, no Typed Racket paper mentions 
bounded quantification, so the rationale is not clear.

Clojure, like Scala, is built on object-oriented abstractions. Clojure protocols
and Java interfaces (interfaces are supported by Clojure) are good candidates
for bounds in bounded or F-bounded polymorphism.

\section{Typed Racket}

Typed Racket is a statically typed sister language of Racket. It
attempts to preserve existing Racket idioms and aims type check
existing Racket code by simply adding top level type annotations \cite{SAMTH:dissertation}.

Typed Racket fully expands all macro calls before type checking \cite{SAMTH:dissertation} to
avoid the complex semantics of type checking macro definitions, an ongoing research area summarised
 by Herman \cite{Herman10:Theory}.
The design of my Clojure type system will follow a similar strategy; only the fully macro-expanded form
will be type checked. Type checking macro definitions are outside the scope of this project, and would
be exceptionally hard.

Along with a full static type system, Typed Racket 
also uses runtime contracts to enforce type invariants at runtime \cite{Tobin-Hochstadt:2008:DIT:1328897.1328486}.
Utilising runtime contracts to aid type checking is outside the scope of this project, but would be 
considered desirable and accessible future work.

Two other Typed Racket features that will be explored are recursive types and refinement types  
\cite{SAMTH:dissertation}. Recursive types allow a type definition to refer to itself, enabling structurally
recursive types like binary trees. Refinement types let the programmer define
new types that are subsets of existing types, such as the type for even integers, a subset of all integers.
Both these features would fit well in a future implementation of this project.

\section{Occurrence Typing}
\label{sec:OccurrenceTyping}

Dynamically typed languages use an ad-hoc combination of type predicates,
selectors, and conditionals to steer execution flow and reason about runtime types of variables.
Typed Racket uses occurrence typing to capture these ad-hoc type refinements.
For example, listing \ref{lst:occ1} shows occurrence typing following the implications 
of the type predicate \emph{number?} and the selector \emph{first}, and utilises those implications to refine
the type of \lstinline|x|. If the test at line 3 succeeds, occurrence typing refines the
type of \lstinline|(first x)| to be \lstinline|Number|, which allows \lstinline|(+ 1 (first x))|
to be well typed. Similarly at line 4, we can be sure that \lstinline|(first x)| is
a \lstinline|String|, since we have ruled out the case of being a \lstinline|Number|.

\begin{lstlisting}[caption=A well typed form utilising occurrence typing with Clojure syntax, label=lst:occ1]
(let [x (list (number-or-string))]
  (cond 
    (number? (first x)) (+ 1 (first x))
    :else               (str (first x))))
\end{lstlisting}

Occurrence typing \cite{Tobin-Hochstadt:2008:DIT:1328897.1328486}
\cite{Tobin-Hochstadt:2010:LTU:1932681.1863561} extends the type 
system with a \emph{proposition environment} that represents 
the type refinements inferred down a particular path.
The 2010 paper \cite{Tobin-Hochstadt:2010:LTU:1932681.1863561}
reformulated occurrence typing, improving the original 2008 paper
\cite{Tobin-Hochstadt:2008:DIT:1328897.1328486}
after revealing that ``three years of practical experience has revealed
serious shortcoming of our type system.''\cite{Tobin-Hochstadt:2010:LTU:1932681.1863561}
The new implementation utilizes a simple proof system to solve
propositional logic statements, in terms of type predicates and selectors.

For occurrence typing to infer propositions from type predicate usages, it requires 
two extra annotations: a ``then'' proposition
when the result is a true value, and an ``else'' proposition for a false value.
For example, \lstinline|number?| has a ``then'' proposition that says its argument
is of type \lstinline|Number|, and an ``else'' proposition that says its argument is not of type \lstinline|Number|.

An exciting application of occurrence typing as yet unexplored is facilitating null-safe interop with Java.
By declaring \lstinline|nil| (Clojure's value of Java's \lstinline|null|) to \emph{not} be a subtype of reference types,
we can begin to statically disallow potential inconsistent usages of \lstinline|nil| as part of the type system.

\begin{lstlisting}[caption=Observing nil-checks using occurrence typing, label=lst:nil]
(let [a (ObjectFactory/getObject)]
  (when a
    (expects-non-nil a)))
\end{lstlisting}

Listing \ref{lst:nil} infers from the Java signature \lstinline|Object getObject()| that
that \lstinline|a| is of type \lstinline|(Union nil Object)|. This is equivalent to Java's
\lstinline|Object| class, as \lstinline|null| is a subtype of all reference types. By surrounding
the call \lstinline|(expects-non-nil a)| with \lstinline|(when a ...)|, we guarantee that
\lstinline|a| is non-nil when passed to \lstinline|expects-non-nil|. Occurrence typing infers
this by observing \lstinline|nil| is a false value in Clojure, therefore \lstinline|a| cannot
be \lstinline|nil| the body of the \lstinline|when|, refining \lstinline|a|'s type to \lstinline|Object|
from \lstinline|(Union nil Object)|.

Occurrence typing is a relatively simple, time worn technique used successfully 
in Typed Racket. Clojure is similar enough to Racket for occurrence typing to work
without issues, and has good potential to enable null-safe Java interop.

\section{Statically Typed Multimethods}

A Clojure multimethod is slightly different to most other
abstractions of the same name. Here, a multimethod 
dispatches on the result of an \emph{arbitrary function},
rather that more traditional multimethods that dispatch on
the types or structure of its arguments.

Millstein and Chambers \cite{Millstein02modularstatically}
describe Dubious, a simple statically typed core language including multimethods that
dispatch on the type of its arguments. They tackle a key challenge for statically typing
multimethods: ``it is possible for two modules containing arbitrary multimethods to typecheck
successfully in isolation but generate type errors when linked together.'' \cite{Millstein02modularstatically}

This challenge holds when statically typing Clojure multimethods.
One additional challenge would be multimethods being imported from 
untyped modules. This would have to be handled carefully.

If we consider a multimethod as a spread out conditional, a
novel use of occurrence typing becomes apparent. We can use
occurrence typing to refine the argument types in the method bodies
by utilizing the results of the dispatch function. This is simple
when dispatching by type, but a common idiom in Clojure is to dispatch
on the value associated with a key in a hash-map. 

To satisfy this idiom, we require occurrence typing with more
finely grained hash-map types (see section~\ref{sec:OccurrenceTyping} for discussion).

\section{Conclusion}

Many related components must come together in the design of a
static type system. Typed Racket achieves a satisfying balance of 
occurrence typing, local type inference and variable-arity polymorphism.
Similarly, Scala features F-bounded polymorphism, a class hierarchy
that is compatible with Java, and colored local type inference.

I intend to design a static type system for Clojure
that features a satisfying combination of features, individually taken from other
languages, when combined produce novel and interesting problems.
I have looked at the limitations of Local Type Inference and identified Colored Local
Type Inference as potentially improving them.
I have proposed a novel use of occurrence typing to prevent erroneous use of Java's
null pointer, and identified a novel problem typing certain Clojure functions with variable
arguments.
There are some interesting problems and novel solutions to be discovered while undertaking this
project, many of which will reveal themselves during development.
