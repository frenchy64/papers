\chapter{Future Work}

\section{Paired Arguments}

It is not clear how to handle several key Clojure functions
that accept an even number of arguments.
For example, valid usages of \lstinline|assoc| are
\lstinline[mathescape]{(assoc m k v $\overrightarrow{k v}$)},
which takes three arguments and any number of paired arguments.

Strickland, Tobin-Hochstadt, and Felleisen's calculus~\cite{STF09}
is insufficient to express this pattern.
Devising and integrating types that can express this pattern 
is set as future work.

\section{Contracts and Blame}
\label{sec:contractsblame}

A key part of Typed Racket~\cite{Tob10} is its contract and blame system.
They enable safe interoperability with untyped Racket by generating runtime
contracts in key places based on static types. Typed Racket also 
includes a sophisticated blame system based on Wadler\cite{WF09}
to provide more accurate error reporting that ensures typed modules
``can't be blamed''.

There are two kinds of interoperability in Typed Clojure
that could utilize these features. First, the set of ``assumptions''
given by the programmer about Java interoperability like those introduced by \lstinline|non-nil-return| 
and \lstinline|nilable-param| (see section \ref{sec:experimentinterop})
could be checked by adding runtime contracts.
Secondly, when typed namespaces import untyped functions, the static type
assigned to the untyped function could be enforced by wrapping the
function in a runtime contract. Typed Racket uses this second approach
when importing untyped Racket functions, and it results in better error messages.

\section{Multimethods}
\label{sec:mmfuture}

Multimethods in Clojure are an idiomatic and often used feature.
It would be essential for an optional static type system for Clojure to support
The major hurdle for type checking multimethod definitions is the
interaction between the multimethod dispatch mechanism and occurrence typing.

\lstinline|clojure.core/isa?| is the core of multimethod dispatch.
It is a function that takes two arguments and is \lstinline|true| if
the first argument is a member of the second argument, otherwise \lstinline|false|.
It gets quite complicated quickly even with common usages of \lstinline|isa?|.
Simple calls involving only \emph{Class} objects return true if the left class
includes the right class in a Java type relationship, eg. \lstinline|(isa? Integer Number)| is \lstinline|true|.
For example, \lstinline|(isa? [Integer Double] [Number Number])| is \lstinline|true|
because the left and right arguments are vectors of classes of the same length
that are subtypes by \lstinline|isa?| pairwise.

It is also unclear whether it is feasible to perform comprehensiveness and
ambiguity prediction tests on multimethod usages. I predict that they will be
very difficult to pull off satisfactorily because of the ``openess'' of multimethods,
and we will fall back on the runtime errors that multimethods already throw.

\section{Records}

A Clojure record is a composite of datatypes and maps; it is a datatype
with fields that can be treated like a map: records support map operations like \lstinline|get| (lookup),
\lstinline|assoc| (add entries), and \lstinline|dissoc| (remove entries).
One interesting property from the perspective of static typing is that if
a entry corresponding to a field of the underlying datatype is removed
with \lstinline|dissoc|, then the resulting type is a plain map: it loses
its record type.
Correspondingly, a record keeps its type if any other entry is removed.

Records are used frequently in Clojure code, so it is desirable
to support them to some degree. Future work is planned to investigate solutions
to satisfactorily capture their subtle semantics in a practical way.

\subsection{Porting to other Clojure implementations}

It would be desirable to port Typed Clojure to other Clojure implementations.
Each implementation would bring its own set of challenges with interoperability.
The core of Typed Clojure would preserved with each port changing at least 
its interoperability with its host platform. Section \label{sec:portability}
discusses potential problems with the porting process, and how
they might be solved.

%\subsection{Extensions to heterogeneous map types}

%\subsection{Java Generics}

%\subsection{Linter}
