
\documentclass[12pt,a4paper]{article}
%\usepackage{graphicx}
%\usepackage{cite}
%\usepackage{latexsym}
\usepackage{biblatex}

\bibliography{bibliography}

\usepackage{color}
\usepackage{listings}
\lstset{ %
  language=Lisp,                % choose the language of the code
  columns=fixed,basewidth=.5em,
  basicstyle=\small\ttfamily,       % the size of the fonts that are used for the code
  numbers=left,                   % where to put the line-numbers
  %numberstyle=\small\ttfamily,      % the size of the fonts that are used for the line-numbers
  %stepnumber=1,                   % the step between two line-numbers. If it is 1 each line will be numbered
  %numbersep=5pt,                  % how far the line-numbers are from the code
  %backgroundcolor=\color{white},  % choose the background color. You must add \usepackage{color}
  %showspaces=false,               % show spaces adding particular underscores
  %showstringspaces=false,         % underline spaces within strings
  %showtabs=false,                 % show tabs within strings adding particular underscores
  frame=single,           % adds a frame around the code
  %tabsize=2,          % sets default tabsize to 2 spaces
  captionpos=b,           % sets the caption-position to bottom
  breaklines=true,        % sets automatic line breaking
  breakatwhitespace=true,    % sets if automatic breaks should only happen at whitespace
  %escapeinside={\%*}{*)},          % if you want to add a comment within your code
}


\title{Type Inference for a Clojure Type System}
\author{Ambrose Bonnaire-Sergeant}

\begin{document}

\maketitle

\begin{abstract}
  
  Type inference empowers programmers to work faster and safer in a statically 
  typed world. The popularity of language idioms is strongly linked
  with type inference: programmers won't use hard-to-express idioms.
  Designing a type system for a dynamic language should preserve existing idioms, 
  and this crucially relies on the design of type inference.
  I explore and evaluate the type inference strategies
  used by Typed Racket and Scala, showing their relevance to a type system for Clojure
  (an existing dynamically typed language).

\end{abstract}

{\bf Keywords:} occurrence typing, local type inference, Typed Racket, Scala, Clojure

\section{Introduction}

Statically typed languages know the types of their programs
at compile time: but how? Explicitly typed languages
put most responsibility on the programmer
to provide type annotations; implicitly typed languages attempt to
recover type information using various type inference strategies. 
A good type inference algorithm can minimise and sometimes completely
eliminate manual type annotations.

Clojure is a dynamically typed language on the Java Virtual Machine.
In designing a type system for Clojure, I explore two existing statically typed
languages: Scala and Typed Racket.
I conclude that several type inference techniques used by Typed Racket
are applicable to a Clojure type system.

At the core of Typed Racket's type inference strategy is Local Type Inference
\cite{Pierce:2000:LTI:345099.345100}.
Local Type Inference requires all top level bindings to be fully annotated,
but promises to almost never require type annotations to local bindings.
Hosoya and Pierce detail the limitations of Local Type Inference,
and note little success in improving them\cite{Hosoya99howgood}.

The Scala type inference strategy utilises Colored Local Type Inference
\cite{Odersky:2001:CLT:373243.360207}, an extension of Pierce and 
Turner's Local Type Inference
\cite{Pierce:2000:LTI:345099.345100}.
Colored Local Type Inference improves Pierce and Turner's algorithm
by allowing partial type information to propagate downwards, where
the original algorithm only allowed the propagation of complete type information.

Typed Racket's type inference is helped by occurrence typing \cite{Tobin-Hochstadt:2010:LTU:1932681.1863561}
, which reasons about predicates directing program flow. Its
goal is to understand dynamic language idioms, such that they can
be preserved and type checked. Occurrence typing is crucial to the
usability of Typed Racket, and it could fit well into a Clojure
type system.

To test the applicability of Typed Racket's type inference strategies,
I port a function provided by Clojure's standard library
to Typed Racket. The function uses multiple type predicates
combined in either disjunction and conjunction, and expects two very general
parameters. This tests Typed Racket's occurrence typing implementation.
The results are quite promising, with the basic structure of the code remaining unchanged.

I also verify some known limitations of Local Type Inference 
\cite{Pierce:2000:LTI:345099.345100}\cite{Hosoya99howgood}
by type checking code known to fail type synthesis. I show the specified cases fail as expected
and then I provide correct type annotations to type check the cases.

From these experiments, I conclude that occurrence typing is
a good candidate for a Clojure type system. It allows idiomatic code to be
typed without structural modifications. I also conclude that Local Type Inference 
\cite{Pierce:2000:LTI:345099.345100}
is more appropriate than Colored Local Type Inference \cite{Odersky:2001:CLT:373243.360207}, since the benefits of the
latter are limited outside Scala, but I note that Colored Local Type Inference
would be a good starting point to improve Local Type Inference.

\section{Previous and related work}

One goal of Typed Racket \cite{Tobin-Hochstadt:2008:DIT:1328897.1328486} is 
adding a satisfying type system to an existing, dynamic, 
language. This is exactly the use-case I explore with a Clojure type system.
The particular mix of type checking features used by Typed Racket harmonise to satisfy the 
goal of type checking idiomatic Racket code.

\subsection{Local Type Inference}

Typed Racket uses Local Type Inference \cite{Pierce:2000:LTI:345099.345100}
as an inference tool and checking tool. Pierce and Turner
\cite{Pierce:2000:LTI:345099.345100} divide Local Type Inference into
two complementary algorithms. \emph{Local type argument synthesis}
synthesises type arguments to polymorphic applications, and \emph{bidirectional
propagation} propagates type information to nodes in both directions of the source tree.

Pierce and Turner split \emph{local type argument synthesis} into two further
algorithms: bounded, and unbounded quantification \cite{Pierce:2000:LTI:345099.345100}. 
Typed Racket 
supports unbounded polymorphism \cite{SAMTH:dissertation}, implementing the latter algorithm by Piece et al. .
Scala supports bounded quantification with F-bounded polymorphism \cite{Canning:1989:FPO:99370.99392},
basing its type argument synthesis on the bounded quantification algorithm.

Pierce and Turner explicitly forbid \cite{Pierce:2000:LTI:345099.345100}
type variables with interdependent bounds, including
F-bounds, having failed to devise an algorithm to infer these cases.
Scala's type argument synthesis implementation deviates from Pierce and Turner to support these features.
I am not aware of papers specifically describing Scala's modifications, but they are at least inspired by
Scala's spiritual ancestors Generic Java \cite{Bracha98gj:extending} and Pizza \cite{Odersky97pizzainto}.

Hosoya and Pierce \cite{Hosoya99howgood} reiterate two common problems with Local Type Inference:
``hard-to-synthesise arguments'' and ``no best type argument''. The first problem occurs because
both local type argument synthesis and bidirectional propagation cannot perform synthesis
simultaneously. Cases where both algorithms can recover new type information are usually ``hard-to-synthesise''.
``No best type argument'' describes the situation where the results of local
type argument synthesis yield more than one type, and no type is better than the other. Sometimes we cannot recover and synthesis
fails.

\subsection{Colored Local Type Inference}

Scala's type checking uses Colored Local Type Inference \cite{Odersky:2001:CLT:373243.360207},
a variant of Local Type Inference \cite{Pierce:2000:LTI:345099.345100} specifically designed to
improve inference with certain kinds of pattern matching expressions in Scala. It allows
\emph{partial} type information to propagate down the syntax tree, instead of only full type information
as required by Local Type Inference.

\emph{Colored} types contain extra information than normal types, including the propagation direction
and missing parts of the type (when representing a \emph{partial type}). They are generally useful
for describing ``information flow in polymorphic type systems with propagation-based type inference''
\cite{Odersky:2001:CLT:373243.360207}. If the scope of my research permits it, I plan to investigate
using colored types to solve the polymorphic higher-order-function limitation of Local Type Inference.

\subsection{Occurrence Typing}

Occurrence typing was devised \cite{Tobin-Hochstadt:2008:DIT:1328897.1328486} and improved
\cite{Tobin-Hochstadt:2010:LTU:1932681.1863561} for the Typed Racket project. It uses propositional 
logic to combine type information gathered by a combination of type predicates and path selectors.
It enables recovery of type information in an ad-hoc fashion, and gives the type checker some insight
in the invariants at given points in the program.

Crucially, occurrence typing allows us to continue writing code in a dynamic style, without restructuring
our functions for the sake of type checking. 

\subsection{A Clojure Type System}

Clojure is similar to both Scala and Typed Racket in a few ways. Scala and Clojure
both run on the JVM, and have extensive object-oriented standard libraries to utilise.
As is common in statically typed object-oriented languages, bounded quantification is
included, with the extension of F-bounded polymorphism.

The parallels with Typed Racket are more fruitful in a type checking context.
Occurrence typing is a good, lightweight solution to the problem faced by Typed Racket:
type check idiomatic and existing Racket code. Clojure and Racket are both Lisps, so even
visually there is a close relationship between them. It seems safe to assume occurrence
typing will bring as many benefits to a Clojure type system as it did to Typed Racket.

\section{Methodology}

Evaluating a type inference implementation should involve several
strategies. I wish to maximise the intersection of 
the set of idiomatic Clojure forms and the set of forms whos type
can be successfully inferred by the type inference algorithm.

The Typed Racket type inference implementation is a combination
of two strategies: Local Type Inference and occurrence typing. 
Local Type Inference can be further described as an interaction between
two further algorithms,
\emph{local type argument synthesis} and \emph{bidirectional checking}\cite{Hosoya99howgood}.
I identify cases that test some components individually.

Some known limitations of Local Type Inference are explored by
Hosoya and Turner \cite{Hosoya99howgood}. They describe the strengths
of Local Type Inference, in particular they claim the limitations of
Local Type Inference are well defined and easy to understand. This claim
is also mirrored by Pierce and Turner \cite{Pierce:2000:LTI:345099.345100}.

One limitation described by both sets of authors is summarised by
Hosoya and Turner as ``hard to synthesise arguments'' \cite{Hosoya99howgood}.
It refers to situations where both the \emph{local type argument synthesis} and
\emph{bidirectional checking} algorithms could help synthesise the type. Because
both algorithms cannot run together, type synthesis fails.

Any example given by Hosoya et at. \cite{Hosoya99howgood} attempts to synthesises the type of an application of a polymorphic
function that takes an unannotated function as an argument.
For example \lstinline|(map (fn [x] x) [1 2 3])|,
where the type of the polymorphic function \emph{map} is 
\lstinline|(All [a b] [[a -> b] (List a) -> (List b)])|. Here, neither 
local type argument synthesis or bidirectional checking have sufficient type information
to infer types, so type inference fails. We should be able to help type inference
by providing parameter types to the anonymous function argument \lstinline|(map (fn [x : Integer] x) [1 2 3])|
or by instantiating \emph{map} with type arguments \lstinline|((inst map Integer Integer) (fn [x] x) [1 2 3])|.

\section{Implementation}

To test the type inference in Typed Racket,
I use the version of Typed Racket packaged with DrRacket 5.2, running on Ubuntu Linux 11.10.

I chose a function from the Clojure standard library to port to Typed Racket [Listing \ref{lst:isaCLJ}].
\lstinline|clojure.core/isa?| was chosen to test the occurrence
typing implementation of Typed Racket; there are several type predicates nested
in conjunctions that guard function applications. For example, the conjunction
of \lstinline|(class? parent)| and \lstinline|(class? child)| on line 5 ensure \lstinline|parent|
and \lstinline|child| are instances of \lstinline|Class| before calling the \lstinline|Class| instance
method \lstinline|isAssignableForm|
on line 6,
which expects a \lstinline|Class| argument. These ad-hoc type predicate applications must be recognised
and utilised by occurrence typing for type inference to succeed.

\begin{lstlisting}[caption=isa? Clojure Version, label=lst:isaCLJ]
(defn isa?
  ([child parent] (isa? global-hierarchy child parent))
  ([h child parent]
   (or (= child parent)
       (and (class? parent) (class? child)
            (. ^Class parent isAssignableFrom child))
       (contains? ((:ancestors h) child) parent)
       (and (class? child) 
            (some #(contains? ((:ancestors h) %) parent)
                  (supers child)))
       (and (vector? parent) (vector? child)
            (= (count parent) (count child))
            (loop [ret true i 0]
              (if (or (not ret) (= i (count parent)))
                ret
                (recur (isa? h (child i) (parent i)) (inc i))))))))
\end{lstlisting}

I also test the Local Type Inference implementation by reproducing
a ``hard to synthesise case'' \cite{Hosoya99howgood}.
I test the often cited common weakness, where type inference
fails because of applying a polymorphic function to an unannotated anonymous function
argument.

%Describe how you went about testing the hypothesis that you have stated
%in the introduction. What equipment did you use? Were there any parameter
%values you used? Give enough detail so that others can replicate your work.

\section{Results}

\subsection{Testing Typed Racket's Occurrence Typing}

I ported \lstinline|clojure.core/isa?| [Listing \ref{lst:isaCLJ}] to Typed Racket [Listing \ref{lst:isaTR}] by
first converting Clojure functions to Racket. If there was no
equivalent, such as the \lstinline|vector-of?| predicate, I define my own
and use it.

Note that \lstinline|isa?| takes two argument of type \lstinline|Any|
and returns a \lstinline|Boolean|. At line 17 is the ported version of
the example described in the previous section. Notice is it now statically
type checked, which means type inference succeeded. We know Typed Racket
recognised the predicate \lstinline|class?| because the call 
\lstinline|class-assignable-from-class| takes two \lstinline|class|
parameters.

I ran some tests with dummy \lstinline|class|s, which reinforced
the correctness of my code.


\begin{lstlisting}[caption=isa? Typed Racket Port, label=lst:isaTR]
(struct: class ...)
(struct: hierarchy ([ancestors : (HashTable class (Setof class))]))

(: class-assignable-from-class? (class class -> Boolean))
(: some-super (hierarchy Any class -> Boolean))

(define-predicate vector-of? (Vectorof Any))

(: isa? (case-lambda
          (Any Any -> Boolean)
          (hierarchy Any Any -> Boolean)))
(define isa?
  (case-lambda
    ((child parent) (isa? global-hierarchy child parent))
    ((h child parent)
     (or (equal? child parent)
         (and (class? parent) (class? child)
              (class-assignable-from-class? child parent))
         (and (class? child)
              (let ([hs (hierarchy-ancestors h)])
                (and (hash-has-key? hs child)
                     (let ([as (hash-ref hs child)])
                       (set-member? as parent)))))
         (and (class? child) (some-super h parent child))
         (and (vector-of? child) (vector-of? parent)
              (= (vector-length child) (vector-length parent))
              (let: loop : Boolean
                ([ret : Boolean #t] 
                 [i : Integer 0])
                (if (or (not ret) (= i (vector-length parent)))
                    ret
                    (loop (isa? h (vector-ref child i)
                                  (vector-ref parent i)) 
                          (+ 1 i)))))))))
\end{lstlisting}

\subsection{Testing Typed Racket's Local Type Inference}

I performed some verification tests on Typed Racket's Local Type Inference 
implementation. I chose a polymorphic function, \lstinline|map| [Listing \ref{lst:maptype}], and applied
an anonymous function.

\begin{lstlisting}[caption=Type of Typed Racket's map, label=lst:maptype]
(:print-type map)
;(All (c a b ...) 
;  (case-> 
;    ((a -> c) (Pairof a (Listof a)) 
;     -> (Pairof c (Listof c)))
;    ((a b ... b -> c) (Listof a) (Listof b) ... b 
;     -> (Listof c))))
\end{lstlisting}

Here I omit type parameters and parameter annotations [Listing \ref{lst:hard1}]. As expected,
type inference fails, so parameter \lstinline|a| is inferred to be
of type \lstinline|Any|, which is too general to be passed to
\lstinline|+|.

\begin{lstlisting}[caption=Hard to Synthesise Polymorphic Application without Type Arguments and Annotations - Typed Racket, label=lst:hard1]
(map (lambda (a) (+ 1 a 3)) (list 1 2 3))
; Type Checker: Expected Number, but got Any in: a
\end{lstlisting}

Here I give sufficient type arguments to instantiate \lstinline|map| [Listing \ref{lst:hard2}], but
this is, perhaps surprisingly, not enough information for Local Type Inference to
infer the correct type for \lstinline|a|.

\begin{lstlisting}[caption=Hard to Synthesise Polymorphic Application with Type Arguments without Annotations - Typed Racket, label=lst:hard2]
((inst map Integer Integer) (lambda (a) (+ 1 a 3)) (list 1 2 3))
; Type Checker: Expected Number, but got Any in: a
\end{lstlisting}

Finally, I show a version compatible with the type inferencer by
providing  the parameter types of the anonymous function argument [Listing \ref{lst:welltyped}].

\begin{lstlisting}[caption=Well Typed - Typed Racket, label=lst:welltyped]
(map (lambda: ([a : Integer]) (+ 1 a 3)) (list 1 2 3))
\end{lstlisting}


%Carefully outline each experiment: what was its purpose, and what
%precisely did you find? Sometimes it is easiest to display your
%results in tabular form, nut note that tables are also floating
%environments in \LaTeX\/, so they don't always end up where you want
%them to be!
%
%\begin{table}[htbp]
%  \caption{A very interesting table of tens.}
%  \begin{center}
%    \begin{tabular}{|l|c|l|} \hline
%      item1 & 10 & integer \\
%      item2 & 10.0 & float \\
%      item3 & `ten' & char \\
%      item4 & 1010 & binary \\
%      item5 & $10 + i0$ & complex \\
%      \hline
%    \end{tabular}
%  \end{center}
%\end{table}

\section{Conclusion}

I have evaluated several type inference techniques for applicability
to a Clojure type system. A type inference technique was considered
applicable if it helped type-check existing Clojure idioms without
adding redundant type annotations.

The type inference strategy is an essential part of any type system
design. When adding type inference to a language with
existing idioms, the search for well fitting type inference techniques
should be as exhaustive as possible, because the type inferencing must
fit with the idioms. If the wrong type inferencing strategy is employed,
it may force the idioms of the language to change. It is risky 
changing time honoured idioms, and everything should be done to avoid
that possibility.

The evaluation of existing type inference techniques consisted of two activities: simulating Clojure idioms
in other languages, and verify limitations to algorithms. I ported a
standard Clojure function to Typed Racket and ensured it was well typed.
The overall structure of the code did not change in the porting process.
I also tested the limitations of Local Type Inference by testing
small, contrived, forms in Typed Racket which I expected to fail. The first
failed without surprise, and the second failed after some uncertainty.

These results indicate that Typed Racket's type inference strategy \cite{SAMTH:dissertation} seems a good
fit for a Clojure type system, and is well worth a prototype. The stress test,
which required the porting effort, was testing occurrence typing and
how it might behave when applied to Clojure code. The results
were pleasing which implies occurrence typing could fit Clojure code as well as
Racket code. The successful small tests of known edge cases for Local Type Inference tells us
that a satisfying type system can be built in spite of the limitations of Local Type Inference\cite{Hosoya99howgood}.
This reinforces my decision to utilise Local Type Inference \cite{Pierce:2000:LTI:345099.345100}
over the more complicated Colored Type Inference \cite{Odersky:2001:CLT:373243.360207}.

This evaluation of type inference techniques is part of a larger project to add a static
type system to Clojure. Type inference is an incredibly important aspect of a type system in terms
of actual day-to-day use. The programmer will have type inference at the tip of their fingers
at every keystroke. How many keystrokes of frustrating redundancy would you tolerate 
before throwing in the towel? The type inference strategy must fit the language like a glove; then
you are in with a fighting chance.

%\section*{Acknowledgements}
%
%Acknowledge any help you might have had, either technical of financial.

\printbibliography

\end{document}
