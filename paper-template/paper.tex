%-----------------------------------------------------------------------------
%
%               Template for sigplanconf LaTeX Class
%
% Name:         sigplanconf-template.tex
%
% Purpose:      A template for sigplanconf.cls, which is a LaTeX 2e class
%               file for SIGPLAN conference proceedings.
%
% Guide:        Refer to "Author's Guide to the ACM SIGPLAN Class,"
%               sigplanconf-guide.pdf
%
% Author:       Paul C. Anagnostopoulos
%               Windfall Software
%               978 371-2316
%               paul@windfall.com
%
% Created:      15 February 2005
%
%-----------------------------------------------------------------------------


\documentclass[preprint]{sigplanconf}

% The following \documentclass options may be useful:
%
% 10pt          To set in 10-point type instead of 9-point.
% 11pt          To set in 11-point type instead of 9-point.
% authoryear    To obtain author/year citation style instead of numeric.

%\usepackage{amsmath}
\usepackage{listings}
\usepackage{biblatex}

\lstset{ %
  language=Lisp,                % choose the language of the code
  columns=fixed,basewidth=.5em,
  basicstyle=\small\ttfamily,       % the size of the fonts that are used for the code
  %numbers=left,                   % where to put the line-numbers
  %numberstyle=\small\ttfamily,      % the size of the fonts that are used for the line-numbers
  %stepnumber=1,                   % the step between two line-numbers. If it is 1 each line will be numbered
  %numbersep=5pt,                  % how far the line-numbers are from the code
  %backgroundcolor=\color{white},  % choose the background color. You must add \usepackage{color}
  %showspaces=false,               % show spaces adding particular underscores
  showstringspaces=false,         % underline spaces within strings
  %showtabs=false,                 % show tabs within strings adding particular underscores
  frame=single,           % adds a frame around the code
  %tabsize=2,          % sets default tabsize to 2 spaces
  captionpos=t,           % sets the caption-position to bottom
  breaklines=true,        % sets automatic line breaking
  breakatwhitespace=true,    % sets if automatic breaks should only happen at whitespace
  %escapeinside={\%*}{*)},          % if you want to add a comment within your code
}

\begin{document}

\conferenceinfo{WXYZ '05}{date, City.} 
\copyrightyear{2013} 
\copyrightdata{[to be supplied]} 

\titlebanner{banner above paper title}        % These are ignored unless
\preprintfooter{short description of paper}   % 'preprint' option specified.

\title{Title}
%\subtitle{Subtitle Text, if any}

\authorinfo{Ambrose Bonnaire-Sergeant}
           {University of Western Australia}
           {abonnairesergeant@gmail.com}

\maketitle

\begin{abstract}
\end{abstract}

\category{CR-number}{subcategory}{third-level}

% - what is the problem?
% - why is it interesting, challenging, important?
% - what have others done before?
% - what did you do?
% - how did you evaluate it?
% - what contribution have you made?
% - how would you continue this in the future?

\terms
%term1, term2

\keywords
%keyword1, keyword2

\section{Introduction}

\section{Static null-safety}

\section{Array covariance}

Array covariance refers to the subtyping relationship between different array types.
In Java, arrays are parameterised by one covariant parameter, such that
\lstinline|t1[] <: t2[]| for all \lstinline|t1 <: t2|. This is well known to be statically unsound,
because arrays are \emph{mutable}. 

The main advantages of covariant array types are increased code reuse
and more straightforward array types.
The disadvantage is loss of static type soundness.
Our approach avoids the pitfalls of array covariance
while facilitating code reuse, at the cost of slightly more complex types.

Listing \ref{lst:bad_java_array} demonstrates how array covariance allows
obviously ill-typed code to pass the static type checker.
By up-casting our \lstinline|Number[]| array to \lstinline|Object[]|,
the static type checker allows subtypes of \lstinline|Object|
to be written to it.
The problem occurs when writing to the up-casted array:
\lstinline|java.util.Observable| cannot be written to arrays with tag \lstinline|Number[]|.
The compiler instead inserts a runtime check which always results in a runtime error.

\begin{lstlisting}[language=java, label=lst:bad_java_array, caption=Statically unsound Java array manipulation]
public static void unsoundModify(Number[] ar) {
  Object[] bad_ar = (Object[])ar;
  bad_ar[0] = new java.util.Observable();
}
\end{lstlisting}

We want the type checker to reject obviously wrong code at compile time.
To achieve this, we use a \emph{bivariant} array type
\lstinline|(Array2 write read)| which has separate type parameters for
reading from and writing to the array. We refer to \lstinline|Array2| as bivariant (or invariant)
because it is contravariant in its first parameter and covariant in its second.

\begin{verb}
(Array2 S T) <: (Array2 S' T') 
  when S' <: S and T <: T'

(Array2 T T) is abbreviated (Array T)
\end{verb}

Listing \ref{lst:modify_array} uses \lstinline|Array2|
to track the types that can be read or written to an array.
We upcast an \lstinline|(Array Number)| (an array that can hold and write \lstinline|Number|s)
to \lstinline|(Array Number Object)| (an array that can hold \lstinline|Object|s and write
\lstinline|Number|s).
Most importantly, the upcast preserves which types we can write to the array.

The type assigned to \lstinline|aset|, the Clojure function for array modification, is slightly more interesting
(Listing \ref{lst:aset_ann}).
It takes an array, an array index and a value, and writes the value to the corresponding position of the array,
and returns the value written.

\begin{lstlisting}[language=lisp, label=lst:modify_array, caption=A type error with bivariant array types in core.typed]
(ann modify-array [(Array Number) -> nil])
(defn modify-array [ar]
  (let [ar2 (ann-form ar (Array Number Object))]
    (aset ar2 0 (new java.util.Observable))
    nil))
\end{lstlisting}

\begin{lstlisting}[language=Lisp, label=lst:aset_ann, caption=Type for aset]
(All [i o]
  [(Array2 i o) AnyInteger i -> o]))
\end{lstlisting}

Listing \ref{lst:array_error} shows the static type error that results when type checking Listing \ref{lst:modify_array}.
We retains enough information to determine we cannot write instances of
\lstinline|java.util.Observable| to \lstinline|(Array2 Number Object)|.

\begin{lstlisting}[language=lisp, label=lst:array_error, caption=Static type error for ill-typed array set]
#<Exception java.lang.Exception: 44: Polymorphic static method clojure.lang.RT/aset could not be applied to arguments:
Domains: 
  (Array2 i o) clojure.core.typed/AnyInteger i

Arguments:
  (Array2 java.lang.Number java.lang.Object) int java.util.Observable

in: (clojure.lang.RT/aset ar2 (clojure.lang.RT/intCast 0) (new java.util.Observable))>
\end{lstlisting}


%\appendix
%\section{Appendix Title}
%
%This is the text of the appendix, if you need one.
%
%\acks
%
%Acknowledgments, if needed.

% We recommend abbrvnat bibliography style.

%\bibliographystyle{abbrvnat}
%
%% The bibliography should be embedded for final submission.
%
%\begin{thebibliography}{}
%\softraggedright
%
%\bibitem[Smith et~al.(2009)Smith, Jones]{smith02}
%P. Q. Smith, and X. Y. Jones. ...reference text...
%
%\end{thebibliography}

\end{document}
